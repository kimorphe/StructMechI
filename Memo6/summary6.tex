\documentclass[10pt,a4j]{jarticle}
%\usepackage{graphicx,wrapfig}
\usepackage{graphicx,amsmath}
\setlength{\topmargin}{-1.5cm}
\setlength{\textwidth}{16.5cm}
\setlength{\textheight}{25.2cm}
\newlength{\minitwocolumn}
\setlength{\minitwocolumn}{0.5\textwidth}
\addtolength{\minitwocolumn}{-\columnsep}
%\addtolength{\baselineskip}{-0.1\baselineskip}
%
\def\Mmaru#1{{\ooalign{\hfil#1\/\hfil\crcr
\raise.167ex\hbox{\mathhexbox 20D}}}}
%
\begin{document}
\newcommand{\fat}[1]{\mbox{\boldmath $#1$}}
\newcommand{\D}{\partial}
\newcommand{\w}{\omega}
\newcommand{\ga}{\alpha}
\newcommand{\gb}{\beta}
\newcommand{\gx}{\xi}
\newcommand{\gz}{\zeta}
\newcommand{\vhat}[1]{\hat{\fat{#1}}}
\newcommand{\spc}{\vspace{0.7\baselineskip}}
\newcommand{\halfspc}{\vspace{0.3\baselineskip}}
\bibliographystyle{unsrt}
%\pagestyle{empty}
\newcommand{\twofig}[2]
 {
   \begin{figure}
     \begin{minipage}[t]{\minitwocolumn}
         \begin{center}   #1
         \end{center}
     \end{minipage}
         \hspace{\columnsep}
     \begin{minipage}[t]{\minitwocolumn}
         \begin{center} #2
         \end{center}
     \end{minipage}
   \end{figure}
 }
%%%%%%%%%%%%%%%%%%%%%%%%%%%%%%%%%
%\vspace*{\baselineskip}
\begin{center}
	{\Large \bf 2019年度 構造力学I及び演習B 講義内容のまとめ3} \\
\end{center}
%%%%%%%%%%%%%%%%%%%%%%%%%%%%%%%%%%%%%%%%%%%%%%%%%%%%%%%%%%%%%%%%
%%%%%%%%%%%%%%%%%%%%%%%%%%%%%%%%%%%%%%%%%%%%%%%%%%%%%%%%%%%%%%%%
\section{静定構造,不静定構造}
%梁に発生する断面力は釣り合い条件を満足する必要がある.
たわみの支配微分方程式$(EIv'')''=q$は,曲げモーメントとせん断力の釣り合い式:
\begin{equation}
	M'-Q=0, \ \ Q'+q=0 
	\label{eqn:equiv_MQ}
\end{equation}
と,剛性関係式:
\begin{equation}
	M=EI\kappa = -EIv''
	\label{eqn:M_kpp}
\end{equation}
から導かれたものである.従って,支配微分方程式を満足するように求められたたわみ$v$を
微分して得られる断面力は,必然的に釣り合い条件を満足する.
一般に,断面力は変形条件と力の釣り合いの両方を考慮しなければ決定することは
できない.しかしながら,支持条件によっては,力とモーメントの釣り合い条件だけから
断面力を決定できる場合がある.そのような構造を,"{\bf 静定構造(statically determinate structure)
}"と呼ぶ.静定でない構造,すなわち,変形と釣り合いの両方を考慮して初めて断面力が
決定できる構造を"{\bf 不静定構造(statically indeterminate structure)"}と呼ぶ.
静定構造では,たわみを経由せず釣り合い条件から直接断面力を決定する方が,
計算が容易になることが多い.
%%%%%%%%%%%%%%%%%%%%%%%%%%%%%%%%%
\section{合力と合モーメントの計算}
釣り合い条件を使った梁の断面力計算において,複数の集中荷重や分布荷重の
合力と合モーメントを求めることがしばしば必要となる.そこで,梁に鉛直に作用する
力について,合力と合モーメントを計算する方法を以下に述べる.
\subsection{集中荷重の組}
図\ref{fig:fig9_1}-(a)に示すように,$n$固の集中荷重が作用する場合を考える.ここで,
$n$個の荷重のうち第$i$番目の荷重の作用点を$x_i$, 大きさを$F_i$とし,このような
荷重の組を$\left\{ \left(x_i, F_i\right), i=1,\dots ,n\right\}$あるいは単に$\{F_i\}$と表す.
このとき,$\{F_i\}$の合力$F$と$x=a$に関する合モーメント$M_a$は,ぞれぞれ以下の式に従って計算
することができる.
\begin{eqnarray}
	F &=& \sum_{i=1}^n F_i 
	\label{eqn:Ftot}
	\\
	M_a &=& \sum_{i=1}^n (x_i-a)F_i 
	\label{eqn:Mtot}
\end{eqnarray}
なお,ここではモーメントの正方向を時計回りの方向としている.
$x=b$に関する合モーメントを$M_b$とすれば,式(\ref{eqn:Ftot})と(\ref{eqn:Mtot})より
\begin{eqnarray}
	M_b &=& \sum_{i=1}^n (x_i-b)F_i  \nonumber \\
	    &=& \sum_{i=1}^n \left\{ (x_i-a)+(a-b)\right\}F_i  \nonumber \\
	    &=& \sum_{i=1}^n (x_i-a)F_i +(a-b)\sum_{i=1}^n F_i  \nonumber \\
	    &=& M_a +(a-b)F 
	\label{eqn:Ma2b}
\end{eqnarray}
となる. ここで,原点$x=0$に関する合モーメントを$M_0$とすれば,
\begin{equation}
	M_0= \sum_{i=1}^n x_i F_i 
	\label{eqn:M0}
\end{equation}
だから,式(\ref{eqn:Mtot})は
\begin{equation}
	M_a = M_0 -aF 
	\label{eqn:}
\end{equation}
と表すことができる.いま,$M_a=0$となるような特別な位置$a$を,特に$a=\bar x$と
書くことにすれば,$\bar x$は
\begin{equation}
	\bar{x} = \frac{M_0}{F}, \ \ (F\neq 0)
	\label{eqn:xbar}
\end{equation}
で与えられる. また,式(\ref{eqn:Ma2b})において$a=\bar x$とすれば,任意の$b$に対して
\begin{equation}
	M_b= \left( \bar x - b\right) F
	\label{eqn:F2M}
\end{equation}
となる.式(\ref{eqn:F2M})は,$\{ (x_i, F_i),\, i=1,\dots ,n\}$に対して合力$F$と
位置$\bar x$を一度求めれば, 全荷重についての和をその都度計算することなく
任意の位置$x=b$に関する合モーメント$M_b$が得られることを意味している.
このことは,$\{F_i\}$の合モーメントが,あたかも$x=\bar x$に大きさ$F$の集中荷重が
作用していると考えて計算ができることを示しており(図\ref{fig:fig9_1}-(c)).
この意味で,図\ref{fig:fig9_1}-(c)の集中荷重と$\{ F_i\}$は,合力と合モーメントの
計算において等価とみなすことができる.
\subsection{分布荷重の合力,合モーメント}
単位長さ辺りの鉛直力$q(x)$が与えられたとき,その合力$F$と$x=a$に関する合モーメント$M_a$は,
それぞれ次の式で与えられる.
\begin{eqnarray}
	F &= & \int q(x) dx 
	\label{eqn:Ftot_q}
	\\
	M_a &= & \int (x-a)q(x) dx 
	\label{eqn:Mtot_q}
\end{eqnarray}
また,$x=b$に関する合モーメント$M_b$と$M_a$の関係は,
\begin{equation}
	M_b = \int (x-b)q(x)dx = \int (x-a)q(x) dx + \int (a-b)q(x)dx = M_a+(a-b) F
	\label{eqn:Mb_q}
\end{equation}
となり,集中荷重の場合と同じ関係が成り立つことが示される.
そこで,$a=\bar x$に関する合モーメント$M_{\bar x}$がゼロとなるよう,
\begin{equation}
	\bar x = \frac{M_0}{F}
\end{equation}
を定め,式(\ref{eqn:Mb_q})で$a=\bar x$とおけば,
\begin{equation}
	M_b=\left(\bar x - b\right)F
	\label{eqn:Mb_F}
\end{equation}
が得られる.
その結果,分布荷重$q(x)$と$x=\bar x$に作用する大きさ$F$の集中荷重は,合モーメントの計算において等価
であることが示される.式(\ref{eqn:Mb_F})を用いれば,積分計算を繰り返し行うこと無く,
任意の基準点についての合モーメントを得ることができる.
\subsection{力とモーメントの釣り合い}
物体に力(集中荷重や分布力)が作用した状態で移動も回転もしないとき,物体は釣り合い状態にある.
このとき,物体に作用する力の合力と合モーメントはゼロでなければならない.
すなわち,任意の$a$について
\begin{eqnarray}
	F &= &  0 
	\label{eqn:F_equib}
	\\
	M_a &= &  
	%\sum_{i=1}^n(x_i-a)F_i =
	0 
	\label{eqn:M_equib}
\end{eqnarray}
が満足されなければならない.なお,$F=0$のとき,式(\ref{eqn:Ma2b})と(\ref{eqn:Mb_q})より任意の$a$と$b$に対して
$M_a=M_b$となり,合モーメントはどの位置を基準として計算しても結果は変わらない.
従って,力とモーメントの釣り合い式(\ref{eqn:F_equib}),(\ref{eqn:M_equib})を利用する
とき,モーメントの基準点$x=a$をどこにおいてもよく,計算上都合の良い位置に取ることができる.
\subsection{偶力}
$\{F_i\}$の合力$F$がゼロとのき,式(\ref{eqn:xbar})で$\bar x$を定義することはできず,
$\{F_i\}$を単一の荷重で代表させることはできない.しかしながら,$F_i >0$となる荷重の
合力と$x=0$に関する合モーメントを$F^+, M_0^+$, $F_i<0$である荷重の合力と合モーメントを同様に
$F^-, M_0^-$とすれば,$F^+>0,\; F^-<0$だから,各々の作用点$\bar x^+$と$\bar x^-$は次の
ように定めることができる.
\begin{equation}
	\bar x ^\pm = \frac{M_0^\pm }{F ^\pm} 
	\label{eqn:xbar_pm}
\end{equation}
ただし,
\begin{eqnarray}
	F^+ = \sum _{ F_i > 0 } F_i, & & F^- = \sum _{F_i < 0 } F_i 
	\label{eqn:F_pm}
	\\
	M_0^+ = \sum _{F_i > 0 } x_i F_i, & & M_0^- = \sum _{F_i < 0 } x_iF_i 
	\label{eqn:M_pm}
\end{eqnarray}
である.このことは, $\{F_i\}$を単一の集中荷重に置き換えることはできないが,
位置$x=\bar x ^\pm$に加えられた大きさ$F^\pm$の2つの集中荷重で代用できることを意味する.
これら2つ荷重の組は,互いに大きさが等しく,向きが反対になっており偶力(couple force)と呼ばれる.
偶力は,$F=F^++F^-=0, M_0=M_0^+ + M_0^-\neq 0$だから,力の大きさはゼロでモーメントだけを持つ.
合力がゼロであることから,合モーメントはどの位置について計算をしても同じで,偶力固有の
作用点位置を定めることはできない(定める必要がない). 
分布力$q(x)$についても,合力がゼロの場合,一つの集中荷重に置き換えることはできないが,
偶力に換算することはできる.このことを理解するには,
$F^\pm, M_0^\pm$を以下のようにおき,上と同じ議論を行えば良い.
\begin{eqnarray}
	F^\pm &=& \pm \int \left< \pm q(x) \right>dx
	\label{eqn:Fpm_q} \\
	M_0^\pm &=& \pm \int \left< \pm xq(x) \right>dx
	\label{eqn:Mpm_q}
\end{eqnarray}
なお,式(\ref{eqn:Fpm_q}),(\ref{eqn:Mpm_q})の被積分関数$\pm q(x)$につた括弧$\left< \cdot \right>$は,
マッコーレーの括弧を意味する.
%--------------------
\begin{figure}[h]
	\begin{center}
	\includegraphics[width=0.7\linewidth]{fig9_1.eps} 
	\end{center}
	\caption{
		(a)集中荷重とその作用点の座標. (b) 分布荷重とその作用範囲.
		(c)位置$\bar x$に作用する大きさ$F$の集中荷重. 
		(d)位置$\bar x^\pm$に作用する大きさ$F^\pm$の力($F^++F^-=0$). 
		複数の集中荷重(a)や分布荷重(b)は,(c)あるいは(d)のような力に
		換算することができる.
	} 
	\label{fig:fig9_1}
\end{figure}
\subsection{例題}
\begin{enumerate}
\item
図\ref{fig:fig9_4}-(1)に示すような2つの集中荷重について,これらと等価な集中荷重を求める.
力の向きは下向きを,モーメントは時計回りの方向を正とすると,合力$F$, $x=0$に関する
合モーメントは
\begin{equation}
	F=F_1+F_2, \ \ M_0=F_1\times 0 + F_2 \times l =F_2l
	\label{eqn:ans1_1}
\end{equation}
である.よって,この場合
\begin{equation}
	\bar x = \frac{M_0}{F}=\frac{F_2l}{F}= \frac{F_2}{F_1+F_2}l
	\label{eqn:xbar1_1}
\end{equation}
となることから,与えられた2つの荷重と等価な荷重は,$x=\bar x = \frac{F_2}{F_1+F_2}l$,
すなわち2つの荷重$F_1,F_2$の作用点位置を$F_2:F_1$に内分する位置に加えられた,
大きさ$F_1+F_2$の集中荷重である.
この結果を用いて,$x=l$に関する合モーメント$M_l$を計算すると,
\begin{equation}
	M_l=(\bar x- l ) F = -\frac{F_1}{F_1+F_2}l \times (F_1+F_2)= -F_1l
	\label{eqn:Ml1_1}
\end{equation}
となる.同じ結果は,式(\ref{eqn:Mtot})の右辺を直接計算して得ることもできる.
\item
図\ref{fig:fig9_5}-(1)に示すような等分布荷重に等価な集中荷重を求め,その結果を利用して
$x=\frac{l}{2}$と$x=l$に関する$q(x)$の合モーメントを求める.
\begin{eqnarray}
	F & =& \int _0^l q(x)dx= q_0l
	\label{eqn:}
	\\
	M_0 & =& \int _0^l xq(x)dx= \frac{q_0l^2}{2}
\end{eqnarray}
よって,
\begin{equation}
	\bar x= \frac{\frac{q_0l^2}{2}}{q_0l}=\frac{l}{2}
	\label{eqn:xbar_rec}
\end{equation}
となり,大きさ$q_0$の等分布荷重は,その中央に作用する大きさ$q_0l$の集中荷重と等価であることが分かる.
また,$M_{\frac{l}{2}}$と$M_{l}$は,
\begin{equation}
	M_{\frac{l}{2}}=\left(\frac{l}{2}-\frac{l}{2}\right)\times q_0l = 0 ,\ \ 
	M_{l}=\left(\frac{l}{2}-l\right)\times q_0l = -\frac{q_0l^2}{2}
	\label{eqn:}
\end{equation}
と求めることができる.
\item
図\ref{fig:fig9_5}-(2)に示すような三角形分布荷重と等価な集中荷重を求め,その結果を利用して
$x=\frac{l}{2}$と$x=l$に関する$q(x)$の合モーメントを求める.
与えられた分布荷重は$q(x)=q_0\frac{x}{l}$と表されるので,
\begin{eqnarray}
	F&= & \int _0^l q(x)dx = \int_0^l q_0 \frac{x}{l}dx=\frac{q_0l}{2}
	\label{eqn:}
	\\
	M_0&= & \int _0^l xq(x)dx = \int_0^l q_0 \frac{x^2}{l}dx=\frac{q_0l^2}{3}
	\label{eqn:}
\end{eqnarray}
となり,三角形分布荷重と等価な集中荷重の作用点位置は
\begin{equation}
	x=\bar x = \frac{M_0}{F}=\frac{2}{3}l
	\label{eqn:xbar_tri}
\end{equation}
すなわち,三角形の図心位置になる.これより,$M_{\frac{l}{2}}$と$M_l$は,それぞれ
\begin{equation}
	M_{\frac{l}{2}}=\left(\frac{2l}{3}-\frac{l}{2}\right) \times \frac{q_0l}{2} =\frac{q_0l^2}{12}, \ \ 
	M_l = \left(\frac{2l}{3}-l \right) \times \frac{q_0l}{2} =-\frac{q_0l^2}{6}
	\label{eqn:}
\end{equation}
と求めることができる.
\end{enumerate}
\begin{figure}[h]
	\begin{center}
	\includegraphics[width=1.\linewidth]{fig9_4.eps} 
	\end{center}
	\caption{
		鉛直方向に作用する集中荷重の組.
	} 
	\label{fig:fig9_4}
\end{figure}
\begin{figure}[h]
	\begin{center}
	\includegraphics[width=1.\linewidth]{fig9_5.eps} 
	\end{center}
	\caption{
		鉛直方向に作用する分布力$q(x)$, (1)一定値,(2)直線,(3)$p$次式.
	} 
	\label{fig:fig9_5}
\end{figure}
\subsection{問題}
\begin{enumerate}
\item
	図\ref{fig:fig9_4}-(2)に示すような荷重の組と等価な集中荷重の作用点位置を求めよ.
	また,その結果を利用して,$x=a,2a,3a$に関する合モーメントを求めよ.
\item
	図\ref{fig:fig9_4}-(3)に示すような荷重の組と等価な集中荷重の作用点位置を求めよ.
	また,その結果を利用して,$x=a$と$2a$に関する合モーメントを求めよ.
\item
	図\ref{fig:fig9_5}-(3)に示す分布荷重と等価な集中荷重の作用点位置を求めよ.
		また,その結果を利用して,$x=\frac{l}{2}$と$x=l$に関する合モーメントを求めよ.
\end{enumerate}
%--------------------
\section{静定梁の支点反力と断面力の求め方}
図\ref{fig:fig9_0}に示す4つの構造は,全て静定構造であり,力とモーメントの釣り合い条件から
支点反力と断面力を決定することができる.
このことを,図\ref{fig:fig9_0}-(1)と(2)を例として具体的に見る.
\subsection{例題}
\begin{enumerate}
\item
図\ref{fig:fig9_0}-(1)に示す単純支持された梁の,支点反力と断面力分布を求める.
この梁の支点Aは,ピン支持されているため,水平方向と鉛直方向に変位が拘束されている.
従って,梁は支点Aから水平反力$H_A$と鉛直反力が$R_A$を受ける.この正方向を図\ref{fig:fig9_2}-(a)
のようにとる.一方,支点Bはローラー支持のため,鉛直変位だけが拘束される結果,
梁は支点から鉛直方向の反力$R_B$を受ける.$R_B$も$R_A$と同様,鉛直上向きを正方向にとる.
ここで,梁全体の力の釣り合い条件を立てると,
\begin{equation}
	H_A=0(水平方向), \ \ 
	F-R_A-R_B=0(鉛直方向)
	\label{eqn:}
\end{equation}
となる.さらに,支点Aに関するモーメントの釣り合い式は
\begin{equation}
	F\times \frac{l}{2}-R_B\times l =0
	\label{eqn:}
\end{equation}
となることから,全ての支点反力が
\begin{equation}
	H_A=0, \ \ R_A=\frac{F}{2}, \ \ R_B=\frac{F}{2}
	\label{eqn:}
\end{equation}
と求まる.次に,断面力分布を求めるために,図\ref{fig:fig9_2}-(a)に示す
a-a'の位置で梁を仮想的に切断して自由物体図を描くと同図(b)のようになる.
このような部分構造について水平方向と鉛直方向の力の釣り合い式を立てると,
\begin{equation}
	H_A+N=0, \ \ Q-R_A=0
	\label{eqn:}
\end{equation}
となる.また,a-a'断面に関するモーメントの釣り合い式は
\begin{equation}
	M-R_A x=0
	\label{eqn:}
\end{equation}
であることから,$N,\, Q,\, M$は
\begin{equation}
	N(x)=0,\ \ Q(x)=R_A=\frac{F}{2}, \ \ 
	M(x)= R_Ax =\frac{F}{2}x
	\label{eqn:}
\end{equation}
となる.断面a-a'の位置は,$0< x< \frac{l}{2}$の範囲で任意だから,
以上で区間ACにおける断面力分が求められたことになる.
次に,右半分の区間CBにおける断面力を分布を求める.そのために,図\ref{fig:fig9_2}-(a)の
b-b'の位置で梁を切断して自由物体図を描くと同図の(c)と(d)が得られる.
釣り合い条件式を立てる際,系に作用する荷重の数が少ない方が計算量が少なくて済む.
そこで,図\ref{fig:fig9_2}-(d)の自由物体図について釣り合い条件式を立てると,
\begin{equation}
	N=0, \ \ Q+R_B=0, \ \ M-R_By=0
	\label{eqn:}
\end{equation}
となる.ここで,第3式はb-b'断面に関するモーメントの釣り合い条件である.
これらの式より,右半分の区間$(0<y<\frac{l}{2})$における断面力分布
\begin{equation}
	N=0, \ \ Q=-R_B=-\frac{F}{2}, \ \ M=R_By =\frac{F}{2}y
	\label{eqn:}
\end{equation}
と,が得られる.以上の結果を,断面力図として示すと,図\ref{fig:fig9_6}-(a)のようになる.
%%%%%%%%%%%%%%%%%%%%%%%%%%%%%%%%%
\item
図\ref{fig:fig9_0}-(2)に示す, 片端固定,片端自由の梁(片持梁)について,
支点反力と断面力分布を求める.固定端Aでは,垂直,水平,回転,全ての変位が拘束
されていることから,点Aで梁は鉛直反力$R_A$,水平反力$H_A$, 曲げモーメント$M_A$を受ける.
これらの反力の正方向を図\ref{fig:fig9_3}-(a)のようにとる.
一方,支点Bは自由端のため変形に対する拘束は無く,反力も発生しない.
ここで,区間CBに作用する分布力は,図\ref{fig:fig9_3}-(a')のような集中荷重と等価である
ことに注意して,梁全体の力の釣り合いと, 点Aに関するモーメントの釣り合い式を立てれば,
全ての反力を次にように決定することができる.
\begin{eqnarray}
	H_A=0 & &  \\
	-R_A+\frac{q_0l}{2}=0 &\Rightarrow & R_A=\frac{q_0l}{2} \\
	M_A+\frac{q_0l}{2}\times \left(\frac{l}{2}+\frac{l}{4}\right)=0
	& \Rightarrow & M_A=-\frac{3q_0l^2}{8}
\end{eqnarray}
次に,区間ABにおける断面力を求めるために,図\ref{fig:fig9_3}-(a)に示す
a-a'の位置で梁を切断し,その結果得られる部分構造の自由物体図を描けば,同図の
(b)のようになる.この図に従い,力とモーメントの釣り合い式を立てると,
\begin{eqnarray}
	H_A+N=0 
	& \Rightarrow & 
	N=-H_A=0
	\label{eqn:} \\
	Q-R_A=0 
	& \Rightarrow & 
	Q=R_A=\frac{q_0l}{2}
	\label{eqn:} \\
	M-M_A-R_A\times x =0 
	& \Rightarrow & 
	M=R_Ax+M_A= \frac{q_0l}{2}x-\frac{3}{8}q_0l^2 
	\label{eqn:}
\end{eqnarray}
と,区間ACにおいて全ての断面力が求まる.

最後に区間CBにおける断面力分布を求めるために,図\ref{fig:fig9_3}-(a)に示した
b-b'の位置で構造を切断して自由物体図を描く.その結果は,同図の(c)のように
なるが,釣り合い条件式を書き下すにあたり,矩形分布荷重を(c')の図のような
集中荷重に置き換えておくと都合が良い.このとき,
力とモーメントの釣り合い式と,そこから求められる断面力分布は以下のようになる.
\begin{eqnarray}
	N=0 & & 
	\label{eqn:}
	\\
	q_0y -Q=0 & \Rightarrow & Q=q_0y 
	\label{eqn:}
	\\
	M+q_0y\times \frac{y}{2} =0 & \Rightarrow & M=-\frac{q_0y^2}{2}
	\label{eqn:}
\end{eqnarray}
以上の結果から断面力図を描くと,図\ref{fig:fig9_6}-(b)のようになる.
\end{enumerate}
\begin{figure}
	\begin{center}
	\includegraphics[width=0.85\linewidth]{fig9_0.eps} 
	\end{center}
	\caption{
		鉛直方向の荷重を受ける静定梁.
	} 
	\label{fig:fig9_0}
\end{figure}
\begin{figure}
	\begin{center}
	\includegraphics[width=0.45\linewidth]{fig9_2.eps} 
	\end{center}
	\caption{
		支点反力と断面力計算のための自由物体図.
		支間中央に集中荷重を受ける単純支持梁の場合.
		(a)支点反力の正方向, (b)区間ACにおける断面力,
		(c),(d)区間BCにおける断面力. 
	} 
	\label{fig:fig9_2}
\end{figure}
\begin{figure}
	\begin{center}
	\includegraphics[width=0.5\linewidth]{fig9_3.eps} 
	\end{center}
	\caption{
		支点反力と断面力計算のための自由物体図.
		矩形分布荷重を受ける片持梁の場合.
		(a)支点反力の正方向, (b)区間ACにおける断面力,
		(c)区間BCにおける断面力. 
	} 
	\label{fig:fig9_3}
\end{figure}
\begin{figure}
	\begin{center}
	\includegraphics[width=0.85\linewidth]{fig9_6.eps} 
	\end{center}
	\caption{
		断面力図. 
	} 
	\label{fig:fig9_6}
\end{figure}
\subsection{問題}
図\ref{fig:fig9_0}の(3)と(4)の構造について,支点反力と断面力分布を求め,
断面力図(せん断力図と曲げモーメント図)を描け. 
%%%%%%
\end{document}
