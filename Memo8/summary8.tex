\documentclass[10pt,a4j]{jarticle}
%\usepackage{graphicx,wrapfig}
\usepackage{graphicx,amsmath}
\setlength{\topmargin}{-1.5cm}
\setlength{\textwidth}{16.5cm}
\setlength{\textheight}{25.2cm}
\newlength{\minitwocolumn}
\setlength{\minitwocolumn}{0.5\textwidth}
\addtolength{\minitwocolumn}{-\columnsep}
%\addtolength{\baselineskip}{-0.1\baselineskip}
%
\def\Mmaru#1{{\ooalign{\hfil#1\/\hfil\crcr
\raise.167ex\hbox{\mathhexbox 20D}}}}
%
\begin{document}
\newcommand{\fat}[1]{\mbox{\boldmath $#1$}}
\newcommand{\D}{\partial}
\newcommand{\w}{\omega}
\newcommand{\ga}{\alpha}
\newcommand{\gb}{\beta}
\newcommand{\gx}{\xi}
\newcommand{\gz}{\zeta}
\newcommand{\vhat}[1]{\hat{\fat{#1}}}
\newcommand{\spc}{\vspace{0.7\baselineskip}}
\newcommand{\halfspc}{\vspace{0.3\baselineskip}}
\bibliographystyle{unsrt}
%\pagestyle{empty}
\newcommand{\twofig}[2]
 {
   \begin{figure}
     \begin{minipage}[t]{\minitwocolumn}
         \begin{center}   #1
         \end{center}
     \end{minipage}
         \hspace{\columnsep}
     \begin{minipage}[t]{\minitwocolumn}
         \begin{center} #2
         \end{center}
     \end{minipage}
   \end{figure}
 }
%%%%%%%%%%%%%%%%%%%%%%%%%%%%%%%%%
%\vspace*{\baselineskip}
\begin{center}
	{\Large \bf 2018年度 構造力学I及び演習B 講義内容のまとめ5} \\
\end{center}
%%%%%%%%%%%%%%%%%%%%%%%%%%%%%%%%%%%%%%%%%%%%%%%%%%%%%%%%%%%%%%%%
%%%%%%%%%%%%%%%%%%%%%%%%%%%%%%%%%%%%%%%%%%%%%%%%%%%%%%%%%%%%%%%%
\section{断面係数}
たわみの支配微分方程式に含まれる曲げ剛性$EI$のうち,ヤング率$E$は
実験によって求めるべき物性値である.一方,断面2次モーメント$I$は,
断面の幾何学的な形状によって決まる断面係数の一つである.
なお,ここでいう断面2次モーメントとは,
断面$S$の水平方向の中立軸$z$からの距離を$y$として,
\begin{equation}
	I=\int_S y^2 dS
	\label{eqn:Iz_xy}
\end{equation}
で与えられる(図\ref{fig:fig11_1}).
従ってより正確には,$I$を"$S$の中立軸($z$軸)に関する断面2次モーメント"と呼ぶ.
\subsection{中立軸と断面1次モーメント}
断面2次モーメント$I$の計算に先立ち,$S$の中立軸($z$軸)位置を決定する必要がある.
%中立面は,梁に軸力が作用せず曲げのみが生じるとき,伸びも縮みも生じない面が当初(変形前)
%に占める平面として定義されていた.
そのために,軸力$N$に関する条件を用いる.梁長手方向に$x$軸をとり,この方向の直応力を
$\sigma_{xx}$と書くとき,軸力$N$と曲げモーメント$M$は次の式で定義される.
\begin{eqnarray}
	N &= & \int_S \sigma_{xx}dS 
	\label{eqn:def_N}
	\\
	M &= & \int_S y\sigma_{xx}dS 
	\label{eqn:def_M}
\end{eqnarray}
また,梁に軸力が働かず($N\equiv 0$)曲げ変形のみが生じるとき,
$x=$一定の断面における直応力分布が次式で与えられることは,既に学んだ通りである.
\begin{equation}
	\sigma_{xx}=\frac{M}{I}y
	\label{eqn:sig_xx}
\end{equation}
このとき,軸力ゼロの条件は
\begin{equation}
	N =  \int_S \sigma_{xx}dydz = 0 
	\label{eqn:N_is_0}
\end{equation}
と表すことができる.ただし,式(\ref{eqn:N_is_0})は中立軸位置($y$軸の原点)が
既知でないことから,これ以上は積分計算を進めることができない.
そこで,図\ref{fig:fig11_1}のような既知の$YZ$直交座標系を新たに導入する.
ここで,$YZ$座標系の原点は任意とし,$Y$軸と$Z$軸は,それぞれ
$y$軸,$z$軸と同じ方向を向くように取る.
また,$z$軸は$Z$軸から$\bar{Y}$だけ離れているとすれば,2つの座標系の関係は
\begin{equation}
	y=Y-\bar Y, \ \ Z=z
	\label{eqn:y_shift}
\end{equation}
となる.これを,式(\ref{eqn:N_is_0})に用いることで
\begin{equation}
	N=\frac{M}{I}\int_S \left(Y-\bar Y \right)dS=0 
	\ \ \Rightarrow \ \
	\int_S Y dS =\bar Y \left| S \right|
	\label{eqn:as_Nis0}
\end{equation}
の関係が得られる.式(\ref{eqn:as_Nis0})より,
\begin{equation}
	\bar Y
	= \frac{\int_S YdS }{\int dS}
	= \frac{G_Z}{\left| S \right|}
	\label{eqn:Ybar}
\end{equation}
となり,中立軸位置が$Z$座標系において$Z=\bar{Y}$にあり,$\bar{Y}$が
式(\ref{eqn:Ybar})で与えられることが示される.ただし,$\left| S\right|$は$S$の
面積:
\begin{equation}
	\left| S \right| = \int_S dS
	\label{eqn:area}
\end{equation}
を表し,$G_Z$は
\begin{equation}
	G_Z=\int_S YdS 
	\label{eqn:GZ}
\end{equation}
と定義され,断面$S$の$Z$軸に関する断面1次モーメントと呼ばれる.
以上より,中立軸位置を求めるには,断面積と断面1次モーメントを求めればよいことが分かる.
%--------------------
\begin{figure}[h]
	\begin{center}
	\includegraphics[width=0.4\linewidth]{./fig11_1.eps} 
	\end{center}
	\caption{
		断面係数計算のために定めた$ZY$座標系と中立軸位置$Y=\bar Y$.
	} 
	\label{fig:fig11_1}
\end{figure}
\section{断面2次モーメント}
以上の方法によって中立軸位置が決まれば,式(\ref{eqn:Iz_xy})の積分をそのまま
$yz$座標系で実行して断面2次モーメント$I$を求めることができる.
一方,引続き$YZ$座標系で$I$の計算を行う場合は,$y=Y-\bar{Y}$であることを利用して
式(\ref{eqn:Iz_xy})を次のように書き直す.
\begin{eqnarray}
	I &= &
	\int_S y^2 dS 
	\nonumber \\
	 &= &
	\int_S \left( Y-\bar{Y}\right)^2 dS 
	\nonumber \\
	 &= &
	\int_S Y^2dS -2\bar{Y}\int_S Y dS -\bar Y^2 \int_S dS
	\nonumber \\
	 &= &
	 I_Z-2\bar{Y}G_Z+\bar{Y}^2\left| S \right|
	\nonumber \\
	 &= &
	 I_Z-2\bar{Y}\times \bar{Y}\left| S \right|+\bar{Y}^2 \left| S \right|
	\nonumber \\
	 &= &
	 I_Z-\bar{Y}^2\left| S \right|
	\label{eqn:IZ2I}
\end{eqnarray}
ここで$I_Z$は,
\begin{equation}
	I_Z=\int_S Y^2 dS
	\label{eqn:def_IZ}
\end{equation}
であり,$S$の$Z$軸に関する断面2次モーメントと呼ばれる.
式(\ref{eqn:def_IZ})のように,基準となる軸($Z$軸)を添字で示すならば,
$z$軸に関する断面2次モーメントである式(\ref{eqn:Iz_xy})の$I$は,
$I$でなく$I_z$と書くことが自然である.
そこで,以後は$I=I_z$と書き,式(\ref{eqn:IZ2I})で得られた関係も,
\begin{equation}
	I_z=I_Z-\bar Y^2 \left| S \right|
	\label{eqn:IZ2Iz}
\end{equation}
と表すことにする.式(\ref{eqn:IZ2Iz})は,$YZ$座標系で計算結果から$I_z$を
求めるための式として用いることができる。式(\ref{eqn:IZ2Iz})の右辺に現れる$\bar{Y}$と
$\left| S\right|$は, $I_z$を$yz$座標系で計算する際にも必要とされる.
従って,$I_z$を計算する際に必要とされる計算に関して,2つの座標系の間での違いは,
$yz$座標系で積分を行い$I_z$を求めるか,$YZ$座標系で$I_Z$を求めるかの違いだけである.
$YZ$座標系は,断面1次モーメント$G_Z$を計算する上で最も都合の座標を選ぶことを考えると,
$I_Z$の計算が$I_z$を直接求めるより煩雑になることはなく,このことから,式(\ref{eqn:IZ2Iz})
に基づいて$I_z$を得ることが得策であると言える.

式(\ref{eqn:IZ2Iz})を変形すれば,
\begin{equation}
	I_Z=I_z+\bar Y^2 \left| S \right|
	\label{eqn:Iz2IZ}
\end{equation}
を得る.これは,断面$S$の中立軸($z$軸)位置が既知のとき,任意の位置にとった別の軸
($Z$軸)に関する断面2次モーメント$I_Z$を求めるために役立つ関係である.
また,式(\ref{eqn:Iz2IZ})において,$\bar{Y}$を変数とするとき,
$\bar{Y}=0$で$I_Z$が最小となることを示している.言い換えれば,
あらゆる軸に関する断面2次モーメントの中で最小値をとるのは,$\bar Y=0 (Z=z)$すなわち中立軸
に関するものであることを示している.
%%%%%%%%%%%%%%%%%%%%%%%%%%%%%%%%%%%%%%%%%%%%%%%%%%%%%%%%%%%%%%%%%%%%%%%555
\section{部分断面への分割を利用した断面係数の計算方法}
断面$S$の,$n$個の互いに共通部分を持たない部分断面$S_i, (i=1,\dots, n)$
への分割:
\begin{equation}
	S=S_1\cup S_2 \cup \dots \cup S_n=\cup_{i=1}^n S_i
	\label{eqn:cup_Si}
\end{equation}
かつ
\begin{equation}
	S_i \cap S_j =\phi(空集合), \ \ (i,j=1,2,\dots ,n ) 
	\label{eqn:cap_Si}
\end{equation}
が与えられたとする.いま,部分断面$\left\{ S_i \right\}$は比較的簡単な形状をしており,
各$S_i$の中立軸位置や断面2次モーメントは既知あるいは容易に計算ができるとする.
図\ref{fig:fig11_2}-(a)は,この状況を$n=2$の場合について示したものである.


ここで,断面$S$の断面積$\left| S \right|$は,$S_i$が互いに共通部分を持たないことから
部分断面の断面積の和で与えられる.
\begin{equation}
	\left|S\right| = \sum_{i=1}^n \left| S_i \right|
	\label{eqn:Stot}
\end{equation}
また,断面係数を求めるための積分計算を行う座標を$(Y,\,Z)$とし,
$S_i$の中立軸を$z_i$, $Z$軸と$z_i$軸の距離を$\bar Y_i$と表す(図\ref{fig:fg11_2}-(b)).
式(\ref{eqn:Ybar})より,$\bar Y_i$は
\begin{equation}
	\bar{Y}_i = \frac{G_Z(S_i)}{\left| S_i \right|}, 
	\ \ G_Z(S_i)=\int_{S_i}YdS
\ \ (i=1,\dots,n)
	\label{eqn:Y_i}
\end{equation}
と,各々の断面に関する断面1次モーメント$G_Z(S_i)$から求められる.
一方,全断面$S$の断面1次モーメント$G_Z(S)$は,
\begin{equation}
	G_Z(S)=\int_SYdS
	=\sum_{i=1}^n \int_{S_i} YdS
	=\sum_{i=1}^n G_Z(S_i)
%	= \sum_{i=1}^n \bar{Y}_i\left| S_i \right|
	\label{eqn:GZ_Stot}
\end{equation}
と$G_Z(S_i)$の和で与えられることから,式(\ref{eqn:GZ_Stot})と式(\ref{eqn:Y_i})より
\begin{equation}
	\bar Y 
	= \frac{G_Z(S)}{\left| S \right|}
	= \sum_{i=1}^n \frac{G(S_i)}{\left| S \right|}
	= \sum_{i=1}^n \frac{\left| S_i\right|}{\left| S \right|}\bar{Y}_i
	\label{eqn:Yi2Y}
\end{equation}
が得られる.式(\ref{eqn:Yi2Y})は,部分断面の断面積と中立軸位置から全断面$S$の
中立軸位置$\bar Y$を計算するために用いることができる.

同様に,$S$の$z$軸に関する断面2次モーメント$I_z(S)$は,部分断面の断面2次モーメント$I_z(S_i)$の
和として,
\begin{equation}
	I_z(S)=\int_S y^2 dS=\sum_{i=1}^n\int_{S_i} y^2dS=\sum_{i=1}^n I_z(S_i)
	\label{eqn:sum_of_Izi}
\end{equation}
と表される.ここで,$I_z(S_i)$(部分断面$S_i$の$z$軸に関する断面2次モーメント)は,
$I_{z_i}(S_i)$($S_i$のそれ自身の中立軸$z_i$に関する断面2次モーメント)の
関係は,式(\ref{eqn:Iz2IZ})より
\begin{equation}
	I_z(S_i)=I_{z_i}(S_i)+\left( \bar Y_i-\bar{Y} \right)^2 \left|S_i\right|
	\label{eqn:Iz_Si}
\end{equation}
となる.よって,式(\ref{eqn:sum_of_Izi})と式(\ref{eqn:Iz_Si})より
\begin{equation}
	I_z(S)=
	\sum_{i=1}^n\left\{
		I_{z_i}(S_i)+\left( \bar Y_i-\bar{Y} \right)^2 \left|S_i\right|
	\right\}
	\label{eqn:IzS_sum}
\end{equation}
の関係が得られる.式(\ref{eqn:IzS_sum})の右辺は,式(\ref{eqn:Stot})と式(\ref{eqn:Yi2Y})を用いるならば
全て部分断面ごとに計算することができる量である.従って,部分断面が簡単な形状をしている場合,
式(\ref{eqn:IzS_sum})を用いることで,比較的簡単な計算の繰り返しにより断面$S$に関する
断面2次モーメントを求めることができる.
%--------------------
\begin{figure}[h]
	\begin{center}
	\includegraphics[width=1.0\linewidth]{./fig11_2.eps} 
	\end{center}
	\caption{
		(a) 複合断面$S$の部分断面$S_1$および$S_2$への分割.
		(b) $ZY$座標系における複合断面とその部分断面の中立軸$z_1,z_2$.
	} 
	\label{fig:fig11_2}
\end{figure}
%--------------------
%--------------------
%%%%%%%%%%
%%%%%%%%%%
\section{計算例}
断面係数や中立軸位置を求めるためには,面積積分(2重積分)の計算を行う必要がある.
そこで本節では,逐次積分法による2重積分の計算方法についてはじめに述べる.
次に,基本な断面形に対する断面係数の計算方法と結果を示し,最後に
部分断面への分割を利用して行った断面係数の計算例を示す.
\subsection{逐次積分による二重積分の計算方法}
$z=f(x,y)$を$xy$直角直交座標系を持つ2次元平面上で定義された2変数関数
とし,$f(x,y)$の二重積分:
\begin{equation}
	F(S)=\int_S f(x,y) dS
	\label{eqn:Int2D}
\end{equation}
について考える.ここで,$S$は積分範囲を,$dS=dxdy$は微小面積要素を表し,
積分範囲$S$は$xy$平面内の点$(x,y)$の集合(領域)として指定される(図\ref{fig:fig10_1}).
図\ref{fig:fig10_1}-(a)に示すように,2重積分$F(S)$の計算は,幾何学的には領域$S$を底面とし,
曲面$z=f(x,y)$を上面とする柱状領域の体積を求めることに相当する.
具体的に与えられた積分範囲$S$に対して積分$F(S)$を計算するには,
積分変数$x$と$y$のそれぞれについて,1変数関数と同様にして順次積分を実行すればよい.
そのためには,どちらの変数について積分を先に行うかを決め,
それぞれの変数が動きうる範囲を書き下す必要がある.この作業と続く変数毎の積分計算の
の煩雑さは,積分範囲$S$の形状によってかなりの程度異なる.例えば,$S$として
図\ref{fig:fig10_1}-(b)に示すような長方形領域:
\begin{equation}
	S=\left\{(x,y):x_{min}<x<x_{max}, \, y_{min}<y<y_{max} \right\}
	\label{eqn:Sa}
\end{equation}
を選べば,$x$と$y$に関する積分範囲の上限と下限は互いに独立かつ一定で,
次のように単純である.
\begin{eqnarray}
	F(S)
	&=&
	\int_{x=x_{min}}^{x_{max}}\left\{\int_{y=y_{min}}^{y_{max}} f(x,y)dy\right\} dx
	\label{eqn:y_then_x}
	\\
	&=&
	\int_{y=y_{min}}^{y_{max}}\left\{\int_{x=x_{min}}^{x_{max}} f(x,y)dx\right\} dy
	\label{eqn:x_then_y}
\end{eqnarray}
ここに,式(\ref{eqn:y_then_x})では最初に$y$について,式(\ref{eqn:x_then_y})は$x$に関する
積分を先に行うことを意味する.なお,一方の変数に関する積分実行時には,もう一方の変数は一定値
に固定されていると考え,1変数関数の積分と同様にして計算を行えばよい.
一方,$S$が図\ref{fig:fig10_1}-(c)に示すような任意形状の場合,積分変数$x$と$y$の上限と下限は互いに
独立でなく,$y$の上下限は$x$に,$x$の上下限は$y$に依存する.
そこで,図\ref{fig:fig10_2}に示すように,$x$を固定したときの$y$の上限,下限をそれぞれ$y_1(x),y_2(x)$,
$y$を固定したときの$x$の上限,下限を,それぞれ$x_1(y),x_2(y)$と
すれば,$S$上での2重積分を次のように書くことができる.
\begin{eqnarray}
	F(S) &=& 
	\int_{x=x_{min}}^{x_{max}}\ \left( \int_{y=y_1(x)}^{y_2(x)}f(x,y)dy\right) dx
	\label{eqn:iint_xy}
	\\
	&=& 
	\int_{y=y_{min}}^{y_{max}}\ \left( \int_{x=x_1(y)}^{x_2(y)}f(x,y)dx\right) dy
	\label{eqn:iint_yx}
\end{eqnarray}
ただし,$x_{min},x_{max}$は$S$における$x$の最大,最小値を,$y_{min},y_{max}$は$S$における$y$の最大,最小値を表す.
%--------------------
\begin{figure}[h]
	\begin{center}
	\includegraphics[width=0.8\linewidth]{./fig10_1.eps} 
	\end{center}
	\caption{
		(a) 2変数関数$z=f(x,y)$が矩形領域上に作る曲面. 
		(b) 矩形および(c)任意の積分領域と, $xy$直角直交座標系における微小面積要素. 
	} 
	\label{fig:fig10_1}
\end{figure}
%-----------------------------
\begin{figure}[h]
	\begin{center}
	\includegraphics[width=0.8\linewidth]{./fig10_2.eps} 
	\end{center}
	\caption{
		領域$S$において
		(a) 固定された$y$に対して$x$の取りうる範囲$\left(x_1(y),x_2(y)\right)$, 
		(b) 固定された$x$に対して$y$の取りうる範囲$\left(y_1(x),y_2(x)\right)$. 
	} 
	\label{fig:fig10_2}
\end{figure}
以上の手順を,図\ref{fig:fig10_3}-(b)に示すような三角形領域$S_b$に適用すると,
\begin{equation}
	x_{min}=y_{min}=0, \ \ x_{max}=b, \ \ y_{max}=h
\end{equation}
で,$x$を固定したときに$y$の動きうる範囲は,
\begin{equation}
	\frac{h}{b}x < y < h, 
	\label{eqn:ybnd_Sb}
\end{equation}
$y$を固定したときに$x$の動きうる範囲は,
\begin{equation}
	0 < x < \frac{b}{h}y
	\label{eqn:xbnd_Sb}
\end{equation}
である.よって,領域$S_b$における二変数関数$f(x,y)$の面積積分は
\begin{eqnarray}
	\int_{S_b} f(x,y) dS
	&=&
	\int_{x=0}^b \left\{ \int_{y=\frac{h}{b}x}^h f(x,y)dy\right\} dx 
	\label{eqn:int_Sb_yx}
	\\
	&=&
	\int_{y=0}^h \left\{ \int_{x=0}^{\frac{b}{h}y} f(x,y)dx\right\}dy
	\label{eqn:int_Sb_xy}
\end{eqnarray}
と表すことができる.

次に,図\ref{fig:fig10_3}-(c)に示す領域$S_c$について,同様にして$x,y$座標のとりうる範囲を調べれば,$S_c$上での積分が
\begin{eqnarray}
	\int_{S_c} f(x,y) dS
	&=&
	\int_{x=0}^b \left\{ \int_{y=0}^{h\left(1-\frac{x}{b}\right)} f(x,y)dy\right\}dx 
	\label{eqn:int_Sc_yx}
	\\
	&=&
	\int_{y=0}^h \left\{ \int_{x=0}^{b\left(1-\frac{y}{h}\right)} f(x,y)dx\right\} dy
	\label{eqn:int_Sc_xy}
\end{eqnarray}
と書けることが分かる.

最後に,図\ref{fig:fig10_3}-(d)の領域$S_d$については,極座標:
\begin{equation}
	\left(x,\,y\right) = \left(\ r\cos\theta,\, r\sin\theta \right)
	\label{eqn:cart2pol}
\end{equation}
を用いて積分を行えばよい.
この場合は,微小面積要素が$dS=r dr d\theta $となることに注意し,
座標$r,\theta$の動きうる範囲を調べればよく,$S_d$において$r$の動きうる
範囲は$\theta$に依らず$0<r<a$,$\theta$の動きうる範囲は$r$に依らず
$0<\theta<\frac{\pi}{2}$である.
よって,
\begin{equation}
	\int_{S_d} f(x,y) dS
	=
	\int_{r=0}^a \int_{\theta=0}^{\frac{\pi}{2}} f(r\cos\theta,r\sin\theta)rd\theta dr
	\label{eqn:int_pol}
\end{equation}
と表され,積分範囲は$r$と$\theta$の積分順序を交換しても同じである.


以上を踏まえて,$f(x,y)=x^my^n$($m,n$は非負整数)として$S_a, S_b$および$S_c$における
面積積分を行うと,その結果は以下のようになる.
\begin{eqnarray}
	F(S_a) &= & 
	\int_{S_a}x^my^ndS \nonumber \\
	&= & 
	\int_0^h \left( \int_0^b x^m y^n dx \right) dy \nonumber \\
	&= & 
	\int_0^h y^n \left[ \frac{x^{m+1}}{m+1} \right]_0^b dy \nonumber \\
	&= & 
	\frac{b^{m+1}}{m+1}
	\left[
		\frac{y^{n+1}}{n+1}
	\right]_0^h
	\nonumber
	\\
	&= & 
	\frac{ b^{m+1} h^{n+1}}{(m+1)(n+1)}
	\label{eqn:int_Sa_xmyn}
\end{eqnarray}
\begin{eqnarray}
	F(S_b) 
	&= & 
	\int_0^h \left(\int_0^{\frac{b}{h}y} x^my^ndx \right) dy
	\nonumber
	\\
	&= & 
	\int_0^h y^n \left[\frac{x^{m+1}}{m+1}\right]_0^{\frac{b}{h}y}dy 
	\nonumber
	\\
	&= & 
	\frac{1}{m+1}\left(\frac{b}{h}\right)^{m+1} \int_0^h y^{m+n+1}dy 
	\nonumber
	\\
	&= & 
	\frac{b^{m+1}h^{n+1}}{(m+1)(m+n+2)}
	\label{eqn:int_Sb_xmyn}
\end{eqnarray}
\begin{eqnarray}
	F(S_c)&= & 
	\int_0^h \left(\int_0^{b\left( 1-\frac{y}{h}\right) } x^my^ndx \right) dy
	\nonumber
	\\
	&=&
	\int_0^h \left[ \frac{x^{m+1}}{m+1}\right]_0^{b\left(1-\frac{y}{h}\right)} dy \nonumber
	\\
	&=&
	\frac{b^{m+1}}{m+1} \int_0^h \left( 1-\frac{y}{h} \right)^{m+1} y^n dy 
	\nonumber
	\\
	&=&
	\frac{b^{m+1}}{m+1}h^{n+1} \int_0^1 \left( 1-\xi \right)^{m+1} \xi^n d\xi, \ \ \left(\xi=\frac{y}{h}\right) 
	\label{eqn:int_Sc_via}
	\\
	&=&
	\frac{m!n!}{(m+n+2)!}b^{m+1}h^{n+1}
	\label{eqn:int_Sc_xmyn}
\end{eqnarray}
以上において,式(\ref{eqn:int_Sc_via})の積分は,次に示すように部分積分を$n$回繰り返して行うことで計算することができる.
\begin{eqnarray}
	\int_0^1 \left( 1-\xi \right)^{m+1} \xi^n d\xi, \ \ (\xi=\frac{y}{h}) 
	&=&
	\left[-\frac{(1-\xi)}{m+2}\xi^n\right]_0^1 + \frac{n}{m+2}\int_0^h (1-\xi)^{m+2}\xi^{n-1}d\xi
	\nonumber
	\\
	&=&
	\cdots
	\nonumber
	\\
	&=&
	\frac{n(n-1)\cdots 1}{(m+2)\cdots (m+n+1)} \int_0^1 (1-\xi)^{m+n+1}d\xi 
	\nonumber
	\\
	&=&
	\frac{(m+1)! n!}{(m+n+2)!}
	\label{eqn:int_by_part}
\end{eqnarray}
なお,これらは全て$x$に関する積分を先に行った場合の過程を示しているが,
$y$から先に積分を行っても結果に変わりはない.

最後に,$S_d$における積分を$f(x,y)=y$,$f(x,y)=y^2$の場合について計算すると
以下のようになる.
\begin{eqnarray}
	F(S_d) 
	&=& \int_{S_d} y dS \nonumber \\
	&=& \int_0^\frac{\pi}{2} \left( \int_0^a r\sin\theta rdr \right)d\theta \nonumber \\
	&=& \frac{a^3}{3}\int_0^\frac{\pi}{2} \sin\theta  d\theta \nonumber \\
	&=& \frac{a^3}{3}
	\label{eqn:int_Sd_y}
\end{eqnarray}
\begin{eqnarray}
	F(S_d) 
	&=& \int_{S_d} y^2 dS \nonumber \\
	&=& \int_0^\frac{\pi}{2} \left( \int_0^a r^2\sin^2\theta rdr \right)d\theta \nonumber \\
	&=& \frac{a^4}{4}\int_0^\frac{\pi}{2} \sin^2\theta d\theta \nonumber \\
	&=& \frac{a^4}{4} \int_0^\frac{\pi}{2} \frac{1-\cos 2\theta}{2} d\theta 
	\nonumber
	\\
	&=& \frac{\pi}{16}a^4
	\label{eqn:int_Sd_y2}
\end{eqnarray}
ここで示した面積積分の結果は,四角形,三角形および円形断面の断面係数を計算するために
利用することができる.
\begin{figure}[h]
	\begin{center}
	\includegraphics[width=0.5\linewidth]{./fig10_3.eps} 
	\end{center}
	\caption{
		簡単な形状をした積分領域の例. 
	} 
	\label{fig:fig10_3}
\end{figure}
%--------------------
\begin{figure}[h]
	\begin{center}
	\includegraphics[width=0.8\linewidth]{./fig11_3.eps} 
	\end{center}
	\caption{
		基本的な断面形と断面係数計算のための座標系.
	} 
	\label{fig:fig11_3}
\end{figure}
\subsection{長方形,三角形,円の断面係数}
図\ref{fig:fig11_3}-(1)から(3)に示す3つの断面について断面係数を求める.
これらの結果は基本的なものであるため,その都度計算するのでなく記憶しておくべきものである.
\subsubsection{長方形}
断面積は$\left|S_1\right|=bh$,断面1次モーメントは
\begin{equation}
	G_Z=\int_{S_1} Y dS=\int_0^h\int_0^b Y dZdY=\int_0^h bYdY=\frac{bh^2}{2}
	\label{eqn:GZ_S1}
\end{equation}
であることから,中立軸の位置が
\begin{equation}
	\bar Y= \frac{G_Z}{\left| S_1 \right|}=\frac{bh^2/2}{bh}=\frac{h}{2}
\end{equation}
と求まる.$Z$軸に関する断面2次モーメントを計算すると
\begin{equation}
	I_Z=\int_{S_1} Y^2dZdY
	=\int_0^h \int_0^b Y^2dZdY 
	=\int_0^h bY^2dY 
	=\frac{bh^3}{3}
	\label{eqn:IZ_S1}
\end{equation}
となるので,式(\ref{eqn:IZ2Iz})を用いて,中立軸周りの断面2次モーメントが
\begin{equation}
	I_z= I_Z-\bar Y^2 \left| S_1 \right| = \frac{bh^3}{3}-\left(\frac{h}{2}\right)^2\times bh=\frac{bh^3}{12}
	\label{eqn:Iz_S1}
\end{equation}
と得られる.
\subsubsection{三角形}
断面積は$\left|S_2\right|=\frac{bh}{2}$,$Z$軸に関する断面1次および2次モーメント
\begin{equation}
	G_Z=\int_{S_2} Y dS, \ \ I_Z=\int_{S_2}Y^2 dS
\end{equation}
は,式(\ref{eqn:int_Sb_xmyn})を利用して計算することができる.
具体的には,式(\ref{eqn:int_Sb_xmyn})において$m=0, n=1$と
した場合$G_Z$が,$m=0, n=2$とした場合に$I_Z$が得られる.
その結果,
\begin{equation}
	G_Z=\frac{bh^2}{3}, \ \ 
	I_Z=\frac{bh^3}{4}
	\label{eqn:GI_S2}
\end{equation}
となり,式(\ref{eqn:Ybar})より
\begin{equation}
	\bar{Y}=\frac{2h}{3}
\end{equation}
が,式(\ref{eqn:IZ2Iz})より
\begin{equation}
	I_z=\frac{bh^3}{4}-\left(\frac{2h}{3}\right)\times \frac{bh}{2}=\frac{bh^3}{36}
	\label{eqn:Iz_S2}
\end{equation}
の結果が得られる.
\subsubsection{円}
円の面積は$\left|S_3\right|=\pi a^2$で与えられ,$G_Z$は,$S_d$が$Z$軸に関して上下対称,
被積分関数である$y$が$Z$軸に関して奇関数であることから$G_Z=0$である.
従って,中立軸の位置は$\bar Y=0$で,この場合$I_Z=I_z$となる.
また,$I_Z$は式(\ref{eqn:int_Sd_y2})の積分の丁度4倍であることから
\begin{equation}
	I_z=I_Z=\int_{S_3}Y^2dS=4\times \frac{\pi a^4}{16}=\frac{\pi a^4}{4}
	\label{eqn:Iz_S3}
\end{equation}
となる.
\subsubsection{問題}
図\ref{fig:fig11_3}-(4)から(6)に示す断面$S_4,S_5,S_6$について以下の問に答えよ.
\begin{enumerate}
\item
三角形$S_4$に関する断面係数と中立軸位置を求めよ.
\item
平行四辺形$S_5$に関する断面係数と中立軸の位置は,長方形$S_1$の場合と同じになることを示せ.
\item
三角形$S_6$に関する断面係数と中立軸の位置は,三角形$S_4$の場合と同じになることを示せ.
\end{enumerate}
\subsection{複合断面に関する断面係数の計算例}
\subsubsection{長方形断面}
図\ref{fig:fig11_4}-(a)のような長方形断面を,2つの長方形$S_1$と$S_2$に分割して計算を行う.
長方形断面の中立軸位置や断面係数は,積分計算により前節で求めた通りであるが,
部分断面分割に基づく断面係数の計算過程を簡単な計算で示し,正しい結果が得られることを見る
ために,ここでは既に答えが分かっている問題にこの方法を適用してみる.

はじめに,断面$S_1,S_2$の断面積は
\begin{equation}
	\left| S_1 \right|
	=
	\left| S_2 \right|
	=
	\frac{bh}{2}
	\label{eqn:area_S1S2}
\end{equation}
で,全断面$S=S_1\cup S_2$の断面積はこれらの和として$\left|S\right|=\frac{bh}{2}+\frac{bh}{2}=bh$と
求められる.これより,部分断面の全断面に対する面積比$\xi_i$は,
\begin{equation}
	\xi_1=\frac{\left| S_1 \right|}{\left|S\right|} =\frac{1}{2}
	, \ \ 
	\xi_2=\frac{\left| S_2 \right|}{\left|S\right|} =\frac{1}{2}
	\label{eqn:ratio_area}
\end{equation}
となる. $i=1,2$に対し,$S_i$の中立軸$z_i$の位置を,$Z$軸からの距離$\bar{Y}_i$で
表すと,$\bar{Y}_i$は
\begin{equation}
	\bar{Y}_1=\frac{h}{4}
	, \  \
	\bar{Y}_2=\frac{3h}{4}
	\label{eqn:}
\end{equation}
であることから
\begin{equation}
	\bar{Y}=\sum_{i=1}^2 \xi_i \bar{Y}_i
	=\frac{h}{8}+\frac{3h}{8}=\frac{h}{2}
	\label{eqn:}
\end{equation}
となり,$S$の中立軸位置$\bar Y$が正しく求められる.この結果を式(\ref{eqn:Yi2Y})に用いれば,
\begin{equation}
	\sum_{i=1}^2 \left( \bar{Y}_i-\bar{Y}\right) \left| S_i\right|
	=
	\left(\frac{h}{4}\right)^2\times \frac{bh}{2}
	+
	\left(\frac{h}{4}\right)^2\times \frac{bh}{2}
	=\frac{bh^3}{16}
\end{equation}
となる,最後に,$S_i$のそれ自身の中立軸$z_i$に関する断面2次モーメント$I_{z_i}$の
和を求めると,
\begin{equation}
	\sum_{i=1}^2 I_{z_i}=
	\frac{b}{12}\left(\frac{h}{2}\right)^3
	+
	\frac{b}{12}\left(\frac{h}{2}\right)^3
	=
	\frac{bh^3}{48}
\end{equation}
となるので,これらを式(\ref{eqn:IzS_sum})に代入すれば,$S$の$z$に関する
断面2次モーメント$I_z(S)$が
\begin{equation}
	I_z(S)=
	\frac{bh^3}{16}
	+
	\frac{bh^3}{48}
	=
	\frac{bh^3}{12}
\end{equation}
となり,期待した通りの結果が得られる.
表\ref{tbl0}は,以上の計算過程をまとめたものである.
一部,"-"を記入した箇所は,合計値を計算して記入してもよいが,その値には特に
意味が無いために未記入としていることを示すものである.
部分断面分割に基づく計算を行う際にはこのような表を作成して計算を進めるとよく,
そのことは3つ以上の部分断面に分割される場合も同様である.
その場合,3つ目以後の部分断面に関する行を追加して,最下行には全ての部分断面
についての和を記入するようにすればよい.
\subsubsection{台形断面}
図\ref{fig:fig11_4}-(b)のような台形断面を,長方形$S_1$と三角形$S_1$に分割する.
$S_1,S_2$の中立軸位置,断面係数を既知として全断面$S=S_1 \cup S_2$の中立軸位置と
断面2次モーメントを計算すると,その過程は表\ref{tbl1}のようにまとめられ,
最終的に次の結果が得られる.
\begin{equation}
	\bar{Y}=\frac{4h}{9}, \ \ 
	I_z=\frac{bh^3}{108}+\frac{bh^3}{9}=\frac{13}{108}bh^3
\end{equation}
\subsubsection{L字型断面}
最後に,図\ref{fig:fig11_4}-(d)のようなL字型断面$S$を,2つの長方形部分断面
$S_1,S_2$に分割して断面係数計算を行った過程を,表\ref{tbl2}に示す.
$S$の中立軸位置と断面2次モーメント$I_z$は,この表に示された計算結果より
\begin{equation}
	\bar{Y}=\frac{19}{5}t, \ \ 
	I_z=\frac{216}{5}t^4+\frac{44}{3}t^4 = \frac{868}{15}t^4
\end{equation}
となる.
\subsubsection{問題}
図\ref{fig:fig11_4}-(c),(e),(f)に示された断面形と部分断面分割について,以下の問に答えよ.
\begin{enumerate}
\item
図\ref{fig:fig11_4}-(c)のように,台形断面を2つの3角形に分割して
中立軸位置$\bar{Y}$と断面2次モーメント$I_z$を求め,その結果が
同図-(b)の断面分割に基づく計算結果と一致することを示せ.
\item
図\ref{fig:fig11_4}-(e)の断面分割に基づいてL字型断面$S=S_1\cup S_2$の中立軸位置
$\bar{Y}$と断面2次モーメント$I_z$を求め,その結果が同図-(d)の断面分割による計算結果
と一致することを示せ.
\item
図\ref{fig:fig11_4}-(f)に示す半径$a$の3つの部分断面$S_i,(i=1,2,3)$から構成される断面$S=\cup_{i=1}^3S_i$ついて,
中立軸位置$\bar{Y}$と断面2次モーメント$I_z$を求めよ.
\end{enumerate}
%--------------------
\begin{figure}[h]
	\begin{center}
	\includegraphics[width=0.8\linewidth]{./fig11_4.eps} 
	\end{center}
	\caption{
		部分断面への分割を利用した断面係数計算の例題
	} 
	\label{fig:fig11_4}
\end{figure}
%--------------------
\begin{table}
\begin{center}
	\caption{部分断面への分割に基づく長方形の断面係数計算の過程}
	\begin{tabular}{c||c|c|c|c|c|c|c}
		&
		$\left| S_i \right|$ & 
		$ \xi_i=\frac{\left| S_i \right|}{\left| S\right|} $  &
		$ \bar{Y}_i $ & 
		$ \xi_i\bar{Y}_i $ & 
		$\bar{Y}_i -\bar Y$ & 
		$ \left(\bar{Y}_i -\bar Y\right)^2\left| S_i \right|$ & 
		$ I_{z_i}$  
		\\
		\hline 
		断面1&	
		$\frac{bh}{2}$ & 
		$\frac{1}{2}$  &
		$\frac{3h}{4}$ & 
		$\frac{3h}{8}$ & 
		$\frac{h}{4}$ & 
		$\left(\frac{h}{4}\right)^2 \times \frac{bh}{2}$ & 
		$\frac{b}{12}\times \left(\frac{h}{2}\right)^3$  
		\\
		\hline
		断面2&	
		$\frac{bh}{2}$ & 
		$\frac{1}{2}$  &
		$\frac{h}{4}$ & 
		$\frac{h}{8}$ & 
		$-\frac{h}{4}$ & 
		$\left(\frac{h}{4}\right)^2 \times \frac{bh}{2}$ & 
		$\frac{b}{12}\times \left(\frac{h}{2}\right)^3$  
		\\
		\hline 
		合計&	
		$bh$ & 
		$1$  &
		$-$ & 
		$\bar Y =\frac{h}{2}$ & 
		$-$ & 
		$\frac{bh^3}{16}$ & 
		$\frac{bh^3}{48}$ 
	\end{tabular}
\label{tbl0}
\end{center}
\end{table}
\begin{table}
\begin{center}
	\caption{部分断面への分割に基づく台形の断面係数計算の過程}
	\caption{複合断面の断面係数計算(台形)}
	\begin{tabular}{c||c|c|c|c|c|c|c}
		&
		$\left| S_i \right|$ & 
		$ \xi_i=\frac{\left| S_i \right|}{\left| S\right|} $  &
		$ \bar{Y}_i $ & 
		$ \xi_i\bar{Y}_i $ & 
		$\bar{Y}_i -\bar Y$ & 
		$ \left(\bar{Y}_i -\bar Y\right)^2\left| S_i \right|$ & 
		$ I_{z_i}$  
		\\
		\hline 
		断面1&	
		$bh$ & 
		$\frac{2}{3}$  &
		$\frac{h}{2}$ & 
		$\frac{h}{3}$ & 
		$\frac{h}{18}$ & 
		$\left(\frac{h}{18}\right)^2 \times bh $ & 
		$\frac{bh^3}{12}$  
		\\
		\hline
		断面2&	
		$\frac{bh}{2}$ & 
		$\frac{1}{3}$  &
		$\frac{h}{3}$ & 
		$\frac{h}{9}$ & 
		$-\frac{h}{9}$ & 
		$\left(-\frac{h}{9}\right)^2 \times \frac{bh}{2}$ & 
		$\frac{bh^3}{36}$  
		\\
		\hline 
		合計&	
		$\frac{3}{2}bh$ & 
		$1$  &
		$-$ & 
		$\bar Y=\frac{4}{9}h$ & 
		$-$ & 
		$\frac{bh^3}{108}$ & 
		$\frac{bh^3}{9}$ 
	\end{tabular}
	\label{tbl1}
\end{center}
\end{table}
\begin{table}
\begin{center}
	\caption{部分断面への分割に基づくL字型断面の断面係数計算の過程}
	\begin{tabular}{c||c|c|c|c|c|c|c}
		&
		$\left| S_i \right|$ & 
		$ \xi_i=\frac{\left| S_i \right|}{\left| S\right|} $  &
		$ \bar{Y}_i $ & 
		$ \xi_i\bar{Y}_i $ & 
		$\bar{Y}_i -\bar Y$ & 
		$ \left(\bar{Y}_i -\bar Y\right)^2\left| S_i \right|$ & 
		$ I_{z_i}$  
		\\
		\hline 
		断面1&	
		$8t^2$ & 
		$\frac{2}{5}$  &
		$2t$ & 
		$\frac{4}{5}t$ & 
		$-\frac{9}{5}t$ & 
		$\left(\frac{9}{5}t\right)^2 \times 8t^2 $ & 
		$\frac{2t\times (4t)^3}{12}=\frac{32}{3}t^4$  
		\\
		\hline
		断面2&	
		$12t^2$ & 
		$\frac{3}{5}$  &
		$5t$ & 
		$3t $ & 
		$\frac{6}{5}t$ & 
		$\left(\frac{6t}{5}\right)^2 \times 12t^2 $ & 
		$\frac{6t\times (2t)^3}{12}=4t^4$  
		\\
		\hline 
		合計&	
		$20t^2$ & 
		$1$  &
		$-$ & 
		$\bar Y=\frac{19}{5}t$ & 
		$-$ & 
		$\frac{216}{5}t^4$ & 
		$\frac{44}{3}t^4$ 
	\end{tabular}
\label{tbl2}
\end{center}
\end{table}
\end{document}
