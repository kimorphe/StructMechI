\documentclass[10pt,a4j]{jarticle}
\usepackage{graphicx,wrapfig}
\setlength{\topmargin}{-1.5cm}
\setlength{\textwidth}{15.5cm}
\setlength{\textheight}{25.2cm}
\newlength{\minitwocolumn}
\setlength{\minitwocolumn}{0.5\textwidth}
\addtolength{\minitwocolumn}{-\columnsep}
%\addtolength{\baselineskip}{-0.1\baselineskip}
%
\def\Mmaru#1{{\ooalign{\hfil#1\/\hfil\crcr
\raise.167ex\hbox{\mathhexbox 20D}}}}
%
\begin{document}
\newcommand{\fat}[1]{\mbox{\boldmath $#1$}}
\newcommand{\D}{\partial}
\newcommand{\w}{\omega}
\newcommand{\ga}{\alpha}
\newcommand{\gb}{\beta}
\newcommand{\gx}{\xi}
\newcommand{\gz}{\zeta}
\newcommand{\vhat}[1]{\hat{\fat{#1}}}
\newcommand{\spc}{\vspace{0.7\baselineskip}}
\newcommand{\halfspc}{\vspace{0.3\baselineskip}}
\bibliographystyle{unsrt}
\pagestyle{empty}
\newcommand{\twofig}[2]
 {
   \begin{figure}[h]
     \begin{minipage}[t]{\minitwocolumn}
         \begin{center}   #1
         \end{center}
     \end{minipage}
         \hspace{\columnsep}
     \begin{minipage}[t]{\minitwocolumn}
         \begin{center} #2
         \end{center}
     \end{minipage}
   \end{figure}
 }
%%%%%%%%%%%%%%%%%%%%%%%%%%%%%%%%%
%\vspace*{\baselineskip}
\begin{center}
{\Large \bf 2019年度 構造力学I及び演習A(第8回) 期末試験 解答} \\
\end{center}
%%%%%%%%%%%%%%%%%%%%%%%%%%%%%%%%%%%%%%%%%%%%%%%%%%%%%%%%%%%%%%%%
%%%%%%%%%%%%%%%%%%%%%%%%%%%%%%%%%%%%%%%%%%%%%%%%%%%%%%%%%%%%%%%%%%%%%%%%%%%%%%%%%%%%%%%%%%
\subsubsection*{問題1.}
\begin{enumerate}
\item
	外力を$p(x)$とする. $p(x)$は
	\begin{equation}
		p(x)=\frac{x}{l}p_0 
		\ \ \left(-l < x < 2l \right)
		\label{eqn:px}
	\end{equation}
	と表される.よって,軸方向変位$u(x)$の支配方程式は,
	\begin{equation}
		\left( EAu' \right)'=-\frac{x}{l}p_0
		\label{eqn:gv_eq}
	\end{equation}
	となる.境界条件は,両端($x=-l,2l$)固定であることから,以下のようである.
	\begin{equation}
		u(-l)=u(2l)=0
		\label{eqn:bcon}
	\end{equation}
\item
	$EA$が一定であることから$u''=-p/EA$である.また,式(\ref{eqn:gv_eq})より,
	積分定数を$C_1, C_2$とすれば,
	\begin{equation}
		u(x)=-\frac{p_0l^2}{6EA}\left\{
			\left( \frac{x}{l}\right)^3
			+
			C_1\left( \frac{x}{l}\right)
			+
			C_2
		\right\}
	\end{equation}
	と表すことができる.積分定数は式(\ref{eqn:bcon})より, $C_1=-3, C_2=-2$となる.
	よって変位分布が
	\begin{equation}
		u(x)=-\frac{p_0l^2}{6EA}\left\{
			\left( \frac{x}{l}\right)^3
			-
			3
			\left( \frac{x}{l}\right)
			-2
		\right\}
		\label{eqn:ux}
	\end{equation}
	と求められる.
\item
	軸力を$N$は,変位を微分して断面剛性$EI$を乗じ,
	\begin{equation}
		N(x)=EAu'=-\frac{p_0l}{2}
			\left\{
			\left(\frac{x}{l}\right)^2 -1
			\right\}
		\label{eqn:Nx}
	\end{equation}
	と得られる.
\item
	式(\ref{eqn:Nx})より,$N(-l)=0$,
	$N(2l)=-\frac{3p_0l}{2}$である.よって,支点Aで反力はゼロ,
	支点Bでは左向きに大きさ$\frac{2p_0l}{3}$の反力を受ける.
\item
	式(\ref{eqn:ux})より,変位は$x=l$で最大値$\frac{2p_0l^2}{3EA}$をとる.
	一方,変位の最小値は,部材両端部における値$u(-l)=u(2l)=0$であることから,
	伸びが最大になる区間は$-l<x<l $で, その区間での伸びは
	$\frac{2p_0l^2}{3EA}$である.
\end{enumerate}
\newpage
%%%%%%%%%%%%%%%%%%%%%%%%%%%%%%%%%%%%%%%%%%%%%%%%%%%%%%%%%%%%%%%%%%%%%%%%%%%%%%%%%%%%%%%%%%
\subsubsection*{問題2.}
\begin{enumerate}
\item
	$\bar{\sigma}=0, \Delta \sigma=2q, \tau = r$より,モールの応力円の中心は
	$(\sigma_{11}',\, \sigma_{12}')=(0,\, 0)$, 半径$R$は
	\begin{equation}
		R=\sqrt{\left(\frac{\Delta \sigma}{2}\right)^2+\tau ^2}=\sqrt{q^2+r^2}
		\label{neq:Radi}
	\end{equation}
	である.
	従って,モールの応力円は図\ref{fig:fig2}のようになる.
\item
	最大主応力を$\sigma_{max}$, $x_1$軸方向から反時計回りの方向に測った最大
	主応力方向を$\theta_{max}$とする.同様に,最小主応力を$\sigma_{min}$, 
	その方向を$\theta_{min}$とする.
	与えられた応力テンソル$\fat{\sigma}$は,モールの応力円において図\ref{fig:fig2}の
	Aの点で表される.一方,最大および最小主応力は,それぞれ,モールの応力円上の
	点BとCに相当することから,
	\begin{equation}
		\sigma_{max}=R, \ \ \theta_{max}=\frac{\phi}{2}
	\end{equation}
	\begin{equation}
		\sigma_{min}=-R, \ \ \theta_{min}=\frac{\phi}{2}+\frac{\pi}{2}
	\end{equation}
	となる.
\item
	\begin{equation}
		\fat{t}^{(n)}
		=\fat{\sigma}^T\fat{n}
		=
		\left( 
		\begin{array}{c}
			q\cos\alpha + r\sin\alpha \\
			r\cos\alpha - q\sin\alpha 
		\end{array}
		\right)
		=
		R
		\left( 
		\begin{array}{c}
			\cos(\phi-\alpha) \\
			\sin(\phi-\alpha) 
		\end{array}
		\right)
	\end{equation}
\item
	$\fat{t}^{(n)}$の$\fat{n}$方向成分は,
	\begin{equation}
		t^{(n)}_n=\fat{t}^{(n)}\cdot \fat{n}
		=
		R\left\{
			\cos(\phi-\alpha)\cos\alpha + \sin(\phi-\alpha) \sin\alpha
		\right\}
		=R\cos(\phi-2\alpha)
	\end{equation}
		だから,$\alpha_{max}=\frac{\phi}{2}$で,$t_{max}=R=\sqrt{q^2+r^2}$.
\item
	直ひずみと直応力の関係は
	\begin{equation}
		\left(
		\begin{array}{c}
			\varepsilon_{11} \\
			\varepsilon_{22} \\
			\varepsilon_{33}
		\end{array}
		\right)
		=
		\frac{1}{E}
		\left(
		\begin{array}{ccc}
			1 & -\nu & -\nu \\
			-\nu & 1 & -\nu \\
			-\nu & -\nu & 1 
		\end{array}
		\right)
		\left(
		\begin{array}{c}
			\sigma_{11} \\
			\sigma_{22} \\
			\sigma_{33}
		\end{array}
		\right)
		\label{eqn:Hooke}
	\end{equation}
	で与えられる.これに、
	\begin{equation}
		\sigma_{11}=q, \ \ \sigma_{22}=-q, \ \ \sigma_{33}=0
	\end{equation}
	を代入すれば,
	\begin{equation}
		\varepsilon_{11}=\frac{1+\nu}{E}q, \ \ 
		\varepsilon_{22}=-\frac{1+\nu}{E}q
	\end{equation}
	となる.
\item
	$\theta=\alpha_{max}=\frac{\phi}{2}$は最大主応力方向だから,このときの$o-x_1'x_2'$座標系に
	おける応力テンソル$\fat{\sigma}'$は,
	\begin{equation}
		\fat{\sigma}'=\left( 
		\begin{array}{cc}
			R & 0 \\
			0 & -R 
		\end{array}
		\right)
		\label{eqn:sigd}
	\end{equation}
	となる.これを,$o-x_1'x_2'$座標系における直ひずみと直応力の関係
	\begin{equation}
		\left(
		\begin{array}{c}
			\varepsilon_{11}' \\
			\varepsilon_{22}' \\
			\varepsilon_{33}'
		\end{array}
		\right)
		=
		\frac{1}{E}
		\left(
		\begin{array}{ccc}
			1 & -\nu & -\nu \\
			-\nu & 1 & -\nu \\
			-\nu & -\nu & 1 
		\end{array}
		\right)
		\left(
		\begin{array}{c}
			\sigma_{11}' \\
			\sigma_{22}' \\
			\sigma_{33}'
		\end{array}
		\right)
		\label{eqn:Hooke_d}
	\end{equation}
	に代入し,$\sigma_{33}'=0$とすれば,
	\begin{equation}
		\varepsilon_{11}=\frac{1+\nu}{E}R, \ \ 
		\varepsilon_{22}=-\frac{1+\nu}{E}R
	\end{equation}
	が得られる.
\item
	$o-x_1'x_2'$座標系において直応力$\sigma_{11}',\sigma_{22}'$は,
	\begin{eqnarray}
		\sigma_{11}'&=&R\cos(\phi-2\theta) \\ 
		\sigma_{22}'&=&-R\cos(\phi-2\theta) 
	\end{eqnarray}
	だから,
	これと$\sigma_{33}'=0$を式(\ref{eqn:Hooke_d})に代入すれば,
	\begin{eqnarray}
		\varepsilon_{11}'&=&\frac{1+\nu}{E}R\cos(\phi-2\theta) \\ 
		\varepsilon_{22}'&=&-\frac{1+\nu}{E}R\cos(\phi-2\theta) 
	\end{eqnarray}
	となり,$\varepsilon_{11}'$は$\theta=\frac{\phi}{2}=\alpha_{max}$
	で,最大値$\frac{1+\nu}{E}R$をとることが示される.
\end{enumerate}
%--------------------
\begin{figure}
	\begin{center}
	\includegraphics[width=0.46\linewidth]{fig2ans.eps} 
	\end{center}
	\caption{モールの応力円.} 
	\label{fig:fig2}
\end{figure}
\end{document}
