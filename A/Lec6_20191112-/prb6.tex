\documentclass[10pt,a4j]{jarticle}
\usepackage{graphicx,wrapfig}
\setlength{\topmargin}{-1.5cm}
\setlength{\textwidth}{15.5cm}
\setlength{\textheight}{25.2cm}
\newlength{\minitwocolumn}
\setlength{\minitwocolumn}{0.5\textwidth}
\addtolength{\minitwocolumn}{-\columnsep}
%\addtolength{\baselineskip}{-0.1\baselineskip}
%
\def\Mmaru#1{{\ooalign{\hfil#1\/\hfil\crcr
\raise.167ex\hbox{\mathhexbox 20D}}}}
%
\begin{document}
\newcommand{\fat}[1]{\mbox{\boldmath $#1$}}
\newcommand{\D}{\partial}
\newcommand{\w}{\omega}
\newcommand{\ga}{\alpha}
\newcommand{\gb}{\beta}
\newcommand{\gx}{\xi}
\newcommand{\gz}{\zeta}
\newcommand{\vhat}[1]{\hat{\fat{#1}}}
\newcommand{\spc}{\vspace{0.7\baselineskip}}
\newcommand{\halfspc}{\vspace{0.3\baselineskip}}
\bibliographystyle{unsrt}
\pagestyle{empty}
\newcommand{\twofig}[2]
 {
   \begin{figure}[h]
     \begin{minipage}[t]{\minitwocolumn}
         \begin{center}   #1
         \end{center}
     \end{minipage}
         \hspace{\columnsep}
     \begin{minipage}[t]{\minitwocolumn}
         \begin{center} #2
         \end{center}
     \end{minipage}
   \end{figure}
 }
%%%%%%%%%%%%%%%%%%%%%%%%%%%%%%%%%
%\vspace*{\baselineskip}
\begin{center}
{\Large \bf2018年度 構造力学I及び演習A 演習問題6(11月13日)} \\
\end{center}
%%%%%%%%%%%%%%%%%%%%%%%%%%%%%%%%%%%%%%%%%%%%%%%%%%%%%%%%%%%%%%%%
\subsubsection*{問題}
2次元空間内の物体がある応力を受けて変形し, そのときの変位ベクトル
$\fat{u}$が位置$\fat{x}$の関数として次のように与えられたとする.
\begin{equation}
	\fat{u}(\fat{x})=\fat{u}_0 + \fat{F}\fat{x}
	\label{eqn:disp_fld}
\end{equation}
ここに,$\fat{u}=(u_1,\, u_2)$は変位ベクトルを,$\fat{x}=(x_1, x_2)$は位置ベクトルを
$\fat{u}_0=(u_{01},u_{02})$は$\fat{x}=(0,0)$における変位ベクトルを表す.また,
$\fat{F}$は定数係数の2$\times$2行列
\begin{equation}
	\fat{F}=
	\left(
	\begin{array}{cc}
		G & H \\
		K & L
	\end{array}
	\label{eqn:Fmat}
	\right)
\end{equation}
であるとし,以上は全て$o-x_1x_2$直交座標系による表現であるとする.
ここで,$o-x_1x_2$座標系を,原点を中心に$\theta$だけ反時計回りの方向へ
回転させて得られるもう一つの直交座標系を$o-x_1'x_2'$座標系と呼ぶ.
$o-x_1'x_2'$座標系では,式(\ref{eqn:disp_fld})の関係を表すときに,
ベクトルやマトリクスとその成分に$(\cdot)'$をつけることで,
それらが依拠する座標系を区別する.
すなわち,$o-x_1'x_2'$座標系では,式(\ref{eqn:disp_fld})の関係を
\begin{equation}
	\fat{u}'(\fat{x}')=\fat{u}'_0 + \fat{F}'\fat{x}'
	\label{eqn:disp_fld2}
\end{equation}
と表す.ここで,
\begin{equation}
	\fat{u}'=(u_1',\, u_2'), \ \ 
	\fat{u}_0'=(u_{01}',\, u_{02}')
\end{equation}
が変位ベクトルを,
\begin{equation}
	\fat{x}'=(x_1',\, x_2')
\end{equation}
が位置ベクトルを,
\begin{equation}
	\fat{F}'=
	\left(
	\begin{array}{cc}
		G' & H' \\
		K' & L'
	\end{array}
	\label{eqn:Fmat2}
	\right)
\end{equation}
は,$\fat{F}$に対応する2$\times$2行列を意味する.
このとき,以下の問に答えよ.
\begin{enumerate}
\item
	$o-x_1x_2$座標系におけるひずみテンソルを
	\[
	\fat{\varepsilon}
	=
	\left(
	\begin{array}{cc}
		\varepsilon_{11} & \varepsilon_{12} \\
		\varepsilon_{21} & \varepsilon_{22} 
	\end{array}
	\right)
	\]
	と表すとき,
	\begin{equation}
		\fat{\varepsilon} = 
		\frac{1}{2} \left(
			\fat{F}+\fat{F}^T
		\right)
		\label{eqn:eij_of_F}
	\end{equation}
	であることを示せ.
	{\small
	(式(\ref{eqn:disp_fld})の変位ベクトルに対してひずみを成分毎に計算し,
	その結果が式(\ref{eqn:eij_of_F})
	のようにまとめられることを示せばよい.)}
\item
	$o-x_1x_2$から$o-x_1'x_2'$座標系への座標変換マトリクスを$\fat{Q}$とするとき,
	\[\fat{F}'=\fat{Q}\fat{F}\fat{Q}^T\]		
	の関係が成り立つことを示せ.
\item
	$o-x_1'x_2'$座標系におけるひずみテンソル
	\[
		\fat{\varepsilon}' = 
		\left(
		\begin{array}{cc}
			\varepsilon_{11}' & \varepsilon_{12}' \\
			\varepsilon_{21}' & \varepsilon_{22}' 
		\end{array}
		\right)
	\]
	を
	\[
		\fat{\varepsilon}' 
		=
		\frac{1}{2} \left(
			\fat{F}'+(\fat{F}')^T
		\right)
	\]
	で定義する.このとき
	\begin{equation}
		\fat{\varepsilon}'=\fat{Q}\fat{\varepsilon}\fat{Q}^T
		\label{eqn:e2ed}
	\end{equation}		
	であることを示せ.
\item
	$\theta=\frac{\pi}{4}$のときに, $\fat{\varepsilon}$と$\fat{\varepsilon}'$
	が以下のようであったとする.
	\begin{equation}
		\fat{\varepsilon}=
		\frac{1}{\sqrt{2}}
		\left(
			\begin{array}{cc}
				1 & \varepsilon_{12} \\
				\varepsilon_{12} & -1 
			\end{array}
		\right)
		, \ \ 
		\fat{\varepsilon}'=
		\frac{1}{\sqrt{2}}
		\left(
			\begin{array}{cc}
				1 & \varepsilon_{12}' \\
				\varepsilon_{12}' & \varepsilon_{22}' 
			\end{array}
		\right)
	\end{equation}
	このとき, $\varepsilon_{12}$と$\varepsilon_{12}',\varepsilon_{22}'$を求めよ.
\item
	前問で求めたひずみテンソル$\fat{\varepsilon}$について,$\varepsilon_{12}'=0$となるような$\theta$と,
	そのときの直ひずみ$\varepsilon_{11}',\varepsilon_{22}'$を求めよ.
\end{enumerate}
\end{document}

%--------------------
\begin{figure}[h]
	\begin{center}
	\includegraphics[width=0.8\linewidth]{fig1.eps} 
	\end{center}
	\caption{(a)$o-x_1x_2$座標系と, (b)固有ベクトル$\fat{n}_1,\fat{n}_2$を基底ベクトルにもつ$o-x_1'x_2'$座標系.} 
	\label{fig:fig1}
\end{figure}
%--------------------
