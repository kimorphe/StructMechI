\documentclass[10pt,a4j]{jarticle}
\usepackage{graphicx,wrapfig}
\setlength{\topmargin}{-1.5cm}
\setlength{\textwidth}{16.5cm}
\setlength{\textheight}{25.2cm}
\newlength{\minitwocolumn}
\setlength{\minitwocolumn}{0.5\textwidth}
\addtolength{\minitwocolumn}{-\columnsep}
%\addtolength{\baselineskip}{-0.1\baselineskip}
%
\def\Mmaru#1{{\ooalign{\hfil#1\/\hfil\crcr
\raise.167ex\hbox{\mathhexbox 20D}}}}
%
\begin{document}
\newcommand{\fat}[1]{\mbox{\boldmath $#1$}}
\newcommand{\D}{\partial}
\newcommand{\w}{\omega}
\newcommand{\ga}{\alpha}
\newcommand{\gb}{\beta}
\newcommand{\gx}{\xi}
\newcommand{\gz}{\zeta}
\newcommand{\vhat}[1]{\hat{\fat{#1}}}
\newcommand{\spc}{\vspace{0.7\baselineskip}}
\newcommand{\halfspc}{\vspace{0.3\baselineskip}}
\bibliographystyle{unsrt}
\pagestyle{empty}
\newcommand{\twofig}[2]
 {
   \begin{figure}
     \begin{minipage}[t]{\minitwocolumn}
         \begin{center}   #1
         \end{center}
     \end{minipage}
         \hspace{\columnsep}
     \begin{minipage}[t]{\minitwocolumn}
         \begin{center} #2
         \end{center}
     \end{minipage}
   \end{figure}
 }
%%%%%%%%%%%%%%%%%%%%%%%%%%%%%%%%%
%\vspace*{\baselineskip}
\begin{center}
	{\Large \bf 2019年度 構造力学I及び演習A 演習問題2 (10月8日)} \\
\end{center}
%%%%%%%%%%%%%%%%%%%%%%%%%%%%%%%%%%%%%%%%%%%%%%%%%%%%%%%%%%%%%%%%
\vspace{15mm}
図\ref{fig:fig1}-(a)のような,ヤング率$E$と断面積$A$が一定の一次元棒部材がある.
部材軸方向に作用する単位長さあたりの力$p$が同図(b)のように放物線の一部として与えられ,
$p_0$は正の定数であるとするとき,以下の問に答えよ.
なお,座標$x$の原点位置は, 計算に都合がよいよう各自で定めて良い.
\begin{enumerate}
\item
	変位$u(x)$が満足すべき微分方程式(支配方程式)と,部材端における支持条件(境界条件)を答えよ. 
\item
	変位$u(x)$を求めよ.
\item
	軸力$N(x)$を求めよ.
\item
	軸力$N(x)$のグラフ(軸力図)を描け.
\item
	変位$u(x)$のグラフを描け.
\item
	棒部材が固定壁から受ける力(反力)の大きさと向きを答えよ.
\end{enumerate}
\begin{figure}[h]
	\vspace{-3mm}
	\begin{center}
	\includegraphics[width=0.6\linewidth]{fig1.eps} 
	\end{center}
	\vspace{-5mm}
	\caption{(a) 両端を固定壁に支持された棒部材と,(b)それに作用する分布力.} 
	\label{fig:fig1}
\end{figure}
%%%%%%%%%%%%%%%%%%%%%%%%%%%%%%%%%%%%%%%%%%%%
\end{document}
