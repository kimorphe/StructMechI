\documentclass[10pt,a4j]{jarticle}
\usepackage{graphicx,wrapfig}
\setlength{\topmargin}{-1.5cm}
\setlength{\textwidth}{15.5cm}
\setlength{\textheight}{25.2cm}
\newlength{\minitwocolumn}
\setlength{\minitwocolumn}{0.5\textwidth}
\addtolength{\minitwocolumn}{-\columnsep}
%\addtolength{\baselineskip}{-0.1\baselineskip}
%
\def\Mmaru#1{{\ooalign{\hfil#1\/\hfil\crcr
\raise.167ex\hbox{\mathhexbox 20D}}}}
%
\begin{document}
\newcommand{\fat}[1]{\mbox{\boldmath $#1$}}
\newcommand{\D}{\partial}
\newcommand{\w}{\omega}
\newcommand{\ga}{\alpha}
\newcommand{\gb}{\beta}
\newcommand{\gx}{\xi}
\newcommand{\gz}{\zeta}
\newcommand{\vhat}[1]{\hat{\fat{#1}}}
\newcommand{\spc}{\vspace{0.7\baselineskip}}
\newcommand{\halfspc}{\vspace{0.3\baselineskip}}
\bibliographystyle{unsrt}
\pagestyle{empty}
\newcommand{\twofig}[2]
 {
   \begin{figure}[h]
     \begin{minipage}[t]{\minitwocolumn}
         \begin{center}   #1
         \end{center}
     \end{minipage}
         \hspace{\columnsep}
     \begin{minipage}[t]{\minitwocolumn}
         \begin{center} #2
         \end{center}
     \end{minipage}
   \end{figure}
 }
%%%%%%%%%%%%%%%%%%%%%%%%%%%%%%%%%
%\vspace*{\baselineskip}
\begin{center}
{\Large \bf 2019年度 構造力学I及び演習A 演習問題3(10月15日)} \\
\end{center}
%%%%%%%%%%%%%%%%%%%%%%%%%%%%%%%%%%%%%%%%%%%%%%%%%%%%%%%%%%%%%%%%
\vspace{5mm}
物体中の着目点$\fat{x}$におけるトラクション$\fat{t}^{(n)}$と応力$\fat{\sigma}$について考える.
物体は2次元的な応力状態にあるとし,ベクトルやテンソル成分を表すために, $x_1x_2$直角直交座標系を用いる.
また,単位法線ベクトル$\fat{n}(\theta)$の向きは,原点を中心に$x_1$軸から反時計回りの方向を
正として測った角度$\theta$で表す.
さらに,$\theta=\theta_1=\frac{\pi}{6}$としたときのトラクションベクトルを$\fat{t}_1$, 
単位法線ベクトルを$\fat{n}_1$とする.同様に,$\theta=\theta_2=\frac{\pi}{3}$の場合の
トラクションベクトルと単位法線ベクトルを, それぞれ$\fat{t}_2$および$\fat{n}_2$と表す. 
トラクションベクトルが以下のように与えられるとき, 以下の問に答えよ.
\[
	\fat{t}_1=
	\left(
		\begin{array}{c}
			-\sqrt{3} \\
			3
		\end{array}
	\right), \ \ 
	\fat{t}_2=
	\left(
		\begin{array}{c}
			1 \\
			\sqrt{3}
		\end{array}
	\right)
\]
\begin{enumerate}
\item
	単位法線ベクトル$\fat{n}_1, \fat{n}_2$の成分を
	\[
	\fat{n}_1=
	\left(
		\begin{array}{c}
			n_{11} \\
			n_{12}
		\end{array}
	\right), \ \ 
	\fat{n}_2=
	\left(
		\begin{array}{c}
			n_{21}\\
			n_{22}
		\end{array}
	\right)
	\]
	と表すとき,
	\[
		\fat{N}=\left\{ \fat{n}_1 \, \fat{n}_2 \right\} =
		\left(
		\begin{array}{cc}
			n_{11} & n_{21}  \\
			n_{12} & n_{22} 
		\end{array}
		\right)
	\]
	で与えられる行列$\fat{N}$と,その逆行列$\fat{N}^{-1}$を求めよ.
\item
	応力テンソル$\fat{\sigma}=\left\{\sigma_{ij}\right\}$の
	各成分を求め,$2\times 2$行列として表せ.
\item
	一般の単位法線ベクトル$\fat{n}(\theta)$に対するトラクションベクトル
	$\fat{t}^{(n)}$の,$x_1$および$x_2$方向成分を$\theta$を用いて表わせ.
\item
	$\fat{t}^{(n)}$の$\fat{n}$方向成分を$\sigma_n$とするとき,
	$\sigma_n$を$\cos 2\theta$と$\sin 2\theta$を用いて表せ.
\item
	法線ベクトル$\fat{n}$に直交する単位ベクトル$\fat{m}$を
	図\ref{fig:fig1}のようにとる.
	$\fat{t}^{(n)}$の$\fat{m}$方向成分を$\tau$とするとき,
	$\tau$を$\cos 2\theta$と$\sin 2\theta$を用いて表せ.
\item
	$\theta$の範囲を$0$から$\pi$とするとき,$\sigma_n$の最大値と, 
	そのときの$\tau$を求めよ.
\item
	$\theta$の範囲を$0$から$\pi$するとき,横軸に$\sigma_n$,縦軸に$\tau$をとった平面上で, 点$(\sigma_n, \tau)$が描く曲線を図示せよ.
\end{enumerate}
%--------------------
\begin{figure}[h]
	\begin{center}
	\includegraphics[width=0.45\linewidth]{fig1.eps} 
	\end{center}
	\caption{
		2次元空間中の物体内部にとった着目点$\fat{x}$と
		トラクション$\fat{t}^{(n)}$.
		破線は, $\fat{t}^{(n)}$に対応する仮想的な切断面を表す.	
	} 
	\label{fig:fig1}
\end{figure}
\end{document}
