\documentclass[10pt,a4j]{jarticle}
%\usepackage{graphicx,wrapfig}
\usepackage{graphicx}
\setlength{\topmargin}{-1.5cm}
\setlength{\textwidth}{15.5cm}
\setlength{\textheight}{25.2cm}
\newlength{\minitwocolumn}
\setlength{\minitwocolumn}{0.5\textwidth}
\addtolength{\minitwocolumn}{-\columnsep}
%\addtolength{\baselineskip}{-0.1\baselineskip}
%
\def\Mmaru#1{{\ooalign{\hfil#1\/\hfil\crcr
\raise.167ex\hbox{\mathhexbox 20D}}}}
%
\begin{document}
\newcommand{\fat}[1]{\mbox{\boldmath $#1$}}
\newcommand{\D}{\partial}
\newcommand{\w}{\omega}
\newcommand{\ga}{\alpha}
\newcommand{\gb}{\beta}
\newcommand{\gx}{\xi}
\newcommand{\gz}{\zeta}
\newcommand{\vhat}[1]{\hat{\fat{#1}}}
\newcommand{\spc}{\vspace{0.7\baselineskip}}
\newcommand{\halfspc}{\vspace{0.3\baselineskip}}
\bibliographystyle{unsrt}
\pagestyle{empty}
\newcommand{\twofig}[2]
 {
   \begin{figure}[here]
     \begin{minipage}[t]{\minitwocolumn}
         \begin{center}   #1
         \end{center}
     \end{minipage}
         \hspace{\columnsep}
     \begin{minipage}[t]{\minitwocolumn}
         \begin{center} #2
         \end{center}
     \end{minipage}
   \end{figure}
 }
%%%%%%%%%%%%%%%%%%%%%%%%%%%%%%%%%
%\vspace*{\baselineskip}
\begin{center}
{\Large \bf 2018年度 構造力学I及び演習A 演習問題4(10月23日)} \\
\end{center}
%%%%%%%%%%%%%%%%%%%%%%%%%%%%%%%%%%%%%%%%%%%%%%%%%%%%%%%%%%%%%%%%
\vspace{15mm}
	図\ref{fig:fig1}に示すような,$o-x_1 y_1$および$o-x_1'x_2'$座標系に
	おける応力テンソルをそれぞれ
\[
	\fat{\sigma}
	=
	\left( 
		\begin{array}{cc}
		\sigma_{11} & \sigma_{12} \\
		\sigma_{12} & \sigma_{22} 
		\end{array}
	\right)
	, 
	\ \
	\fat{\sigma}'
	=
	\left( 
		\begin{array}{cc}
		\sigma_{11}' & \sigma_{12}' \\
		\sigma_{12}' & \sigma_{22}'
		\end{array}
	\right)
\]
	と表す.$\fat{\sigma}$が
\begin{equation}
	\fat{\sigma}
	=
	\left( 
		\begin{array}{cc}
		3 & 2\sqrt{3} \\
		2\sqrt{3} &  -1
		\end{array}
	\right)
	\label{eqn:sigma}
\end{equation}
であるとき,以下の問に答えよ.なお,$\theta$は$0 \leq \theta < \pi$において
任意の角度をとりうるとする.
\begin{enumerate}
\item
	$\fat{\sigma}'$の各成分を求め, その結果を$2\theta$を用いて表せ.
\item
	$\sigma_{11}'$の最大値を$\sigma_1$, そのときの$\theta$を$\theta_1$とする。
	このとき、$\sigma_1$と$\theta_1$を求めよ.
\item
	$\theta=\theta_1$となるときの$\sigma_{22}'$と$\sigma_{12}'$を求めよ.
\item
	$\sigma_{12}'$の最大を$\tau$,そのときの$\theta$を$\theta_\tau$とする。
	このとき、$\tau$と$\theta_\tau$を求めよ.
\item
	$\theta=\theta_\tau$となるときの$\sigma_{11}'$と$\sigma_{22}'$を求めよ.
\item
	$o-x_1x_2$座標系において, 法線ベクトル$\fat{n}$が
	\[
		\fat{n}=\left( 
		\begin{array}{c}
			-\sin \theta \\
			\cos \theta
		\end{array}
		\right)
	\]
	となる面,すなわち$x_2'$軸を法線とする面について考える.
	この面に発生するトラクションベクトルが, $o-x_1'x_2'$座標系において
	どのように表されるか答えよ.ただし,応力テンソルは式(\ref{eqn:sigma})で
	与えられた通りとする.
\end{enumerate}
%--------------------
%\begin{figure}[here]
\begin{figure}[h]
	\begin{center}
	\includegraphics[width=0.4\linewidth]{fig1.eps} 
	\end{center}
	\vspace{-5mm}
	\caption{二つの直角直交座標系.} 
	\label{fig:fig1}
\end{figure}
\end{document}
