\documentclass[10pt,a4j]{jarticle}
\usepackage{graphicx,wrapfig}
\setlength{\topmargin}{-1.5cm}
\setlength{\textwidth}{15.5cm}
\setlength{\textheight}{25.2cm}
\newlength{\minitwocolumn}
\setlength{\minitwocolumn}{0.5\textwidth}
\addtolength{\minitwocolumn}{-\columnsep}
%\addtolength{\baselineskip}{-0.1\baselineskip}
%
\def\Mmaru#1{{\ooalign{\hfil#1\/\hfil\crcr
\raise.167ex\hbox{\mathhexbox 20D}}}}
%
\begin{document}
\newcommand{\fat}[1]{\mbox{\boldmath $#1$}}
\newcommand{\D}{\partial}
\newcommand{\w}{\omega}
\newcommand{\ga}{\alpha}
\newcommand{\gb}{\beta}
\newcommand{\gx}{\xi}
\newcommand{\gz}{\zeta}
\newcommand{\vhat}[1]{\hat{\fat{#1}}}
\newcommand{\spc}{\vspace{0.7\baselineskip}}
\newcommand{\halfspc}{\vspace{0.3\baselineskip}}
\bibliographystyle{unsrt}
\pagestyle{empty}
\newcommand{\twofig}[2]
 {
   \begin{figure}[here]
     \begin{minipage}[t]{\minitwocolumn}
         \begin{center}   #1
         \end{center}
     \end{minipage}
         \hspace{\columnsep}
     \begin{minipage}[t]{\minitwocolumn}
         \begin{center} #2
         \end{center}
     \end{minipage}
   \end{figure}
 }
%%%%%%%%%%%%%%%%%%%%%%%%%%%%%%%%%
%\vspace*{\baselineskip}
\begin{center}
{\Large \bf 2019年度 構造力学I及び演習A 演習問題5(11月5日)} \\
\end{center}
%%%%%%%%%%%%%%%%%%%%%%%%%%%%%%%%%%%%%%%%%%%%%%%%%%%%%%%%%%%%%%%%
\vspace{15mm}
応力テンソル$\fat{\sigma}$が, $o-x_1x_2$直角直交座標系において
\begin{equation}
	\fat{\sigma}
	=
	\left( 
		\begin{array}{cc}
		-4  & 2\sqrt{3} \\
		2\sqrt{3} & 0 
		\end{array}
	\right)
	\label{eqn:sigma}
\end{equation}
で与えられるとき,以下の問に答えよ.
なお,ベクトルや面の方向は,$x_1$座標から反時計回りの方向を
正として測った角度で表すものとする.
\begin{enumerate}
\item
	モールの応力円を描け.
\item
	式(\ref{eqn:sigma})で与えられる応力成分は,モールの応力円上
	のどの点に相当するか答えよ.
\end{enumerate}
\begin{enumerate}
\setcounter{enumi}{2}
\item
	最大主応力$\sigma_{max}$とその方向$\theta_{max}$を求めよ.
\item
	最小主応力$\sigma_{min}$とその方向$\theta_{min}$を求めよ.
\item
	最大せん断応力$\tau_{max}$とそれを与える面の方向$\theta_\tau$を求めよ.
%\item
%	$\cos\theta_{max}$と$\sin\theta_{max}$を求めよ.
\item
	応力テンソル$\fat{\sigma}$の二つの固有値$\lambda_1,\lambda_2$を求めよ.
\item
	固有値$\lambda_1,\lambda_2$に対応する固有ベクトル$\fat{n}_1, \fat{n}_2$を求めよ.
	ただし,$\fat{n}_1,\fat{n}_2$は単位ベクトルとして表すこと.		
\item
	最大主応力面に発生するトラクションベクトル$\fat{t}^{(n)}$が, 
	$o-x_1x_2$座標系でどのように表されるか答えよ.
\end{enumerate}
\end{document}
%--------------------
\subsubsection*{問題2.}
引張力に対する強度が非常に小さな土などの材料では,せん断応力$\tau$と
直応力$\sigma_n$が,
\begin{equation}
	\left| \tau \right| < -\sigma_{n} \tan \phi 
	\label{eqn:ineq}
\end{equation}
の条件を満足する限り破壊しないというモデルで,材料の強度を表現すること
ができる.このとき,式(\ref{eqn:ineq})の$\phi (>0)$は内部摩擦角
と呼ばれる.いま,式(\ref{eqn:ineq})の破壊基準に従う材料に対して,
ある直交座標系($o-x_1x_2$座標系)で次のように表される応力を加えたとする.
\begin{equation}
	\fat{\sigma} = \left( 
	\begin{array}{cc}
			-p+R & 0 \\
			0  & -p-R 
	\end{array}
	\right)
\end{equation}
ただし,$p$および$R$はいずれも正で,$p$は一定, 応力は引張を正とする.
$R$をゼロから次第に大きくしたところ, 材料は$R=\frac{1}{2}p$に
達したときに破壊した.このとき,内部摩擦角$\phi$および破断面の方向を答えよ.
%--------------------
\end{document}
\begin{figure}[here]
	\begin{center}
	\includegraphics[width=0.8\linewidth]{fig1.eps} 
	\end{center}
	\caption{(a)二つの直角直交座標系.(b)鉛直および水平方向の外力を受ける部材.} 
	\label{fig:fig1}
\end{figure}
