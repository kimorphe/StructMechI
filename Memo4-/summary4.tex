\documentclass[10pt,a4j]{jarticle}
%\usepackage{graphicx,wrapfig}
\usepackage{graphicx,amsmath}
\setlength{\topmargin}{-1.5cm}
\setlength{\textwidth}{16.5cm}
\setlength{\textheight}{25.2cm}
\newlength{\minitwocolumn}
\setlength{\minitwocolumn}{0.5\textwidth}
\addtolength{\minitwocolumn}{-\columnsep}
%\addtolength{\baselineskip}{-0.1\baselineskip}
%
\def\Mmaru#1{{\ooalign{\hfil#1\/\hfil\crcr
\raise.167ex\hbox{\mathhexbox 20D}}}}
%
\begin{document}
\newcommand{\fat}[1]{\mbox{\boldmath $#1$}}
\newcommand{\D}{\partial}
\newcommand{\w}{\omega}
\newcommand{\ga}{\alpha}
\newcommand{\gb}{\beta}
\newcommand{\gx}{\xi}
\newcommand{\gz}{\zeta}
\newcommand{\vhat}[1]{\hat{\fat{#1}}}
\newcommand{\spc}{\vspace{0.7\baselineskip}}
\newcommand{\halfspc}{\vspace{0.3\baselineskip}}
\bibliographystyle{unsrt}
%\pagestyle{empty}
\newcommand{\twofig}[2]
 {
   \begin{figure}
     \begin{minipage}[t]{\minitwocolumn}
         \begin{center}   #1
         \end{center}
     \end{minipage}
         \hspace{\columnsep}
     \begin{minipage}[t]{\minitwocolumn}
         \begin{center} #2
         \end{center}
     \end{minipage}
   \end{figure}
 }
%%%%%%%%%%%%%%%%%%%%%%%%%%%%%%%%%
%\vspace*{\baselineskip}
\begin{center}
	{\Large \bf 2018年度 構造力学I及び演習B 講義内容のまとめ1} \\
\end{center}
%%%%%%%%%%%%%%%%%%%%%%%%%%%%%%%%%%%%%%%%%%%%%%%%%%%%%%%%%%%%%%%%
\section{はりのたわみとたわみ角}
モーメントを伝達する部材を梁(はり)部材とよぶ.
梁部材にモーメントが生ずる典型的な状況は,部材長手方向に対して垂直な荷重
が作用するときである.

図\ref{fig:fig7_0}に示すような梁ABが, 鉛直下向きの分布荷重を受ける場合について考える.
梁部材の断面位置を参照するために,梁の長手方向に$x=x_1$軸を,鉛直下向きに
$y=x_2$軸を取る.梁は実際には奥行き方向にも広がりを持つため,$z=x_3$軸は
$xyz$座標が右手系となるよう,紙面奥行き方向にとる.
座標原点位置は, 後に述べるように,中立面内であればどこでもよい.
このように設定した$xyz$座標系において,梁に加えれられた分布荷重を
位置の関数として$q(x)$と表す.今後明らかになるように,一点に集中して
加えられた荷重(点荷重あるいは集中荷重)も,ディラクのデルタ関数
を用いることで,分布荷重の一種として取り扱うことができる.
\begin{figure}[h]
	\begin{center}
	\includegraphics[width=0.6 \linewidth]{fig7_0.eps} 
	\end{center}
	\caption{
		鉛直下向きの分布荷重$q(x)$を受けて変形する梁AB.
		載荷に伴う変形前後の様子を表す. 
	} 
	\label{fig:fig7_0}
\end{figure}

梁に対して垂直にに外力が加えられたときに生じる変形は,梁全体がどのように
支持されているかにも依存する.今,梁は,その端部において回転を拘束しない
ような方法で支持されているとする.
梁を支持する機構や支持点のことを支点とよぶ.ここでは,支点が沈下する場合
も許容すれば,鉛直力を受けた梁は図\ref{fig:fig7_0}下の図のように変形すると予想される.
ただし,議論を簡単にするために梁に生じる変形は非常に小さいと仮定し,
図\ref{fig:fig7_0}は変形状況をわかり易く表現するために誇張して描かれていると解釈する.


幅$\Delta x$の微小区間abdcに着目すると,当初長方形であった領域が
変形後は図\ref{fig:fig7_1}に示すような下に凸な曲線で囲まれた形状に
なると考えられる.ここで,鉛直面adとbcは変形後も直線(平面)をのままである
ことを仮定している.このような仮定のことを"平面保持の仮定"と言う.一方,下
縁部abと上縁部cdは変形の結果湾曲し, それぞれ伸びと縮みを生じている.
このことと平面保持の仮定を踏まえれば,鉛直方向のどこかで伸びも縮もしない
中立状態を保つ位置efが存在すると考えられる.
変形後も中立の状態を保つ点全体はある曲面を描き,これを中立面と呼ぶ.
図\ref{fig:fig7_1}-(a)に示すように,$y=0$,すなわち$y$軸の原点は,
変形前の状態における中立面内にとる.
変形後の梁における中立面を,$xy$平面に投影して得られる曲線$y=v(x)$は,たわみ曲線と呼ばれる.
与えられた荷重と支持条件に対してたわみ曲線が求められれば,梁の変形状態が
くまなく分かったことになる.たわみ曲線の勾配を$\theta(x)$と書き,これを
たわみ角と呼ぶ.たわみ角は,時計回りの方向が正となっていることに注意する.
また,変形が小さいことを考慮し,たわみ角とたわみの関係は以下の近似式で
代用する.たわみ$v$とたわみ角$\theta$の関係を図\ref{fig:fig7_2}-(a)に示す.
\begin{equation}
	\theta(x) \simeq \tan \theta =\frac{dv}{dx}
	\label{eqn:th_apprx}
\end{equation}
\begin{figure}
	\begin{center}
	\includegraphics[width=0.8\linewidth]{fig7_1.eps} 
	\end{center}
	\caption{
		梁の微小区間abdcの変形前後の様子.
		(a)変形前, (b)変形後.破線efは,変形によって伸縮しない
		面(中立面)の位置を表す.
	} 
	\label{fig:fig7_1}
\end{figure}
\begin{figure}
	\begin{center}
	\includegraphics[width=0.85\linewidth]{fig7_2.eps} 
	\end{center}
	\caption{
	(a)梁のたわみ$v(x)$とたわみ角$\theta(x)$. 
	(b)微小区間$\left[x, \, x+\Delta x \right]$におけるたわみと
	たわみ角の変化,ならびに曲率半径.
	 } 
	\label{fig:fig7_2}
\end{figure}
以下ではたわみ$v(x)$を求めるための方程式系を導き,その解法について学習する.
\section{曲げ変形におけるたわみと曲率の関係}
以上に述べたような変形の状態で,梁は区分的には上あるいは下に凸な形状に
曲げられている.従って,このとき梁の内部に生じているモーメントを特に曲げモーメント,
変形を曲げ変形と呼ぶ.曲げモーメントの正確な定義は次節において述べる.
まず,梁部材の各点において生じている曲げ変形の程度を定量的(数値的)に表現する
ために,たわみ曲線$v(x)$の位置$x$における曲率$\kappa(x)$, あるいは曲率半径
$\rho(x)=\kappa^{-1}(x)$について調べる.
曲率円とは,曲線の局所的な曲がり具合を最もよく近似する円のことを意味し,
曲率半径$\rho$はそのような円の半径を, 曲率$\kappa$は$\rho$の逆数を表す.
局所的な変形の大小と$\rho, \kappa$の関係は表\ref{tbl:tbl7_1}のようになる.
\begin{table}
\label{tbl:tbl7_1}
\caption{曲率,曲率半径と曲げ変形の大小}
\begin{center}
\begin{tabular}{c|c|c}
小& 曲げ変形 & 大\\
\hline\hline
大&曲率半径$\rho$ & 小\\
\hline
小& 曲率$\kappa$& 大\\
\end{tabular}
\end{center}
\end{table}
いま,微小区間$[x,\, x+\Delta x]$におけるたわみ曲線に関する曲率
円を図示すると図\ref{fig:fig7_2}-(b)のようになる.弧efの開き角は, 
当該区間におけるたわみ角$\theta(x)$の変化量$\Delta \theta (x)$に
よって表すことができる.$v(x)$は$\Delta \theta >0$で上に凸,$\Delta \theta<0$
で下に凸であることから,正の量となるべき開き角は$-\Delta \theta$であることに
注意する. 区間$[x,\,x+\Delta x]$におけるたわみ曲線の弧長を$\Delta s$とすれば,
\begin{equation}
	\Delta s(x) = \rho (x) \left(-\Delta \theta (x) \right)
	\label{eqn:sx_of_rho}
\end{equation}
である.式(\ref{eqn:sx_of_rho})において$\Delta x \rightarrow 0$の極限を取れば,
\begin{equation}
	\kappa = \frac{1}{\rho}=-\frac{d\theta}{ds}	
\end{equation}
が得られ,これに式(\ref{eqn:th_apprx})を用いれば,概ね
\begin{equation}
	\kappa=\frac{1}{\rho}\simeq \frac{d^2 v}{dx^2}=v''(x)
	\label{eqn:kpp_v2}
\end{equation}
となり,たわみと曲率を関係づける式が得られる.目的はたわみを求めることにあるが,
以下に見るように$\rho$や$\kappa$は着目断面に生じる曲げ変形や,それを引き起こす
応力の分布(曲げ応力分布)を記述する上で有用である.
なお,曲率半径とたわみの2階微分は厳密には一致しない.両者の関係の正確な導出は
以下のようである.
\subsubsection*{曲率$\kappa$-たわみ$v$関係の正確な導出}
以下,図\ref{fig:fig7_2}-(b)を参照.
\begin{equation}
	\rho  \times \left( { - \Delta \theta } \right) = \Delta s
\end{equation}
より,曲率は
\[
	\kappa  = \frac{1}{\rho } =  - \frac{{\Delta \theta }}{{\Delta s}} =  - \frac{{{{\Delta \theta } \mathord{\left/
 {\vphantom {{\Delta \theta } {\Delta x}}} \right.
 \kern-\nulldelimiterspace} {\Delta x}}}}{{{{\Delta s} \mathord{\left/
 {\vphantom {{\Delta s} {\Delta x}}} \right.
 \kern-\nulldelimiterspace} {\Delta x}}}} \to  - \frac{{\frac{{d\theta }}{{dx}}}}{{\frac{{ds}}{{dx}}}}
\]
と書ける.ただし$\rightarrow$は$\Delta x \rightarrow 0$の極限をとること
を表す.
ここで,
\begin{equation}
	\frac{{ds}}{{dx}} = \frac{d}{{dx}}\sqrt {d{x^2} + d{y^2}}  = \sqrt {1 + {{\left( {\frac{{dy}}{{dx}}} \right)}^2}}  = \sqrt {1 + {{\left( {v'} \right)}^2}} 
\end{equation}
であり,$v'=\tan \theta$より
\begin{equation}
	\frac{{dv'}}{{dx}} = \frac{d}{{dx}}\tan \theta  = \frac{1}{{{{\cos }^2}\theta }}\frac{{d\theta }}{{dx}} = \left( {1 + {{\tan }^2}\theta } \right)\frac{{d\theta }}{{dx}} = \left\{ {1 + {{\left( {v'} \right)}^2}} \right\}\frac{{d\theta }}{{dx}}
\end{equation}
となることから,曲率$\kappa=\frac{1}{\rho}$とたわみの関係は
\begin{equation}
	k =  - \frac{{v''}}{{{{\sqrt {1 + {{\left( {v'} \right)}^2}} }^3}}} \approx  - v'',\quad \left( {\left| {v'} \right| \ll 1} \right)
\end{equation}
となることが示される.
\section{曲げ問題における断面力}
\subsection{断面力の定義}
変形が生じている梁の断面には,一般に,一様でない直応力とせん断応力が発生している.
梁の奥行き方向の幅は,梁の長さに比べて十分小さいとき,梁は平面応力状態にあると考えらる.
この場合,考慮すべき応力成分は
$\sigma_{xx}(=\sigma_{11}),\sigma_{yy}(=\sigma_{22})$と
$\sigma_{xy}=(\sigma_{12})$の3つである.
このうち,$x(=x_1)$軸に垂直な断面に関与する成分は,図\ref{fig:fig7_3}-(a)に示すように,
$\sigma_{xx}$と$\sigma_{xy}$である.
応力は,力の釣り合い条件を満足する必要がある.釣り合い条件式を書き下にあたり,これらの
応力が断面におよぼす合力と合モーメントを求めておくと都合がよい.
そこで,$\sigma_{xx}$に関する断面全体での合力を$N$,$\sigma_{xy}$の合力を$Q$と表す.
$N$は軸力,$Q(x)$はせん断力と呼ばれる.軸力問題では,軸力$N$から直応力を定義したが,
応力$\sigma_{xx}$が断面内で一様に分布していない場合,本節で述べる定義を採用する
必要がある.
一方,$\sigma_{xx}$がつくる中立面位置$y=0$に関する合モーメントを$M$と書き,
これを曲げモーメントと呼ぶ.$N,Q,M$は断面力と総称され,これらは以下のように
定義される(図\ref{fig:fig7_3}-(b)).
\begin{eqnarray}
	N &=& \int_S \sigma_{xx} dS 
	\label{eqn:def_N}\\
	Q &=& \int_S \sigma_{xy} dS
	\label{eqn:def_Q}\\
	M &=& \int_S y\sigma_{xx} dS
	\label{eqn:def_M}
\end{eqnarray}
ここに$\int_S(\cdot)dS$は梁の断面$S$全体を積分範囲とする面積積分(2重積分)を, 
$dS$は微小面積要素をそれぞれ意味する.これらの面積積分を$yz$直角直交座標系に
おいて評価する場合,微小面積要素は$dS=dydz$である.面積積分の具体的な計算方法は,
断面2次モーメントの計算方法に関する節で詳しく説明する.
\subsection{断面力の釣り合い式}
微小区間$[x,\,x+\Delta x]$における断面力の釣り合い式は以下のように導出される.
断面力$N,Q$および$M$を用い, 微小区間の自由物体図を描くと,図\ref{fig:fig7_3}-(c)に示すようになる.
ここで,微小区間の右側面は$x$軸の正方向,左側面は負の方向を向く面であることから,
各々の断面に作用する断面力の正方向が互いに反対となるように定められていることに注意する.
この図を参照して水平方向,鉛直方向,およびモーメントの釣り合い条件式を
立てれば,次の結果が得られる.
\begin{eqnarray}
	&&-N(x)+N(x+\Delta x) =0 \label{eqn:equib_x} \\
	&&-Q(x)+Q(x+\Delta x)+q(x)\Delta x =0 \label{eqn:equib_y} \\
	&&-M(x)+M(x+\Delta x)-Q(x)\Delta x +q(x)\Delta x \times \frac{\Delta x}{2}=0 \label{eqn:equib_th}
\end{eqnarray}
ただし,モーメントの釣り合い式は,右側面の中立面位置$(x+\Delta x,\, y=0)$を基準とした場合のものである.
これらの関係式の両辺を$\Delta x$で割り,$\Delta x \rightarrow =0$の極限を取れば,
\begin{eqnarray}
	\frac{dN}{dx}&=&0  \label{eqn:equib_N} \\
	\frac{dQ}{dx}&=&-q(x) \label{eqn:equib_Q} \\
	\frac{dM}{dx}&=&Q(x) \label{eqn:equib_M}
\end{eqnarray}
となる.これらが,断面力の釣り合い条件を表す.
%%%%%%%%%%%%%%%%%%%%%%%%%%%%%%%%%%%%%%%%%%%%%%%%%%%%%%%%%%%%%%%%
\section{たわみの支配方程式}
\subsection{曲げひずみと,曲げ応力}
図\ref{fig:fig7_1}-(b)のような変形状態にある部分では,中立面から下の範囲$y(>0)$で
部材には伸びは生じている.つまり,$x$方向の直ひずみ$\varepsilon_{xx}(=\varepsilon_{11})$
は正であると考えられる.実際,中立面から下に$y$だけ離れた部分の長さは,変形前は
$\Delta x=\rho (-\Delta \theta)$,変形後は$(\rho + y )(-\Delta \theta)$である.
よって,$\varepsilon_{xx}$は
\begin{equation}
	\varepsilon_{xx}
	=
	\frac{(\rho +y )(-\Delta \theta) - \rho (-\Delta \theta)}{\rho (-\Delta \theta)}
	=
	\frac{y}{\rho}
\end{equation}
となり,$\rho>0$のとき$y>0$ならば$\varepsilon_{xx}$も正となる.
$\varepsilon_{xx}$は一般化されたフックの法則によれば,直応力成分と
\begin{equation}
	\varepsilon_{xx}=\frac{\sigma_{xx}-\nu(\sigma_{yy}+\sigma_{zz})}{E}
\end{equation}
の関係にあるが,ここではポアソン比$\nu$の効果は小さいと仮定すれば,
$\varepsilon_{xx}=\sigma_{xx}/E$となる.この関係を仮定して直応力の分布を書けば,
\begin{equation}
	\sigma_{xx}=\frac{E}{\rho}y
	\label{eqn:sxx_y}
\end{equation}
と,応力は$y$に関して直線的に変化するとの結果が得られる.このような応力分布を曲げ応力分布と呼ぶ.
これは,平面保持とポアソン比による影響を無視した結果の帰結である.
次に,式\ref{eqn:sxx_y}を式(\ref{eqn:def_M})に代入すると,
\begin{equation}
	M=\int_S \frac{M}{\rho}y^2 dS=\frac{E}{\rho}I
	\label{eqn:M_rho}
\end{equation}
となる.ただし,$I$は
\begin{equation}
	I=\int_S y^2 dS
	\label{eqn:def_I}
\end{equation}
と定義される量で,中立面に関する"断面2次モーメント"と呼ばれる.
ここで,式(\ref{eqn:M_rho})において曲げモーメント$M$と曲率半径$\rho$
は$x$にのみ依存し, $y$と$z$には依らない量であることに注意する.
次に,式(\ref{eqn:M_rho})に,曲率$\kappa$とそのたわみとの関係(\ref{eqn:kpp_v2})を用いれば
\begin{equation}
	M = EI \kappa = -EIv''
	\label{eqn:M_kpp}
\end{equation}
が得られる.この式は,曲げモーメント$M$と曲げ変形$\kappa$が比例係数$EI$を介して
結び付けられることを表している.$EI$は梁の曲げ剛性と呼ばれ,部材の曲げに対する
固さを表す指標と理解することができる.最後に,式(\ref{eqn:M_kpp})に
式(\ref{eqn:equib_Q})と式(\ref{eqn:equib_M})を用いれば,
\begin{equation}
	\frac{d^2}{dx^2}\left\{ EI \left(\frac{d^2v}{dx^2}\right)\right\}=q(x)
	\label{eqn:gveq_v}
\end{equation}
となり,たわみの支配微分方程式が得られる.この式を適切な支持条件の元で解くことにより,
梁のたわみが得られる.また,たわみが得られれば,その結果を繰り返して微分することで,順次
たわみ角$\theta$, 曲げモーメント$M$, せん断力$Q$を求めることができる.
このことを具体的に書くと,次のようである.
\begin{eqnarray}
	\theta(x) &=& v'(x) \\
	M(x) &=& -EIv '' \\
	Q(x) &=& M'(x) = \left(-EIv''\right)' 
\end{eqnarray}
なお,曲げ剛性$EI$が位置に依らず一定の場合,式(\ref{eqn:gveq_v})は単に
\begin{equation}
	\frac{d^4v}{dx^4}=\frac{q}{EI}
	\label{eqn:gveq_uniform}
\end{equation}
となり,$Q(x)=M'(x)=-EIv'''(x)$でもある.ここで,$q(x)/EI$の$x$に関する4回の不定積分を
\[
	\iiiint \frac{q(x)}{EI}dx^4
\]
と表せば,式\ref{eqn:gveq_uniform}の解は,$A_1\sim A_4$を任意の定数として
\begin{equation}
	v(x)= 
	\iiiint \frac{q(x)}{EI}dx^4
	+A_1 x^3 +A_2x^2 + A_3x + A_4
	\label{eqn:vx_gsol}
\end{equation}
と表すことができる.従って,$v(x)$に関する4つの互いに独立な条件が与え
られれば,積分定数$A_1\sim A_4$が決定でき,たわみ$v(x)$を求めることができる.
%--------------------
%%%%%%
\subsection{問題}\label{prb}
図\ref{fig:fig7_4}に示すような,長さ$l$で曲げ剛性$EI$が一定の梁を考える.
梁に外力は作用せず,梁の両端で,たわみとたわみ角が指定された状態で固定されている.
このとき,以下4通りの支持条件のもとで生じるたわみ分布を求めよ.
\begin{enumerate}
\item
$v(0)=1,\, v(l)=0,\, v'(0)=0,\, v'(l)=0$
\item
$v(0)=0,\, v(l)=1,\, v'(0)=0,\, v'(l)=0$
\item
$v(0)=0,\, v(l)=0,\, v'(0)=1,\, v'(l)=0$
\item
$v(0)=0,\, v(l)=0,\, v'(0)=0,\, v'(l)=1$
\end{enumerate}
\subsubsection*{問題\ref{prb}-1の解答}
梁に外力は働かないので,たわみ$v(x)$は式(\ref{eqn:vx_gsol})において$q(x)\equiv 0$とし,
\begin{equation}
	v(x)=A_1 x^3 +A_2x^2 + A_3x + A_4
\end{equation}
と表すことができる.積分定数$A_1\sim A_4$のうち$A_3$と$A_4$は, $v(0)=1$より$A_4=1$,
$v'(0)=0$より$A_3=0$と決まる.一方,$A_1$と$A_2$は
\[
	v(l)=A_1l^3+A_2l^2+1=0, \ \ 
	v'(l)=3A_1l^2+2A_2l=0
\]
より, 
\[
	A_1=\frac{2}{l^3}, \ \ A_2=-\frac{3}{l^2}
\]
と求まる.
\[
	v(x)=
	2\left(\frac{x}{l}\right)^3
	-
	3\left(\frac{x}{l}\right)^2
	+
	1
	=\left(\frac{x}{l}-1\right)^2\left\{ 2\left(\frac{x}{l}\right)+1\right\}
\]
この結果を微分することで,たわみ角,曲げモーメント,せん断力が以下のように順に求められる.
\[
	v'(x)=\frac{6}{l}\left\{ \left(\frac{x}{l}\right)^2-\left(\frac{x}{l} \right) \right\}
\]
\[
	M(x)=-\frac{6EI}{l^2}\left\{ 2\left(\frac{x}{l}\right)-1 \right\}
\]
\[
	Q(x)=-\frac{12EI}{l^3}
\]
以上で求めた曲げモーメント分布とせん断力分布をグラフとして示すと,
図\ref{fig:fig7_5}のようになる.このように断面力を示すグラフを
断面力図と呼び,特に,曲げモーメント分布を示すグラフを曲げモーメント図,
せん断力を示す図をせん断力図と呼ぶ.
\begin{figure}
	\begin{center}
	\includegraphics[width=0.6\linewidth]{fig7_3.eps} 
	\end{center}
	\caption{
	(a)梁の断面に作用する直応力$\sigma_{xx}$と
	せん断断応力$\sigma_{xy}$の正方向,
	(b)断面力, 
	(c)断面力のつり合い条件を求めるための自由物体図. 
	 } 
	\label{fig:fig7_3}
\end{figure}
\begin{figure}
	\begin{center}
	\includegraphics[width=0.5\linewidth]{fig7_4.eps} 
	\end{center}
	\caption{両端のたわみとたわみ角が固定された梁AB.} 
	\label{fig:fig7_4}
\end{figure}
\begin{figure}
	\begin{center}
	\includegraphics[width=0.8\linewidth]{fig7_5.eps} 
	\end{center}
	\caption{(a)曲げモーメント図と(b)せん断力図.} 
	\label{fig:fig7_5}
\end{figure}
\section{梁の支持条件}
梁を支持する方法(支持条件)には種々のものが考えられる.その中で, 図\ref{fig:fig8_1}の4種類が基本的である.
この図では,左側2つのカラムに支持条件の名称と支持方法のイメージが示されている.
このようなイメージを表す図は,直感的には理解しやすいものの,描画に手間がかかること, 
描き方に個人差が生じうることから,各々の支持条件を図示する場合には第3カラムに示した
記号が用いられる.この中で,(b)ピン支持と(c)ローラー支持条件の記号は互いによく似ているため,
いずれであるかを取り違えることがないよう注意が必要である.
三角形とその下に引かれた横線の間に隙間がある場合はローラー支持を, そうでない場合
はピン支持を意味する.なお,これらの支点において成立すべき力学的条件は、
図\ref{fig:fig8_1}の右半分に示した通りで,変形に関してはたわみ$v$や軸変位$u$, たわみ角$\theta$が, 
力とトルクに関してはせん断力$Q$や曲げモーメント$M$,軸力$N$等の量が, 支点位置において
ゼロになることが示されている.なお,$v$と$Q$, $\theta$と$M$, $u$と$N$は互いに共役な量であり,
一方が指定されているとき,他方は未知量となる.例えば,ピン支持の場合, 支点上で$v=0$である
一方$Q$は未知である.これは,たわみをゼロとするためには,その点において一般にはゼロでないせん断力が
必要とされるためである.同様に,ピン支点においては, $M=0$に対して$\theta$は未知, 
$u=0$である一方軸力$N$未知で一般にゼロでない.
\begin{figure}
	\begin{center}
	\includegraphics[width=1.0\linewidth]{fig8_1.eps} 
	\end{center}
	\caption{梁端部における4種類の基本的な支持条件. $*$は,一般にゼロでない値であることを表す.}
	\label{fig:fig8_1}
\end{figure}
\end{document}
