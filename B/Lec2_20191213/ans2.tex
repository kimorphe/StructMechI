\documentclass[10pt,a4j]{jarticle}
\usepackage{graphicx,wrapfig}
\setlength{\topmargin}{-1.5cm}
\setlength{\textwidth}{15.5cm}
\setlength{\textheight}{25.2cm}
\newlength{\minitwocolumn}
\setlength{\minitwocolumn}{0.5\textwidth}
\addtolength{\minitwocolumn}{-\columnsep}
%\addtolength{\baselineskip}{-0.1\baselineskip}
%
\def\Mmaru#1{{\ooalign{\hfil#1\/\hfil\crcr
\raise.167ex\hbox{\mathhexbox 20D}}}}
%
\begin{document}
\newcommand{\fat}[1]{\mbox{\boldmath $#1$}}
\newcommand{\D}{\partial}
\newcommand{\w}{\omega}
\newcommand{\ga}{\alpha}
\newcommand{\gb}{\beta}
\newcommand{\gx}{\xi}
\newcommand{\gz}{\zeta}
\newcommand{\vhat}[1]{\hat{\fat{#1}}}
\newcommand{\spc}{\vspace{0.7\baselineskip}}
\newcommand{\halfspc}{\vspace{0.3\baselineskip}}
\bibliographystyle{unsrt}
\pagestyle{empty}
\newcommand{\twofig}[2]
 {
   \begin{figure}[h]
     \begin{minipage}[t]{\minitwocolumn}
         \begin{center}   #1
         \end{center}
     \end{minipage}
         \hspace{\columnsep}
     \begin{minipage}[t]{\minitwocolumn}
         \begin{center} #2
         \end{center}
     \end{minipage}
   \end{figure}
 }
%%%%%%%%%%%%%%%%%%%%%%%%%%%%%%%%%
%\vspace*{\baselineskip}
\begin{center}
{\Large \bf 2018年度 構造力学I及び演習B(12月14日) 演習問題2 解答} \\
\end{center}
%%%%%%%%%%%%%%%%%%%%%%%%%%%%%%%%%%%%%%%%%%%%%%%%%%%%%%%%%%%%
%%%%%%%%%%%%%%%%%%%%%%%%%%%%%%%%%%%%%%%%%%%%%%%%%%%%
%%%%%%%%%%%%%%%%%%%%%%%%%%%%%%%%%%%%%%%%%%%%%%%%%%%%
\subsubsection*{問題(1)}
はじめに,図\ref{fig:fig1_1}-(a)に示すような, 単純支持された梁に単一の荷重が作用する問題
について考える.支点反力の正方向を同図にあるように定めると,はり全体の力とモーメントの
釣り合いから,
\begin{equation}
	R_A=\frac{b}{l}F, \ \ 
	R_C=\frac{a}{l}F, \ \ 
	H_A=0 
	\label{eqn:Rs_simple}
\end{equation}
となる.また,a-a'の位置で梁を切断して自由物体図を描くと,
図\ref{fig:fig1_1}-(b)のようになる.この図に基づき,
力とモーメントの釣り合い条件を立てると,
\begin{eqnarray}
	Q(x)-R_A=0  &\Rightarrow & Q(x)= R_A=\frac{b}{l}F
	\label{eqn:Qx}
	\\
	M(x)-R_Ax=0  &\Rightarrow & M(x)= R_Ax=\frac{b}{l}Fx
	\label{eqn:Mx}
\end{eqnarray}
と$0<x<a $における断面力が求められる.一方,b-b'で梁を切断した
場合,自由物体図は図\ref{fig:fig1_1}-(c)のようであり,釣り合い条件から
断面力が
\begin{eqnarray}
	Q(y)+R_C=0  &\Rightarrow & Q(y)= -R_C=-\frac{a}{l}F
	\label{eqn:Qy}
	\\
	M(y)-R_Cy=0  &\Rightarrow & M(y)= R_Cy=\frac{a}{l}Fy
	\label{eqn:My}
\end{eqnarray}
と求められる.以上の結果を断面力図として示すと,図\ref{fig:fig1_1}-(d)と(e)の
ようになる.

問題で与えられた系では,左から,大きさ$P,2P$および$3P$の集中荷重が加えられている.
これら3つの荷重のうち,ひとつだけが加えられた場合のせん断力を順に$Q_1, Q_2,Q_3$, 
曲げモーメントを$M_1, M_2, M_3$とすれば,問題で与えられた系のせん断力$Q$と
曲げモーメント$M$は
\begin{equation}
	Q=Q_1+Q_2+Q_3, \ \ 
	M=M_1+M_2+M_3
	\label{eqn:}
\end{equation}
と表すことができる.$Q_i,M_i(i=1,2,3)$は,式(\ref{eqn:Qx})-(\ref{eqn:My})の
特別な場合として得ることができる.すなわち,
$a=\frac{l}{4}, F=P$とすれば,$Q_1$と$M_1$が,$a=\frac{l}{2}, F=2P$とすれば
$Q_2$と$M_2$が,$a=\frac{3l}{4}, F=3P$とすれば,$Q_3$と$M_3$が
得られる.その結果を断面力図として示すと,図\ref{fig:fig1_2}のようになる.
支点反力についても,荷重$P$, $2P$および$3P$に起因して発生する支点反力の
合計が,問題で与えられた系における支点反力となることから,支点AとEにおける鉛直反力
$R_A$と$R_E$は次のようになる.
\begin{equation}
	R_A=\frac{3}{4}P+ P + \frac{3}{4}P=\frac{5}{2}P, \ \ 
	R_E=\frac{1}{4}P+ P + \frac{9}{4}P=\frac{7}{2}P
	\label{eqn:Rs_prb1}
\end{equation}
断面力分布と支点反力は,問題で与えられた系とその適当な部分構造に対して
釣り合い式を立てることでも求めることができる.その場合は,区間毎に
構造を切断して自由物体図を描き,力とモーメントの釣り合い条件式をたてればよい.
%--------------------
\begin{figure}[h]
	\begin{center}
	\includegraphics[width=0.50\linewidth]{fig1ans1.eps} 
	\end{center}
	\caption{単一の集中荷重を受ける単純支持された梁.} 
	\label{fig:fig1_1}
\end{figure}
%--------------------
\begin{figure}[h]
	\begin{center}
	\includegraphics[width=1.0\linewidth]{fig1ans2.eps} 
	\end{center}
	\caption{曲げモーメント図及びせん断力図(問題1).} 
	\label{fig:fig1_2}
\end{figure}
%%%%%%%%%%%%%%%%%%%%%%%%%%%%%%%%%%%%%%%%%%%%%%%%%%%%
%%%%%%%%%%%%%%%%%%%%%%%%%%%%%%%%%%%%%%%%%%%%%%%%%%%%
\subsubsection*{問題(2)}
支点反力の正方向を図\ref{fig:fig2_1}-(a)のように定め,鉛直力及びC点周りの
モーメントのつり合い式を立てれば
\begin{eqnarray}
	\left\{
	\begin{array}{ll}
		R_C - \frac{q_0l}{2}-\frac{q_0l}{4}= 0 \\
		M_C+\frac{q_0l}{2} \times \frac{l}{4}+
		\frac{q_0l}{4} \times \left( \frac{l}{2}+\frac{l}{6} \right)=0 
	\end{array}
	\right.
\end{eqnarray}
より,
\begin{equation}
	R_C = \frac{3}{4}q_0l, \ \ M_C=-\frac{7}{24}q_0l^2, \ \ H_C=0
\end{equation}
となる.
次に,図\ref{fig:fig2_1}-(a)のa-a'および b-b'で梁を仮想的に
切断し, その結果得られる部分構造について自由物体図を描けば, 
同図(b)と(c)のようになる.また,これらの自由物体図に現れる分布荷重を
それと等価な集中荷重に置き換えれば,図\ref{fig:fig2_1}-(b')と(c')のよう
になる.この結果を踏まえて釣り合い条件式を立てれば,断面力分布が
区間毎に以下のように求められる.
\begin{itemize}
\item
	区間AB$\left( 0<x<\frac{l}{2}\right)$:\\
	\begin{equation}
		Q(x) + \frac{q_0}{l}x^2 = 0
	\end{equation}
	\begin{equation}
		M(x) + \frac{q_0}{l}x^2 \times \frac{x}{3}= 0
	\end{equation}
	だから,
	\begin{equation}
		Q(x) =- \frac{q_0}{l}x^2
	\end{equation}
	\begin{equation}
		M(x) =-\frac{q_0}{3l}x^3.
	\end{equation}
\item
	区間BC$\left( 0<y<\frac{l}{2}\right)$:\\
	\begin{equation}
		Q(y) + \frac{q_0l}{4}+q_0y = 0
	\end{equation}
	\begin{equation}
		M(y)+ \frac{q_0l}{4}\times \left(y+\frac{l}{6}\right) + q_0y \times \frac{y}{2} =0
	\end{equation}
	より,
	\begin{equation}
		Q(y) =- \frac{q_0l}{4}-q_0y
	\end{equation}
	\begin{eqnarray}
		M(y)  
		&=& 
		-\frac{q_0l^2}{24} \left\{ 
			1 + 6\left( \frac{y}{l}\right)
			+ 12\left( \frac{y}{l}\right)^2
		\right\}	
		\nonumber 
		\\
		&=& 
		-\frac{q_0l^2}{24} \left\{ 
			1 + 3\left( \frac{y}{l/2}\right)
			+ 3\left( \frac{y}{l/2}\right)^2
		\right\}	
		\label{eqn:My_prb2}
	\end{eqnarray}
	となる.
\end{itemize}
これらの結果を断面力図として示すと,図\ref{fig:fig2_2}のようになる.
%--------------------
\begin{figure}[h]
	\begin{center}
	\includegraphics[width=0.6\linewidth]{fig2ans1.eps} 
	\end{center}
	\caption{問題(2)の解答のための自由物体図.} 
	\label{fig:fig2_1}
\end{figure}
%--------------------
\begin{figure}[h]
	\begin{center}
	\includegraphics[width=0.5\linewidth]{fig2ans2.eps} 
	\end{center}
	\caption{曲げモーメント図及びせん断力図(問題2).} 
	\label{fig:fig2_2}
\end{figure}
%%%%%%%%%%%%%%%%%%%%%%%%%%%%%%%%%%%%%%%%%%%%%%%%%%%%
%%%%%%%%%%%%%%%%%%%%%%%%%%%%%%%%%%%%%%%%%%%%%%%%%%%%
\end{document}
