\documentclass[10pt,a4j]{jarticle}
\usepackage{graphicx,wrapfig}
\setlength{\topmargin}{-1.5cm}
\setlength{\textwidth}{15.5cm}
\setlength{\textheight}{25.2cm}
\newlength{\minitwocolumn}
\setlength{\minitwocolumn}{0.5\textwidth}
\addtolength{\minitwocolumn}{-\columnsep}
%\addtolength{\baselineskip}{-0.1\baselineskip}
%
\def\Mmaru#1{{\ooalign{\hfil#1\/\hfil\crcr
\raise.167ex\hbox{\mathhexbox 20D}}}}
%
\begin{document}
\newcommand{\fat}[1]{\mbox{\boldmath $#1$}}
\newcommand{\D}{\partial}
\newcommand{\w}{\omega}
\newcommand{\ga}{\alpha}
\newcommand{\gb}{\beta}
\newcommand{\gx}{\xi}
\newcommand{\gz}{\zeta}
\newcommand{\vhat}[1]{\hat{\fat{#1}}}
\newcommand{\spc}{\vspace{0.7\baselineskip}}
\newcommand{\halfspc}{\vspace{0.3\baselineskip}}
\bibliographystyle{unsrt}
\pagestyle{empty}
\newcommand{\twofig}[2]
 {
   \begin{figure}[h]
     \begin{minipage}[t]{\minitwocolumn}
         \begin{center}   #1
         \end{center}
     \end{minipage}
         \hspace{\columnsep}
     \begin{minipage}[t]{\minitwocolumn}
         \begin{center} #2
         \end{center}
     \end{minipage}
   \end{figure}
 }
%%%%%%%%%%%%%%%%%%%%%%%%%%%%%%%%%
%\vspace*{\baselineskip}
\begin{center}
{\Large \bf 2019年度 構造力学I及び演習B 演習問題2 解答} \\
\end{center}
%%%%%%%%%%%%%%%%%%%%%%%%%%%%%%%%%%%%%%%%%%%%%%%%%%%%%%%%%%%%
%%%%%%%%%%%%%%%%%%%%%%%%%%%%%%%%%%%%%%%%%%%%%%%%%%%%
%%%%%%%%%%%%%%%%%%%%%%%%%%%%%%%%%%%%%%%%%%%%%%%%%%%%
\subsubsection*{問題(1)}
支点反力の正方向を図\ref{fig:fig1_1}-(a)にあるように定めると,はり全体の力とモーメントの
釣り合いから,
\begin{eqnarray}
	R_A &=& P-2P+3P=2P \\
	H_A &=&0 \\
	M_A &=& -P\times l + 2P\times \frac{2}{3}l -3P\times \frac{1}{3}l=-\frac{2}{3}Pl
	\label{eqn:Rs_simple}
\end{eqnarray}
となる.また,a-a'の位置で梁を切断して自由物体図を描くと,
図\ref{fig:fig1_1}-(b)のようになる.この図に基づき,力とモーメントの釣り合い条件を立てると,
\begin{eqnarray}
	Q(s_1) &=& P \\
	M(s_1) &=& -Ps_1, \ \ \left(0<s_1 < \frac{l}{3}\right)
	\label{eqn:eqlbl}
\end{eqnarray}
となる.ただし,$s_1$はは図\ref{fig:fig1_1}-(b)に示すように,点Dから切断位置までの距離を表す.
一方,b-b'で梁を切断した場合,自由物体図は図\ref{fig:fig1_1}-(c)のようであり,この場合,
釣り合い条件から断面力が
\begin{eqnarray}
	Q(s_2) &=& P-2P=-P \\ 
	M(s_2) &=& -P\left(s_2+\frac{l}{3}\right) +2Ps_2 = P \left( s_2-\frac{l}{3}\right)
	\label{eqn:eqlbl}
\end{eqnarray}
と求められる.最後に,c-c'の断面における断面力を,図\ref{fig:fig1_1}-(c)を参照して
求めると,
\begin{eqnarray}
	Q(s_3) &=&P-2P+3P=2P \\ 
	M(s_3) &=&-P\left(s_3+\frac{2l}{3}\right) +2P\left(s_3+\frac{l}{3}\right) -3P s_3
	=-2Ps_3 
	\label{eqn:eqlbl}
\end{eqnarray}
となる.以上の結果を断面力図として示すと,図\ref{fig:fig1_1}-(d)と(e)の
ようになる.
%--------------------
\begin{figure}[h]
	\begin{center}
	\includegraphics[width=0.50\linewidth]{fig1ans1.eps} 
	\end{center}
	\caption{解答に用いた自由物体図(a)$\sim$(d)と断面力図(e)と(f)(問題1).} 
	\label{fig:fig1_1}
\end{figure}
%--------------------
%%%%%%%%%%%%%%%%%%%%%%%%%%%%%%%%%%%%%%%%%%%%%%%%%%%%
%%%%%%%%%%%%%%%%%%%%%%%%%%%%%%%%%%%%%%%%%%%%%%%%%%%%
\subsubsection*{問題(2)}
支点反力の正方向を図\ref{fig:fig2_1}-(a)のように定め,鉛直力及びC点周りの
モーメントのつり合い式を立てれば
\begin{eqnarray}
	\left\{
	\begin{array}{ll}
		R_A+R_C-\left(
			\frac{1}{2}\times \frac{l}{2}\times q_0 + \frac{l}{2}\times q_0
			\right)
			=0
		\\
		R_A\times l 
		+\frac{q_0l}{4}\times \left(\frac{l}{2}+\frac{l}{2}\times \frac{1}{3}\right)
		+\frac{q_0l}{2}\times \frac{l}{4}
		=0
	\end{array}
	\right.
\end{eqnarray}
より,
\begin{equation}
	R_A=\frac{7}{24}q_0l, \ \ R_C=\frac{11}{24}q_0l, \ \ H_A=0
\end{equation}
となる.
次に,図\ref{fig:fig2_1}-(a)のa-a'および b-b'で梁を仮想的に
切断し, その結果得られる部分構造について自由物体図を描けば, 
同図(b)と(c)のようになる.また,これらの自由物体図に現れる分布荷重を,
それと等価な集中荷重に置き換えれば,図\ref{fig:fig2_1}-(b')と(c')のよう
になる.この結果を踏まえて釣り合い条件式を立てれば,断面力分布が
区間毎に以下のように求められる.
\begin{itemize}
\item
	区間AB$\left( 0<x<\frac{l}{2}\right)$:\\
	\begin{equation}
		Q(x)=R_A-\frac{x}{2}\times \frac{2q_0}{l}x 
	\end{equation}
	\begin{equation}
		M=R_Ax-
		\frac{x}{2}\times
		\frac{2q_0}{l}x \times\frac{x}{3}
	\end{equation}
	だから,
	\begin{equation}
		Q(x) =
		q_0l \left\{ 
			\frac{7}{24} -\left( \frac{x}{l} \right)^2
		\right\}
	\end{equation}
	\begin{equation}
		M(x) =
		\frac{q_0l^2}{3} \left\{ 
			\frac{7}{8}\left(\frac{x}{l}\right) - \left( \frac{x}{l} \right)^3
		\right\}
	\end{equation}
\item
	区間BC$\left( 0<y<\frac{l}{2}\right)$:\\
	\begin{equation}
		Q(s_)=q_0s-R_C
	\end{equation}
	\begin{equation}
		M(s)=R_Cs -\frac{1}{2}q_0s^2 
	\end{equation}
	より,
	\begin{equation}
		Q(s) =q_0l\left( \frac{s}{l}-\frac{11}{24}\right)
	\end{equation}
	\begin{equation}
		M(s) =\frac{q_0l^2}{2} 
		\left\{ \frac{11}{12}\left(\frac{s}{l}\right)-\left(\frac{s}{l}\right)^2 \right\}
	\end{equation}
	となる.
\end{itemize}
これらの結果を断面力図として示すと,図\ref{fig:fig2_2}のようになる.
%--------------------
\begin{figure}[h]
	\begin{center}
	\includegraphics[width=0.6\linewidth]{fig2ans1.eps} 
	\end{center}
	\caption{問題(2)の解答のための自由物体図.} 
	\label{fig:fig2_1}
\end{figure}
%--------------------
\begin{figure}[h]
	\begin{center}
	\includegraphics[width=0.5\linewidth]{fig2ans2.eps} 
	\end{center}
	\caption{曲げモーメント図及びせん断力図(問題2).} 
	\label{fig:fig2_2}
\end{figure}
%%%%%%%%%%%%%%%%%%%%%%%%%%%%%%%%%%%%%%%%%%%%%%%%%%%%
%%%%%%%%%%%%%%%%%%%%%%%%%%%%%%%%%%%%%%%%%%%%%%%%%%%%
\end{document}
