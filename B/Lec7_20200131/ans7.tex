\documentclass[10pt,a4j]{jarticle}
\usepackage{graphicx,wrapfig}
\setlength{\topmargin}{-1.5cm}
\setlength{\textwidth}{15.5cm}
\setlength{\textheight}{25.2cm}
\newlength{\minitwocolumn}
\setlength{\minitwocolumn}{0.5\textwidth}
\addtolength{\minitwocolumn}{-\columnsep}
%\addtolength{\baselineskip}{-0.1\baselineskip}
%
\def\Mmaru#1{{\ooalign{\hfil#1\/\hfil\crcr
\raise.167ex\hbox{\mathhexbox 20D}}}}
%
\begin{document}
\newcommand{\fat}[1]{\mbox{\boldmath $#1$}}
\newcommand{\D}{\partial}
\newcommand{\w}{\omega}
\newcommand{\ga}{\alpha}
\newcommand{\gb}{\beta}
\newcommand{\gx}{\xi}
\newcommand{\gz}{\zeta}
\newcommand{\vhat}[1]{\hat{\fat{#1}}}
\newcommand{\spc}{\vspace{0.7\baselineskip}}
\newcommand{\halfspc}{\vspace{0.3\baselineskip}}
\bibliographystyle{unsrt}
\pagestyle{empty}
\newcommand{\twofig}[2]
 {
   \begin{figure}[h]
     \begin{minipage}[t]{\minitwocolumn}
         \begin{center}   #1
         \end{center}
     \end{minipage}
         \hspace{\columnsep}
     \begin{minipage}[t]{\minitwocolumn}
         \begin{center} #2
         \end{center}
     \end{minipage}
   \end{figure}
 }
%%%%%%%%%%%%%%%%%%%%%%%%%%%%%%%%%
%\vspace*{\baselineskip}
\begin{center}
{\Large \bf 2018年度 構造力学I及び演習B(2月1日) 演習問題7 解答} \\
\end{center}
%%%%%%%%%%%%%%%%%%%%%%%%%%%%%%%%%%%%%%%%%%%%%%%%%%%%%%%%%%%%%%%%
\subsubsection*{問題1.}
材料$i(=1,2,3)$に発生するひずみは$\varepsilon_i$は,熱ひずみを$\varepsilon_i^T$
と機械的ひずみ$\varepsilon_i^M$の和として
\begin{equation}
	\varepsilon_i =
	\varepsilon_i^T 
	+
	\varepsilon_i^M, \ \ (i=1,2,3) 
\end{equation}
と表される.熱歪みは,線膨張係数$\alpha_i$と部材1と2の上昇温度$\Delta T$を用いて
\begin{equation}
	\varepsilon_i^T= \alpha_i \Delta T, \ \ (i=1,2), \ \ 
	\varepsilon_3^T=0
\end{equation}
と表される.一方,機械的ひずみは,ヤング率$E_i$と応力$\sigma_i$により,
\begin{equation}
	\varepsilon_i^M= \frac{\sigma_i}{E_i}, \ \ (i=1,2,3)
\end{equation}
で与えられる.材料$i$の部分に発生する軸力を$N_i$とすれば,
中央の剛体壁に作用する軸力は釣り合う必要があることから
\begin{equation}
	-N_1-N_2+N_3=0 
	\label{eqn:equib}
\end{equation}
である.一方,材料$i$に発生する伸び$\Delta l_i$は,
\begin{equation}
	\Delta l_i = \varepsilon_i l, \ \ (i=1,2,3) 
\end{equation}
で,適合条件:
\begin{equation}
	\Delta l_1=\Delta l_2 \ \ \Delta_1+\Delta l_3=0
	\label{eqn:compati}
\end{equation}
を満足しなければならない.そこで,式(\ref{eqn:equib}),(\ref{eqn:compati})から
$\sigma_1, \sigma_2, \sigma_3$を求めれば
\begin{eqnarray}
	\sigma_1 &=& 
	\frac{E_1}{E_1+E_2+E_3} \left\{E_2\alpha_2-(E_2+E_3)\alpha_1\right\}\Delta T
	\\
	\sigma_2 &=& 
	\frac{E_2}{E_1+E_2+E_3} \left\{E_1\alpha_1-(E_1+E_3)\alpha_2\right\}\Delta T
	\\
	\sigma_3 &=& 
	-\frac{E_3}{E_1+E_2+E_3} \left( \alpha_1 E_1+\alpha_2E_2\right)\Delta T
\end{eqnarray}
となる.以上より,
\begin{equation}
	\Delta l_3= \varepsilon_3 l = \varepsilon^M_3 l = \frac{\sigma_3}{E_3}l
	=
	-\frac{
		( \alpha_1 E_1+\alpha_2E_2)
	}{E_1+E_2+E_3}\Delta Tl
\end{equation}
と,部材efの伸びが求めらる.よって,$\Delta T>0$のとき,剛体壁2は右方向に
%\begin{equation}
$\frac{( \alpha_1 E_1+\alpha_2E_2)}{E_1+E_2+E_3} \Delta Tl$
%\end{equation}
だけ移動することが分かる.
\end{document}
%%%%%%%%%%%%%%%%%%%%%%%%%%%%%%%%%%%%%%%%%%%%%%%%%%%%%%%%%%%%%%%
