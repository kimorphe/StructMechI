\documentclass[10pt,a4j]{jarticle}
\usepackage{graphicx,wrapfig}
\setlength{\topmargin}{-1.5cm}
\setlength{\textwidth}{15.5cm}
\setlength{\textheight}{25.2cm}
\newlength{\minitwocolumn}
\setlength{\minitwocolumn}{0.5\textwidth}
\addtolength{\minitwocolumn}{-\columnsep}
%\addtolength{\baselineskip}{-0.1\baselineskip}
%
\def\Mmaru#1{{\ooalign{\hfil#1\/\hfil\crcr
\raise.167ex\hbox{\mathhexbox 20D}}}}
%
\begin{document}
\newcommand{\fat}[1]{\mbox{\boldmath $#1$}}
\newcommand{\D}{\partial}
\newcommand{\w}{\omega}
\newcommand{\ga}{\alpha}
\newcommand{\gb}{\beta}
\newcommand{\gx}{\xi}
\newcommand{\gz}{\zeta}
\newcommand{\vhat}[1]{\hat{\fat{#1}}}
\newcommand{\spc}{\vspace{0.7\baselineskip}}
\newcommand{\halfspc}{\vspace{0.3\baselineskip}}
\bibliographystyle{unsrt}
\pagestyle{empty}
\newcommand{\twofig}[2]
 {
   \begin{figure}[h]
     \begin{minipage}[t]{\minitwocolumn}
         \begin{center}   #1
         \end{center}
     \end{minipage}
         \hspace{\columnsep}
     \begin{minipage}[t]{\minitwocolumn}
         \begin{center} #2
         \end{center}
     \end{minipage}
   \end{figure}
 }
%%%%%%%%%%%%%%%%%%%%%%%%%%%%%%%%%
%\vspace*{\baselineskip}
\begin{center}
{\Large \bf 2018年度 構造力学I及び演習B(12月21日) 演習問題3 解答} \\
\end{center}
%%%%%%%%%%%%%%%%%%%%%%%%%%%%%%%%%%%%%%%%%%%%%%%%%%%%%%%%%%%%%%%%
\subsubsection*{問題}
\begin{enumerate}
\item
	$q(x)=q_0 \left\{ 1- H\left(x-l \right)\right\}$.
	ただし$H(x)$は単位ステップ関数を表す.
\item
$q(x)$を4回積分した結果は
\begin{equation}
	\int q(x) dx^4
	= 
	\frac{q_0}{24}
	\left( x^4-\left< x-l \right>^4 \right)
	=
	\frac{2}{3}q_0l^4 
	\left\{ 
		\left( \frac{x}{2l} \right )^4 
		- 
		\left< \left( \frac{x}{2l} \right)- \frac{1}{2} \right> ^4
	\right\}
\end{equation}
と書けるので,梁(b$_1$)のたわみ$v_1(x)$は,$C_1\sim C_4$を積分定数として
次のように表すことができる.
	\begin{equation}
	v_1(x)=\frac{2q_0l^4}{3EI}\left\{
		\left( \frac{x}{2l} \right )^4 
		- 
		\left< \left( \frac{x}{2l} \right)- \frac{1}{2} \right> ^4
		+
		C_1
		\left(\frac{x}{2l} \right)^3
		+
		C_2
		\left(\frac{x}{2l} \right)^2
		+
		C_3
		\left(\frac{x}{2l} \right)
		+
		C_4,
	\right\}
	\end{equation}
これを微分することで,曲げモーメントは
	\begin{equation}
		M_1(x)=-EIv_1''=-\frac{q_0l^2}{6}\left\{
			12\left( \frac{x}{2l} \right )^2
			- 
			12\left< \left( \frac{x}{2l} \right)- \frac{1}{2} \right> ^2
			+
			6C_1
			\left( \frac{x}{2l} \right)
			+
			2C_2
		\right\}
\end{equation}
となる. 積分定数$C_2$と$C_4$は,$x=0$における支持条件:$v(0)=0,\, M(0)=0$より
$C_2=C_4=0$と決まる.一方,$C_1,C_3$は,$x=2l$における支持条件
$v(2l)=0,\, M(2l)=0$より,$C_1=-\frac{3}{2},\, C_3=\frac{9}{16}$
となる.以上より,たわみは
	\begin{equation}
	v_1(x)=\frac{2q_0l^4}{3EI}\left\{
		\left( \frac{x}{2l} \right )^4 
		- 
		\left< \left( \frac{x}{2l} \right)- \frac{1}{2} \right> ^4
		-
		\frac{3}{2}
		\left(\frac{x}{2l} \right)^3
		+
		\frac{9}{16}
		\left(\frac{x}{2l} \right)
	\right\}
	\label{eqn:vx1}
	\end{equation}
となる.これより,点Bのたわみは
\begin{equation}
	v_B=v\left( l \right) 
	= \frac{5}{48}\frac{q_0l^4}{EI}.
	\label{eqn:vc_q}
\end{equation}
と求められる.
%%%%%%%%%%%%%%
\item
問題で与えられた梁(b)に作用する鉛直方向の支点反力を図\ref{fig:fig1}-(b)に示すようにとる.
この梁を同図(b$_1$)と(b$_2$)のような2つの系の重ね合わせとして表現する.
このとき,梁(b$_1)$のB点におけるたわみは式(\ref{eqn:vc_q})で与えられる.
一方,梁(b$_2$)のたわみ分布$v_2$は,たわみの方程式を単純支持条件の元で解くことにより,
\begin{equation}
	v_2(x)=\frac{4(-R_B)l^3}{3EI}\left\{
		\left< \frac{x}{2l}-\frac{1}{2}\right>^3
		-
		\frac{1}{2}
		\left(\frac{x}{2l} \right)^3
		+
		\frac{3}{8}
		\left(\frac{x}{2l} \right)
	\right\}
	\label{eqn:vx2}
\end{equation}
となることから,点Bにおけるたわみは
\begin{equation}
	v_2\left( l \right) 
	= -\frac{R_Bl^3}{6EI}.
	\label{eqn:vc_P}
\end{equation}
である.$v_B$と$v_2$は相殺すべきものであることから,
\begin{equation}
	v_B+v_2(l)=0 \ \ \Rightarrow \ \ R_B=\frac{5}{8}q_0l
\end{equation}
より,点Bにおける鉛直反力が求められる.これを踏まえ,
梁(b)全体の力とモーメントの釣り合い条件を使えば,
\begin{equation}
	R_A=\frac{7}{16}q_0l, \ \ R_C=-\frac{1}{16}q_0l
\end{equation}
と他の反力が決定される.
\item
梁(b$_1$)の曲げモーメント$M_1(x)$は,式(\ref{eqn:vx1})より
\begin{equation}
	M_1(x)=-EIv_1''(x)=
	-\frac{q_0l^2}{2}		
	\left\{
		4 \left( \frac{x}{2l} \right)^2
		-
		4
		\left< \left(\frac{x}{2l}\right)-\frac{1}{2} \right>^2
		-
		3
		\left(\frac{x}{l} \right)
	\right\}
	\label{eqn:Mx}
\end{equation}
梁(b$_2$)の曲げモーメント$M_2(x)$は,式(\ref{eqn:vx2})より
\begin{equation}
	M_2(x)=-EIv_2''(x)=
	R_Bl		
	\left\{
		2\left< \frac{x}{2l} -\frac{1}{2}\right>
		-
		\frac{x}{2l}
	\right\}
	\label{eqn:Mx2}
\end{equation}
である.梁(b)の曲げモーメントは$M(x)=M_1(x)+M_2(x)$で与えられ,これを曲げモーメント図として
表すと,図\ref{fig:fig2}のようになる.
なお,支点反力が求められた時点で,断面力は全て釣り合い条件から決定することができるので,
たわみを微分する以外にも,区間毎に釣り合い式を立てて断面力を求めてもよい.
%--------------------
\begin{figure}[h]
	\begin{center}
	\includegraphics[width=0.6\linewidth]{fig1ans.eps} 
	\end{center}
	\caption{2つの系$(b_1)$と$(b_2)$による梁(b)の表現.}
	\label{fig:fig1}
\end{figure}
%--------------------
%--------------------
\begin{figure}[h]
	\begin{center}
	\includegraphics[width=0.6\linewidth]{fig2ans.eps} 
	\end{center}
	\caption{曲げモーメント図.}
	\label{fig:fig2}
\end{figure}
%--------------------
%%%%%%%%%%%%%%%%%%%%%%%%%%%%%%
\end{enumerate}
\end{document}
x
