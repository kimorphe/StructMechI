\documentclass[10pt,a4j]{jarticle}
\usepackage{graphicx,wrapfig}
\setlength{\topmargin}{-1.5cm}
\setlength{\textwidth}{15.5cm}
\setlength{\textheight}{25.2cm}
\newlength{\minitwocolumn}
\setlength{\minitwocolumn}{0.5\textwidth}
\addtolength{\minitwocolumn}{-\columnsep}
%\addtolength{\baselineskip}{-0.1\baselineskip}
%
\def\Mmaru#1{{\ooalign{\hfil#1\/\hfil\crcr
\raise.167ex\hbox{\mathhexbox 20D}}}}
%
\begin{document}
\newcommand{\fat}[1]{\mbox{\boldmath $#1$}}
\newcommand{\D}{\partial}
\newcommand{\w}{\omega}
\newcommand{\ga}{\alpha}
\newcommand{\gb}{\beta}
\newcommand{\gx}{\xi}
\newcommand{\gz}{\zeta}
\newcommand{\vhat}[1]{\hat{\fat{#1}}}
\newcommand{\spc}{\vspace{0.7\baselineskip}}
\newcommand{\halfspc}{\vspace{0.3\baselineskip}}
\bibliographystyle{unsrt}
\pagestyle{empty}
\newcommand{\twofig}[2]
 {
   \begin{figure}[h]
     \begin{minipage}[t]{\minitwocolumn}
         \begin{center}   #1
         \end{center}
     \end{minipage}
         \hspace{\columnsep}
     \begin{minipage}[t]{\minitwocolumn}
         \begin{center} #2
         \end{center}
     \end{minipage}
   \end{figure}
 }
%%%%%%%%%%%%%%%%%%%%%%%%%%%%%%%%%
%\vspace*{\baselineskip}
\begin{center}
{\Large \bf 2019年度 構造力学I及び演習B 演習問題1 解答} \\
\end{center}
%%%%%%%%%%%%%%%%%%%%%%%%%%%%%%%%%%%%%%%%%%%%%%%%%%%%%%%%%%%%%%%%
問題1, 2とも$EI$が一定のため,
微分方程式$EIv^{(4)}=q(x)$を,与えられた外力$q(x)$と
支持条件の元に解けばよい($v^{(4)}$は$v$の4階微分を表す).
曲げモーメントとせん断力は, それぞれたわみの2階および3階の導関数から
求められる.また,支点反力は支点部における断面力の値から得ることができる.
\subsubsection*{問題1. }
\begin{enumerate}
\item
	$q(x)=q_0H\left(\frac{l}{2}-x\right)=q_0H\left(\frac{1}{2}-\frac{x}{l}\right)$より,
\begin{equation}
	v(x)=\frac{q_0l^4}{24EI}\left\{
			\left< \frac{1}{2} -\frac{x}{l} \right>^4
			+
			C_1\left(\frac{x}{l}\right)^3
			+
			C_2\left(\frac{x}{l}\right)^2
			+
			C_3\left(\frac{x}{l}\right)
			+
			C_4
		\right\}
		\label{eqn:gen_sol1}
\end{equation}
と表すことができる.式(\ref{eqn:gen_sol1})に含まれる積分定数は,支点AとBにおける
支持条件から,以下のように決まる.
支点Bにおける支持条件:
\begin{equation}
	M(l)=-EIv''(l) =0,
	\ \  
	Q(l)=-EIv'''(l) =0,
	\label{eqn:BC1_B}
\end{equation}
より,$C_1=C_2=0$となる.一方,支点Aにおける支持条件:
\begin{equation}
	v(0) =0, \, v'(0)=0.
	\label{eqn:BC1_A}
\end{equation}
より, $C_3=\frac{1}{2},\, C_4=-\frac{1}{16}$.

よってたわみ$v$は
\begin{equation}
	v(x)=\frac{q_0l^4}{24EI}\left\{
			\left< \frac{1}{2} -\frac{x}{l} \right>^4
			+
			\frac{1}{2}	
			\left(\frac{x}{l}\right)
			-
			\frac{1}{16}	
			\right\}
	\label{eqn:v1}
\end{equation}
と求められる.
\item
曲げモーメント$M(x)$とせん断力$Q(x)$は,式(\ref{eqn:v1})より
\begin{eqnarray}
	M(x) &=& 
		-\frac{q_0l^2}{2}
				\left< \frac{1}{2} -\frac{x}{l} \right>^2
	\label{eqn:M1}
	\\
	Q(x) &=&
		q_0l \left<\frac{1}{2} -\frac{x}{l}\right>
	\label{eqn:Q1}
\end{eqnarray}
となる.これらを断面力図として示せば図\ref{fig:fig1}の(b)および(c)のようになる.
\begin{figure}[h]
	\begin{center}
	\includegraphics[width=0.65\linewidth]{fig1ans.eps} 
	\end{center}
	\vspace{-5mm}
	\caption{(a)支点反力の正方向.
	(b)せん断力図, 及び(c) 曲げモーメント図(問題1).}
	\label{fig:fig1}
\end{figure}
\item
支点反力の向きを図\ref{fig:fig1}-(a)に示すようにとれば,
\begin{equation}
	R_A = Q(0) = \frac{q_0l}{2}, \ \ 
	M_A = M(0) = -\frac{1}{8}q_0l^2
\end{equation}
となる.
\end{enumerate}
%%%%%%%%%%%%%%%%%%%%%%%%%%%%%%%%%%%%%%%%%%%%%%%%%%%%%%%%%%%%%%%%%%%
%%%%%%%%%%%%%%%%%%%%%%%%%%%%%%%%%%%%%%%%%%%%%%%%%%%%%%%%%%%%%%%%%%%
\subsubsection*{問題2. }
\begin{enumerate}
\item
外力$q(x)$はディラクのデルタ関数を用いて
\begin{equation}
	q(x)=P\delta \left(x-\frac{2l}{3} \right)
	\label{eqn:qx_dlt}
\end{equation}
と表される.よって,たわみ$v(x)$は,$C_1\sim C_4$を積分定数として
\begin{equation}
	v(x)=\frac{Pl^3}{6EI}\left\{
			\left< \frac{x}{l} -\frac{2}{3}\right>^3
			+
			C_1\left(\frac{x}{l}\right)^3
			+
			C_2\left(\frac{x}{l}\right)^2
			+
			C_3\left(\frac{x}{l}\right)
			+
			C_4
		\right\}
		\label{eqn:gen_sol2}
\end{equation}
	と書くことができる.式(\ref{eqn:gen_sol2})より,曲げモーメントは
\begin{equation}
	M(x)=-EIv''=-\frac{Pl}{6}\left\{
			6\left< \frac{x}{l} -\frac{1}{3}\right>
			+
			6C_1\left(\frac{x}{l}\right)
			+
			2C_2
		\right\}
\end{equation}
だから,点Aにおける支持条件:
\begin{equation}
	v(0)=0,\, v'(0)=0
\end{equation}
より$C_3=C_4=0$が, 点Cにおける支持条件
\begin{equation}
	v(l)=0,\, M(l)=-EIv''(l)=0
\end{equation}
より,$C_1=-\frac{13}{27}, \, C_2=\frac{4}{9}$ときまる.以上より,
たわみは
\begin{equation}
	v(x)=\frac{Pl^3}{6EI}\left\{
			\left< \frac{x}{l} -\frac{2}{3}\right>^3
			-
			\frac{13}{27}\left(\frac{x}{l}\right)^3
			+
			\frac{4}{9}\left( \frac{x}{l} \right)^2
			\right\}
	\label{eqn:v2}
\end{equation}
となる.
\item
式(\ref{eqn:v2})より,曲げモーメント$M$とせん断力$Q$は以下のようになる.
\begin{eqnarray}
	M(x) &=& 
		-\frac{Pl}{3}
			\left\{
				3\left< \frac{x}{l} - \frac{2}{3}\right> 
				-\frac{13}{9}\left(\frac{x}{l}\right)
				+\frac{4}{9}
			\right\}
	\label{eqn:M2}
	\\
	Q(x) &=&
		-\frac{P}{3}
		\left\{
				3H\left(\frac{x}{l} -\frac{2}{3}\right)
				-
				\frac{13}{9}
		\right\}
	\label{eqn:Q2}
\end{eqnarray}
これらを断面力図として示せば図\ref{fig:fig2}-(b)および(c)のようになる.
\begin{figure}[h]
	\begin{center}
	\includegraphics[width=0.65\linewidth]{fig2ans.eps} 
	\end{center}
	\vspace{-5mm}
	\caption{(a)支点反力の正方向.
	(b)せん断力図, 及び(c) 曲げモーメント図(問題2).}
	\label{fig:fig2}
\end{figure}
\item
支点反力の向きを図\ref{fig:fig2}-(a)のようにとれば,
\begin{equation}
	R_A = Q(0) =  \frac{13}{27}P, \ \  
	M_A = M(0) =  -\frac{4}{27}Pl, \ \  
	R_C = -Q(l) =  \frac{14}{27}P
\end{equation}
となる.
\end{enumerate}
\end{document}
