\documentclass[10pt,a4j]{jarticle}
\usepackage{graphicx,wrapfig}
\setlength{\topmargin}{-1.5cm}
\setlength{\textwidth}{15.5cm}
\setlength{\textheight}{25.2cm}
\newlength{\minitwocolumn}
\setlength{\minitwocolumn}{0.5\textwidth}
\addtolength{\minitwocolumn}{-\columnsep}
%\addtolength{\baselineskip}{-0.1\baselineskip}
%
\def\Mmaru#1{{\ooalign{\hfil#1\/\hfil\crcr
\raise.167ex\hbox{\mathhexbox 20D}}}}
%
\begin{document}
\newcommand{\fat}[1]{\mbox{\boldmath $#1$}}
\newcommand{\D}{\partial}
\newcommand{\w}{\omega}
\newcommand{\ga}{\alpha}
\newcommand{\gb}{\beta}
\newcommand{\gx}{\xi}
\newcommand{\gz}{\zeta}
\newcommand{\vhat}[1]{\hat{\fat{#1}}}
\newcommand{\spc}{\vspace{0.7\baselineskip}}
\newcommand{\halfspc}{\vspace{0.3\baselineskip}}
\bibliographystyle{unsrt}
\pagestyle{empty}
\newcommand{\twofig}[2]
 {
   \begin{figure}[h]
     \begin{minipage}[t]{\minitwocolumn}
         \begin{center}   #1
         \end{center}
     \end{minipage}
         \hspace{\columnsep}
     \begin{minipage}[t]{\minitwocolumn}
         \begin{center} #2
         \end{center}
     \end{minipage}
   \end{figure}
 }
%%%%%%%%%%%%%%%%%%%%%%%%%%%%%%%%%
%\vspace*{\baselineskip}
\begin{center}
{\Large \bf 2018年度 構造力学I及び演習B (1月16日)演習問題5} \\
\end{center}
%%%%%%%%%%%%%%%%%%%%%%%%%%%%%%%%%%%%%%%%%%%%%%%%%%%%%%%%%%%%%%%%
\subsubsection*{問題1.}
図\ref{fig:fig1}に示す骨組み構造について, 軸力図,せん断力図および
曲げモーメント図を描け.なお, 集中荷重$P$は部材ACに垂直に加えられている
ものとする.また,解答には, 計算に用いた座標系と断面力の正方向を明記すること.
%--------------------
\begin{figure}[h]
	\begin{center}
	\includegraphics[width=0.6\linewidth]{fig1.eps} 
	\end{center}
	\caption{部材ABに垂直な等分布荷重を受ける骨組み構造.}
	\label{fig:fig1}
\end{figure}
\subsubsection*{問題2.}
図\ref{fig:fig2}に示すような, 水平力$P$をうけるトラス構造の, 
部材1$\sim$6に発生する軸力を求めよ.なお,トラスの部材長は斜材の
長さが$\sqrt{2}l$, それ以外の部材は$l$であるとする.
\begin{figure}[h]
	\begin{center}
	\includegraphics[width=0.5\linewidth]{fig2.eps} 
	\end{center}
	\caption{水平力を受けるトラス構造.} 
	\label{fig:fig2}
\end{figure}
\end{document}
