\documentclass[10pt,a4j]{jarticle}
\usepackage{graphicx,wrapfig}
\setlength{\topmargin}{-1.5cm}
\setlength{\textwidth}{15.5cm}
\setlength{\textheight}{25.2cm}
\newlength{\minitwocolumn}
\setlength{\minitwocolumn}{0.5\textwidth}
\addtolength{\minitwocolumn}{-\columnsep}
%\addtolength{\baselineskip}{-0.1\baselineskip}
%
\def\Mmaru#1{{\ooalign{\hfil#1\/\hfil\crcr
\raise.167ex\hbox{\mathhexbox 20D}}}}
%
\begin{document}
\newcommand{\fat}[1]{\mbox{\boldmath $#1$}}
\newcommand{\D}{\partial}
\newcommand{\w}{\omega}
\newcommand{\ga}{\alpha}
\newcommand{\gb}{\beta}
\newcommand{\gx}{\xi}
\newcommand{\gz}{\zeta}
\newcommand{\vhat}[1]{\hat{\fat{#1}}}
\newcommand{\spc}{\vspace{0.7\baselineskip}}
\newcommand{\halfspc}{\vspace{0.3\baselineskip}}
\bibliographystyle{unsrt}
\pagestyle{empty}
\newcommand{\twofig}[2]
 {
   \begin{figure}[h]
     \begin{minipage}[t]{\minitwocolumn}
         \begin{center}   #1
         \end{center}
     \end{minipage}
         \hspace{\columnsep}
     \begin{minipage}[t]{\minitwocolumn}
         \begin{center} #2
         \end{center}
     \end{minipage}
   \end{figure}
 }
%%%%%%%%%%%%%%%%%%%%%%%%%%%%%%%%%
%\vspace*{\baselineskip}
\begin{center}
{\Large \bf 2018年度 構造力学I及び演習B (1月16日) 演習問題5 解答} \\
\end{center}
%%%%%%%%%%%%%%%%%%%%%%%%%%%%%%%%%%%%%%%%%%%%%%%%%%%%%%%%%%%%%%%%
%%%%%%%%%%%%%%%%%%%%%%%%%%%%%%%%%%%%%%%%%%%%%%%%%%%%%%%%%%%%%%%%%%%%%%%%%%%%%%%%%%%%%%%%%%
\subsubsection*{問題1.}
\begin{itemize}
\item
	{\rm 反力:}\\
	骨組み構造に作用する支点反力を図\ref{fig:fig1}-(a)のように表す.これらは,
	力とモーメントの釣り合い式:
	\begin{eqnarray*}
		水平 &:& \frac{\sqrt{3}q_0l}{2}-H_D=0 \\
		鉛直 &:& \frac{q_0l}{2}+R_A+R_D=0 \\
		モーメント(基準点D) &:& R_A \times \frac{3l}{2}+\frac{\sqrt{3}}{2}q_0l\times \frac{\sqrt{3}l}{4}
		-\frac{q_0l}{2}\times \frac{5l}{4}=0 \\
	\end{eqnarray*}
	より,
	\[
		H_D=\frac{\sqrt{3}}{2}q_0l, \ \ 
		R_A=\frac{1}{6}q_0l, \ \
		R_D=\frac{1}{3}q_0l
	\]	
	と求められる.

	以上を元に断面力計算を行うにあたり,部材AB,BC,CDをそれぞれ部材1,2,3と呼ぶ.
	また, 部材$i,\,(i=1,2,3)$の軸力,せん断力,曲げモーメントをそれぞれ$N_i, \, Q_i,\, M_i$と表す.
	これらの断面力は,図\ref{fig:fig1}-(a)に示した,a-a', b-b', c-c'において
	構造を仮想的に切断したときの自由物体図について釣り合い式を立てることによって決定できる.
\item
	{\rm 部材1:} \\
	図\ref{fig:fig1}-(b)の自由物体図について, 部材軸および部材軸直角方向の力と, 
	モーメントのつり合い式を立てればよい.その結果は,
	\begin{eqnarray}
		N_1(x_1) & =& -\frac{\sqrt{3}}{2}R_A=-\frac{\sqrt{3}}{12}q_0l \\
		Q_1(x_1) & =& \frac{1}{2}R_A-q_0x_1=q_0l\left(\frac{1}{12}-\frac{x_1}{l}\right)\\
		M_1(x_1) & =& -\frac{q_0x_1^2}{2}-\frac{R_A}{2}x_1=-\frac{q_0l^2}{2}
		\left(\frac{x_1}{l}\right)
		\left\{
		\left(\frac{x_1}{l}\right)
		-\frac{1}{6}
		\right\}
	\end{eqnarray}
\item
	{\rm 部材2:} \\
	部材2の断面力計算には図\ref{fig:fig1}-(c)の自由物体図を用いることができる.この自由物体図に対する
	釣り合い条件から, 断面力は
	\begin{eqnarray}
		N_2(s_2) 
			&=& 
			-H_D=-\frac{\sqrt{3}}{2}q_0l\\
		Q_2(s_2) 
			&=&
			-R_D=
			-\frac{1}{3}q_0l\\
		M_2(s_2) 
			&=&
			=R_Ds_2-H_D\times \frac{\sqrt{3}}{2}l
			= q_0l^2 \left( \frac{s_2}{3l}-\frac{3}{4}\right)
	\end{eqnarray}
	と求められる.
\item
	{\rm 部材3:} \\
	部材3の断面力計算には図\ref{fig:fig1}-(c)の自由物体図を用いることができる.この自由物体図に対する
	釣り合い条件から, 断面力は次のように求められる.
	\begin{eqnarray}
		N_3(s_3) 
			&=& 
			-R_D=-\frac{1}{3}q_0l\\
		Q_3(s_3) 
			&=&
			H_D=
			\frac{\sqrt{3}}{2}q_0l\\
		M_3(s_3) 
			&=&
			-H_Ds_3
			=-\frac{\sqrt{3}}{2}q_0ls_3
	\end{eqnarray}
\end{itemize}
以上の結果を断面力図として図示すれば,図\ref{fig:fig2}のようになる.
\begin{figure}[h]
	\begin{center}
	\includegraphics[width=0.9\linewidth]{fig1ans.eps} 
	\end{center}
	\caption{骨組み構造ABCDの断面力の計算に用いた自由物体図.} 
	\label{fig:fig1}
\end{figure}
\begin{figure}[h]
	\begin{center}
	\includegraphics[width=0.5\linewidth]{fig2_2ans.eps} 
	\end{center}
	\caption{骨組み構造ABCDの断面力図.} 
	\label{fig:fig2}
\end{figure}
%%%%%%%%%%%%%%%%%%%%%%%%%%%%%%%%%%%%%%%%%%%%%%%%%%%%%%%%%%%%%%%%%%%%%%%%%%%%%%%%%%%%%%%%%%
\subsubsection*{問題2.}
支点反力の正方向を図\ref{fig:fig3}-(a)のようにとれば,
トラス全体のつり合い条件:
\begin{eqnarray}
	&& P-H_A=0  \\
	&& R_A+R_D=0 \\  
	&& -R_D\times 3l + P \times 3l=0 
\end{eqnarray}
より
\begin{equation}
	H_A=P, \ \ 
	R_A=-P, \ \ 
	R_D=P
\end{equation}
となる.

次に,a-a'でトラス構造を仮想的に切断し,上側の部分構造について自由物体図を描くと,
図\ref{fig:fig3}-(b)のようになる.これに対して,点Gに関するモーメントのつり合い式を
立てれば,$N_5$が
\begin{equation}
	P\times l +N_5\times l=0
	\; \Rightarrow \;
	N_5=-P
\end{equation}
と求まる.次に,点Iに関するモーメントのつり合いより,
\begin{equation}
	N_5\times l -N_4\times \frac{l}{\sqrt{2}}=0
	\; \Rightarrow \;
	N_4=-\sqrt{2}P
\end{equation}
と$N_4$が求められる.さらに,点Jに関するモーメントの釣り合い式からは
\begin{equation}
	N_3\times l +\sqrt{2}l\times N_4 =0
	\; \Rightarrow \;
	N_3=2P
\end{equation}
が得られる.

次に,図\ref{fig:fig3}-(c)を参照して節点Gに関する水平力の釣り合い条件を用いれば,
\begin{equation}
	N_6=-N_4\times \frac{1}{\sqrt{2}}=P, \ \ 
\end{equation}
と$N_6$が求められる.

最後に,c-c'の位置で構造を切断して得られる部分構造について自由物体図を描き(図\ref{fig:fig3}-(d)),$N_1$と$N_2$を求める.
このとき,鉛直方向の力の釣り合い式より,
\begin{equation}
	N_3-N_1\frac{1}{\sqrt{2}}+R_A=0 \; \Rightarrow \; N_1=\sqrt{2}P
\end{equation}
が,点Bに関するモーメントの釣り合い式より
\begin{equation}
	N_2l+\frac{N_1}{\sqrt{2}}l +R_Al=0 \; \Rightarrow \; N_2=0
\end{equation}
と$N_1$と$N_2$を得ることができる.
%--------------------
\begin{figure}[h]
	\begin{center}
	\includegraphics[width=1.0\linewidth]{fig3ans.eps} 
	\end{center}
	\caption{トラス構造の軸力計算に用いた部分構造と自由物体図.} 
	\label{fig:fig3}
\end{figure}
\end{document}
