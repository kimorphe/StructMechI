\documentclass[10pt,a4j]{jarticle}
\usepackage{graphicx,wrapfig}
\setlength{\topmargin}{-1.5cm}
\setlength{\textwidth}{15.5cm}
\setlength{\textheight}{25.2cm}
\newlength{\minitwocolumn}
\setlength{\minitwocolumn}{0.5\textwidth}
\addtolength{\minitwocolumn}{-\columnsep}
%\addtolength{\baselineskip}{-0.1\baselineskip}
%
\def\Mmaru#1{{\ooalign{\hfil#1\/\hfil\crcr
\raise.167ex\hbox{\mathhexbox 20D}}}}
%
\begin{document}
\newcommand{\fat}[1]{\mbox{\boldmath $#1$}}
\newcommand{\D}{\partial}
\newcommand{\w}{\omega}
\newcommand{\ga}{\alpha}
\newcommand{\gb}{\beta}
\newcommand{\gx}{\xi}
\newcommand{\gz}{\zeta}
\newcommand{\vhat}[1]{\hat{\fat{#1}}}
\newcommand{\spc}{\vspace{0.7\baselineskip}}
\newcommand{\halfspc}{\vspace{0.3\baselineskip}}
\bibliographystyle{unsrt}
\pagestyle{empty}
\newcommand{\twofig}[2]
 {
   \begin{figure}[h]
     \begin{minipage}[t]{\minitwocolumn}
         \begin{center}   #1
         \end{center}
     \end{minipage}
         \hspace{\columnsep}
     \begin{minipage}[t]{\minitwocolumn}
         \begin{center} #2
         \end{center}
     \end{minipage}
   \end{figure}
 }
%%%%%%%%%%%%%%%%%%%%%%%%%%%%%%%%%
%\vspace*{\baselineskip}
\begin{center}
{\Large \bf 2018年度 構造力学I及び演習B (1月11日)演習問題4 解答} \\
\end{center}
%%%%%%%%%%%%%%%%%%%%%%%%%%%%%%%%%%%%%%%%%%%%%%%%%%%%%%%%%%%%%%%%
\subsubsection*{問題1}
断面$S$の面積$\left| S \right|$は, $k=\frac{\pi}{b}$として,
\begin{eqnarray}
	\left| S\right| &= & \int _{x=0}^b\int _{y=0}^{h\sin kx} dydx   
	\nonumber
	\\
	&=& \int_0^b h\sin(kx) dx 
	\nonumber
	\\
	&=& 
	\frac{2bh}{\pi}
\end{eqnarray}
となる.一方,$x$軸に関する断面1次モーメント$G$と断面2次モーメント$I_x$は,
それぞれ以下のように求められる.
\begin{eqnarray}
	G &=& 
	\int _{x=0}^b\int _{y=0}^{h\sin kx} ydydx   
	\nonumber \\
	&=&
	\int _{x=0}^b\left[ \frac{y^2}{2} \right]_0^{h\sin(kx)}dx   
	\nonumber \\
	&=& \frac{h^2}{2} \int_0^b \sin^2 (kx)dx
	\nonumber \\
	&=& \frac{h^2}{2k} \int_0^\pi \frac{1-\cos2\theta }{2}d\theta, \ \ (\theta=kx) 
	\nonumber \\
	&=& \frac{bh^2}{4}
\end{eqnarray}
\begin{eqnarray}
	I_x &= & 
	\int _{x=0}^b\int _{y=0}^{h\sin kx} y^2dydx   
	\nonumber \\
	&=& \frac{h^3}{3} \int_0^b \sin^3 (kx) dx 
	\nonumber \\
	&=& \frac{h^3}{3k} \int_0^\pi \sin^3 \theta d\theta , \ \ (\theta=kx)
	\nonumber \\
	&=& \frac{h^3}{3k} \int_0^\pi \left( \frac{3}{4}\sin \theta -\frac{1}{4}\sin3\theta \right)d\theta 
	\nonumber \\
	&=& 
	\frac{4}{9} \frac{bh^3}{\pi}
	\\
\end{eqnarray}
以上より,中立軸位置$\bar Y$は
\begin{equation}
	\bar Y = \frac{G}{\left|S\right|}=\frac{\pi h}{8}
\end{equation}
となり,中立軸まわりの断面2次モーメント$I$は
\begin{equation}
	I = I_x -S\bar Y^2 =  \left(\frac{4}{9\pi}-\frac{\pi}{32} \right) bh^3
\end{equation}
と得ることができる.
%%%%%%%%%%%%%%%%%%%%%%%%%%%%%%%%%%%%%%%%%%%%%%%%%%%%%%%%%%%%%%%%%%%%%%%%%%%%
%%%%%%%%%%%%%%%%%%%%%%%%%%%%%%%%%%%%%%%%%%%%%%%%%%%%%%%%%%%%%%%%%%%%%%%%%%%%
\subsubsection*{問題2}
図\ref{fig:fig1}-(a)に示すような,幅$b$,高さ$h$の直角三角形の,$X$軸に関する
断面2次モーメント$I(b,h)$は
\begin{equation}
	I(b,h)=\int_{Y=0}^h\int_{X=0}^{b\left(1-\frac{Y}{h}\right)}Y^2dXdY= \frac{bh^3}{12}
	\label{eqn:Ibh}
\end{equation}
である.一方,図\ref{fig:fig1}-(b)のような正方形は,幅$a/\sqrt{2}$,高さ$a/\sqrt{2}$
の直角三角形4つに分割することができるので,その$X$軸に関する断面2次モーメント$I_b$は
\begin{equation}
	I_b
	=
	4 \left. I(b,h) \right|
	_{b=h=\frac{a}{\sqrt{2}}}
	=\frac{a^4}{12}
\end{equation}
である.
ここで,問題で与えられた断面を図\ref{fig:fig1}-(c)のような,3つの正方形部分断面$S_1, S_2$と$S_3$に
分割する.部分断面$S_i,\,(i=1,2,3)$の面積を$\left|S_i\right|$, 中立軸位置を$\bar Y_i$, 中立軸回りの
断面2次モーメントを$I_i$とする.これらは,
\begin{equation}
	\left| S_i\right|=a^2, \ \ I_i=\frac{a^4}{12} \ \ (i=1,2,3) 
\end{equation}
\begin{equation}
	\bar Y_1=\bar Y_2=0, \ \ \bar Y_3=\frac{a}{\sqrt{2}}
\end{equation}
である.従って, 断面$S=S_1\cup S_2 \cup S_3$の中立軸位置は
\begin{equation}
	\bar Y=\sum _{i=1}^3 \frac{\left|S_i\right|}{\left|S\right|}\bar Y_i =\frac{a}{3\sqrt{2}}
\end{equation}
となる.以上の結果を利用して断面$S$の中立軸$\bar Y$に関する断面2次モーメント$I(S)$を求めると
\begin{equation}
	I(S)=\sum_{i=1}^3 \left \{ I_i + \left|S_i\right|\left(\bar Y_i-\bar Y\right)^2 \right\} =\frac{7}{12}a^4
\end{equation}
となる.
%%%%%%%%%%%%%%%%%%%%%%%%%%%%%%%%%%%%%%%%%%%%%%%%%%%%%%%%%%%%%%%%%%%%%%%%%%%%
%%%%%%%%%%%%%%%%%%%%%%%%%%%%%%%%%%%%%%%%%%%%%%%%%%%%%%%%%%%%%%%%%%%%%%%%%%%%
%--------------------
\begin{figure}[h]
	\begin{center}
	\includegraphics[width=0.6\linewidth]{fig1ans.eps} 
	\end{center}
	\caption{断面2次モーメントの計算に用いた部分断面.} 
	\label{fig:fig1}
\end{figure}
%--------------------
\end{document}
%--------------------
\begin{figure}[h]
	\begin{center}
	\includegraphics[width=0.60\linewidth]{fig2ans.eps} 
	\end{center}
	\caption{断面係数計算のための座標系.} 
	\label{fig:fig2}
\end{figure}
%--------------------
\end{document}
