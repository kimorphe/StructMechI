\documentclass[10pt,a4j]{jarticle}
\usepackage{graphicx,wrapfig}
\setlength{\topmargin}{-1.5cm}
\setlength{\textwidth}{15.5cm}
\setlength{\textheight}{25.2cm}
\newlength{\minitwocolumn}
\setlength{\minitwocolumn}{0.5\textwidth}
\addtolength{\minitwocolumn}{-\columnsep}
%\addtolength{\baselineskip}{-0.1\baselineskip}
%
\def\Mmaru#1{{\ooalign{\hfil#1\/\hfil\crcr
\raise.167ex\hbox{\mathhexbox 20D}}}}
%
\begin{document}
\newcommand{\fat}[1]{\mbox{\boldmath $#1$}}
\newcommand{\D}{\partial}
\newcommand{\w}{\omega}
\newcommand{\ga}{\alpha}
\newcommand{\gb}{\beta}
\newcommand{\gx}{\xi}
\newcommand{\gz}{\zeta}
\newcommand{\vhat}[1]{\hat{\fat{#1}}}
\newcommand{\spc}{\vspace{0.7\baselineskip}}
\newcommand{\halfspc}{\vspace{0.3\baselineskip}}
\bibliographystyle{unsrt}
\pagestyle{empty}
\newcommand{\twofig}[2]
 {
   \begin{figure}[h]
     \begin{minipage}[t]{\minitwocolumn}
         \begin{center}   #1
         \end{center}
     \end{minipage}
         \hspace{\columnsep}
     \begin{minipage}[t]{\minitwocolumn}
         \begin{center} #2
         \end{center}
     \end{minipage}
   \end{figure}
 }
%%%%%%%%%%%%%%%%%%%%%%%%%%%%%%%%%
%\vspace*{\baselineskip}
\begin{center}
{\Large \bf 2019年度 構造力学I及び演習B 演習問題4 解答} \\
\end{center}
%%%%%%%%%%%%%%%%%%%%%%%%%%%%%%%%%%%%%%%%%%%%%%%%%%%%%%%%%%%%%%%%
\subsubsection*{問題1}
断面$S$の面積$\left| S \right|$は, 
\begin{eqnarray}
	\left| S\right| &= & 
	\int _{y=-h/2}^{h/2}\int _{x=0}^{b\cos\left(\frac{\pi y}{h}\right) } dxdy
	\nonumber
	\\
	&=& 
	2\int_{0}^{h/2} b\cos \left( \frac{\pi y}{h}\right) dy 
	\nonumber
	\\
	&=& 
	\frac{2bh}{\pi}\left[ \sin \left( \frac{\pi y}{h} \right)\right]_0^{h/2}
	\nonumber
	\\
	&=&
	\frac{2bh}{\pi}
\end{eqnarray}
となる.一方,$x$軸に関する断面1次モーメント$G$は
\begin{eqnarray}
	G &=& 
	\int _{y=-h/2}^{h/2}\int_{x=0}^{b\cos\left(\frac{\pi y}{h}\right) } ydxdy \\
	 &=& 
	\int_{-h/2}^{h/2} by\cos \left( \frac{\pi y}{h}\right) dy 
\end{eqnarray}
だが,$y\cos\left(\frac{\pi y}{h}\right)$は$y$に関する奇関数だから$G=0$となることが分かる.
よって,中立軸 位置$\bar Y$は,$\bar Y=0$となる.
従って,$x$軸に関する断面2次モーメント$I_x$は中立軸に関する断面2次モーメントでもあり,それは
以下のようにして求めることができる.
\begin{eqnarray}
	I=I_x &= & 
	\int _{y=-h/2}^{h/2} \int _{x=0}^{b\cos \left(\frac{\pi y }{h}\right)} y^2dxdy   
	\nonumber \\
	&=& 
	2\int _{y=0}^{h/2} b y^2\cos \left(\frac{\pi y }{h}\right) dxdy   
	\nonumber \\
	&=& 
	\frac{2bh^3}{\pi^3}\int_0^{\pi/2}	
	Y^2 \cos Y dY, \ \ \left(Y=\frac{\pi}{h}y\right)
	\nonumber \\
	&=& 
	\frac{2bh^3}{\pi^3}
	\left[ \left(Y^2-2\right)\sin Y +2 Y\cos Y \right]_{0}^{\pi/2}
	\nonumber \\
	&=& 
	\frac{bh^3}{\pi^3}\left( \frac{\pi^2}{2}-4 \right)
\end{eqnarray}
なお,上の計算では,部分積分によって導かれる次の関係を利用した.
\begin{equation}
	\int x^2 \cos x dx = (x^2-2)\sin x +2x\cos x
	\label{eqn:int_idt}
\end{equation}
%%%%%%%%%%%%%%%%%%%%%%%%%%%%%%%%%%%%%%%%%%%%%%%%%%%%%%%%%%%%%%%%%%%%%%%%%%%%
%%%%%%%%%%%%%%%%%%%%%%%%%%%%%%%%%%%%%%%%%%%%%%%%%%%%%%%%%%%%%%%%%%%%%%%%%%%%
\subsubsection*{問題2}
部分断面$S_1$自体の中立軸は,$S_1$の形状上下対称であることから
図\ref{fig:fig1}-(b)の$X$軸に一致する.そのため,$S_1$の中立軸に関する
断面2次モーメント$I(S_1)$は,図\ref{fig:fig1}-(a)の$X$軸に関する断面2次モーメント
を$b=\frac{a}{2},\,h=\frac{\sqrt{3}}{2}a$として,4倍することで
\begin{equation}
	I(S_1) =  4\times \left. \frac{1}{12}bh^3 \right|_{b=\frac{a}{2},h=\frac{\sqrt{3}}{2}a}=\frac{\sqrt{3}}{16}a^4
	\label{eqn:eq_lbl}
\end{equation}
と得られる.
一方,部分断面$S_2$の中立軸位置は,図\ref{fig:fig1}-(c)の$X$軸に一致し,
$S_2$の中立軸に関する断面2次モーメント$I(S_2)$は,
\begin{equation}
	I(S_2)= 
		\frac{a}{12} \times \left( \frac{\sqrt{3}}{2}a \right)^3=\frac{\sqrt{3}}{32}a^4
	\label{eqn:eq_lbl}
\end{equation}
以上のことと,$S_2$と$S_3$の中立軸位置と部分断面の中立軸に関する断面2次モーメントは互いに一致することを踏まえれば,
表\ref{tbl:tbl2}に示すような計算過程を経て,複合断面$S=S_1\cup S_2 \cup S_3$の中立軸位置$\bar Y$と
断面2次モーメント$I(S)$が,次のように求められる.
\begin{equation}
	\bar Y = \sum_{i=1}^3 \frac{\left|S_i\right|}{\left| S \right|}\bar{Y}_i=0
	\label{eqn:Yb2}
\end{equation}
\begin{equation}
	I(S)=\sum_{i=1}^3\left\{
		I(S_i)+\left( \bar Y_i -\bar Y\right)^2\left| S_i \right|
	\right\}
		=\frac{11\sqrt{3}}{16}a^4
	\label{eqn:Iz_2}
\end{equation}
\begin{table}
\begin{center}
	\caption{部分断面への分割に基づく断面係数計算の過程}
	\begin{tabular}{c||c|c|c|c|c|c|c}
		&
		$\left| S_i \right|$ & 
		$ \xi_i=\frac{\left| S_i \right|}{\left| S\right|} $  &
		$ \bar{Y}_i $ & 
		$ \xi_i\bar{Y}_i $ & 
		$\bar{Y}_i -\bar Y$ & 
		$ \left(\bar{Y}_i -\bar Y\right)^2\left| S_i \right|$ & 
		$ I(S_i)$  
		\\
		\hline 
		\hline 
		断面1&	
		$\frac{\sqrt{3}}{2}a^2$ & 
		$\frac{1}{3}$  &
		$\frac{\sqrt{3}}{2}a$ & 
		$\frac{\sqrt{3}}{6}a$ & 
		$\frac{\sqrt{3}}{2}a$ & 
		$\frac{3\sqrt{3}}{8}a^4$ &
		$\frac{\sqrt{3}}{16}a^4$ 
		\\
		\hline
		断面2&	
		$\frac{\sqrt{3}}{2}a^2$ & 
		$\frac{1}{3}$  &
		$-\frac{\sqrt{3}}{4}a$ & 
		$-\frac{\sqrt{3}}{12}a $ & 
		$-\frac{\sqrt{3}}{4}a$ & 
		$\frac{3\sqrt{3}}{32}a^4$ &
		$\frac{\sqrt{3}}{32}a^4$ 
		\\
		\hline 
		断面3&	
		$\frac{\sqrt{3}}{2}a^2$ & 
		$\frac{1}{3}$  &
		$-\frac{\sqrt{3}}{4}a$ & 
		$-\frac{\sqrt{3}}{12}a $ & 
		$-\frac{\sqrt{3}}{4}a$ & 
		$\frac{3\sqrt{3}}{32}a^4$ &
		$\frac{\sqrt{3}}{32}a^4$ 
		\\
		\hline 
		\hline 
		合計&	
		$\frac{3\sqrt{3}}{2}a^2$ & 
		$1$  &
		$-$ & 
		$\bar Y=0 $ & 
		$-$ & 
		$\frac{9\sqrt{3}}{12}a^4$ & 
		$\frac{\sqrt{3}}{8}a^4$ 
	\end{tabular}
\label{tbl:tbl2}
\end{center}
\end{table}
%%%%%%%%%%%%%%%%%%%%%%%%%%%%%%%%%%%%%%%%%%%%%%%%%%%%%%%%%%%%%%%%%%%%%%%%%%%%
%%%%%%%%%%%%%%%%%%%%%%%%%%%%%%%%%%%%%%%%%%%%%%%%%%%%%%%%%%%%%%%%%%%%%%%%%%%%
%--------------------
\begin{figure}[h]
	\begin{center}
	\includegraphics[width=0.8\linewidth]{fig1ans.eps} 
	\end{center}
	\caption{断面2次モーメントの計算に用いた部分断面.} 
	\label{fig:fig1}
\end{figure}
%--------------------
\end{document}
%--------------------
