\documentclass[10pt,a4j]{jarticle}
\usepackage{graphicx,wrapfig}
\setlength{\topmargin}{-1.5cm}
\setlength{\textwidth}{15.5cm}
\setlength{\textheight}{25.2cm}
\newlength{\minitwocolumn}
\setlength{\minitwocolumn}{0.5\textwidth}
\addtolength{\minitwocolumn}{-\columnsep}
%\addtolength{\baselineskip}{-0.1\baselineskip}
%
\def\Mmaru#1{{\ooalign{\hfil#1\/\hfil\crcr
\raise.167ex\hbox{\mathhexbox 20D}}}}
%
\begin{document}
\newcommand{\fat}[1]{\mbox{\boldmath $#1$}}
\newcommand{\D}{\partial}
\newcommand{\w}{\omega}
\newcommand{\ga}{\alpha}
\newcommand{\gb}{\beta}
\newcommand{\gx}{\xi}
\newcommand{\gz}{\zeta}
\newcommand{\vhat}[1]{\hat{\fat{#1}}}
\newcommand{\spc}{\vspace{0.7\baselineskip}}
\newcommand{\halfspc}{\vspace{0.3\baselineskip}}
\bibliographystyle{unsrt}
\pagestyle{empty}
\newcommand{\twofig}[2]
 {
   \begin{figure}[h]
     \begin{minipage}[t]{\minitwocolumn}
         \begin{center}   #1
         \end{center}
     \end{minipage}
         \hspace{\columnsep}
     \begin{minipage}[t]{\minitwocolumn}
         \begin{center} #2
         \end{center}
     \end{minipage}
   \end{figure}
 }
%%%%%%%%%%%%%%%%%%%%%%%%%%%%%%%%%
%\vspace*{\baselineskip}
\begin{center}
{\Large \bf 2019年度 構造力学I及び演習B (1月24日) 演習問題6} \\
\end{center}
%%%%%%%%%%%%%%%%%%%%%%%%%%%%%%%%%%%%%%%%%%%%%%%%%%%%%%%%%%%%%%%%
\subsubsection*{問題1.}
図\ref{fig:fig1}に示すような片端が固定された梁の,支点Bにおける
鉛直反力の影響線を描け.
%--------------------
\begin{figure}[h]
	\begin{center}
	\includegraphics[width=0.55\linewidth]{fig1.eps} 
	\end{center}
	\caption{部材中央でローラー支持された片端固定梁.} 
	\label{fig:fig1}
\end{figure}
%
\subsubsection*{問題2.}
図\ref{fig:fig2}のような梁について以下の問に答えよ.
なお,梁の曲げ剛性$EI$は全断面で一定とし, 以下で言う"最大値"とは, 
絶対値としての最大値を意味する.
\begin{enumerate}
\item
	ヒンジBとCで伝達される鉛直方向の力の大きさを答えよ.
\item
	支点反力を求めよ.
\item
	曲げモーメント図を描け.
\item
	曲げモーメントの最大値$M_{max}$とそれを与える断面位置を求めよ.
\end{enumerate}
梁の断面が半径$a$の円であるとして以下の問に答えよ.
\begin{enumerate}
\setcounter{enumi}{2}
\item
	曲げ応力の最大値$\sigma_{max}$を求めよ.
\item
	曲げ応力の最大値が$\sigma_a$を超えないように梁を設計したい.
	すなわち
	\[
		\sigma_{max} \leq \sigma_a
	\]
	となるように断面寸法を決定する.このとき,
	許容される断面半径$a$の最小値を求めよ.
\end{enumerate}
%--------------------
\begin{figure}[h]
	\begin{center}
	\includegraphics[width=0.8\linewidth]{fig2.eps} 
	\end{center}
	\caption{区間ACに等分布荷重を受ける梁.} 
	\label{fig:fig2}
\end{figure}
\end{document}
