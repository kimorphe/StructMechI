\documentclass[10pt,a4j]{jarticle}
\usepackage{graphicx,wrapfig}
\setlength{\topmargin}{-1.5cm}
\setlength{\textwidth}{15.5cm}
\setlength{\textheight}{25.2cm}
\newlength{\minitwocolumn}
\setlength{\minitwocolumn}{0.5\textwidth}
\addtolength{\minitwocolumn}{-\columnsep}
%\addtolength{\baselineskip}{-0.1\baselineskip}
%
\def\Mmaru#1{{\ooalign{\hfil#1\/\hfil\crcr
\raise.167ex\hbox{\mathhexbox 20D}}}}
%
\begin{document}
\newcommand{\fat}[1]{\mbox{\boldmath $#1$}}
\newcommand{\D}{\partial}
\newcommand{\w}{\omega}
\newcommand{\ga}{\alpha}
\newcommand{\gb}{\beta}
\newcommand{\gx}{\xi}
\newcommand{\gz}{\zeta}
\newcommand{\vhat}[1]{\hat{\fat{#1}}}
\newcommand{\spc}{\vspace{0.7\baselineskip}}
\newcommand{\halfspc}{\vspace{0.3\baselineskip}}
\bibliographystyle{unsrt}
\pagestyle{empty}
\newcommand{\twofig}[2]
 {
   \begin{figure}[h]
     \begin{minipage}[t]{\minitwocolumn}
         \begin{center}   #1
         \end{center}
     \end{minipage}
         \hspace{\columnsep}
     \begin{minipage}[t]{\minitwocolumn}
         \begin{center} #2
         \end{center}
     \end{minipage}
   \end{figure}
 }
%%%%%%%%%%%%%%%%%%%%%%%%%%%%%%%%%
%\vspace*{\baselineskip}
\begin{center}
{\Large \bf 2018年度 構造力学I及び演習B (1月25日) 演習問題6 解答} \\
\end{center}
%%%%%%%%%%%%%%%%%%%%%%%%%%%%%%%%%%%%%%%%%%%%%%%%%%%%%%%%%%%%%%%%
\subsubsection*{問題1.}
図\ref{fig:fig1}-(a)に示すように,区間BC内に単位荷重が加えられたとき,
支点部Bに発生する曲げモーメントを求める.そのために,(a)に示した構造を,
同図の(b)と(c)のような2つの静定構造の重ね合わせで表現する.
ここで,$R_B$は,系(a)の点Bにおいて発生する鉛直方向の支点反力を表す.
はじめに,系(c)のたわみ$v_c$を,たわみの微分方程式を単純支持条件の元で
解くことによって求めると,その結果は$\alpha=a/l,\, \beta=b/l, \, \xi=x/l$として,
\begin{equation}
 v_c=\frac{\tilde Pl^3}{6EI}\left\{
 \left< \xi-\alpha \right>^3 -\beta \xi^3 +\beta (1-\beta^2)\xi 
 \right\}
 \label{eqn:vc}
\end{equation}
と表される.一方,系(b)のたわみ$v_b$は,式(\ref{eqn:vc})において$\tilde P=-R_B, a=b=\frac{l}{2}$と
置くことで,
\begin{equation}
 v_b=\frac{(-R_B)l^3}{6EI}\left\{
		\left< \xi-\frac{1}{2} \right>^3 -\frac{1}{2} \xi^3 +\frac{3}{8}\xi 
	\right\}
	\label{eqn:vb}
\end{equation}
と得られる.以上の結果を,系(a)の支点Bにおける拘束条件:
\begin{equation}
	v_b\left(\frac{l}{2}\right) 
	+
	v_c\left(\frac{l}{2}\right) =0
\end{equation}
に代入すれば,未知の支点反力$R_B$が
\begin{equation}
	R_B=(3\beta-4\beta^3)\tilde P
	\label{eqn:RB}
\end{equation}
と求めれられる.系(a)の支点AとCにおける反力は,$R_B$が既知となったため,釣り合い条件
から決定することができ,その結果は
\begin{equation}
	R_A=\beta \tilde P -\frac{R_B}{2}=\left(2\beta^3-\frac{1}{2}\beta \right)\tilde P, \ \ 
	R_C=\alpha \tilde P -\frac{R_B}{2}=\left(2\beta^3-\frac{3}{2}\beta +\alpha \right)\tilde P, \ \ 
\end{equation}
となる.また,点Bにおける曲げモーメント$M_B$も釣り合い条件から求まり,
\begin{equation}
	M_B=R_A\times\frac{l}{2}=\frac{\tilde Pl}{4} \beta (4\beta^2-1)
\end{equation}
となる.$\alpha \in (1/2,1), \, \beta \in (0,1)$なので,以上により,影響線の右半分(区間BC)
が得られたことになる.単位荷重が区間AC内にあるときも,同様にして影響線を計算することができる. 
しかしながらこの問題では,曲げに関する支持条件の対称性から,影響線は点Bに関して左右対照となる
ことがわかるので,上で求めた結果に加えてこのことを踏まえれば,
左半分の区間について実際に計算を行うことなく影響線が
図\ref{fig:fig2}のようになることが分かる.
%%--------------------
\begin{figure}[h]
	\begin{center}
	\includegraphics[width=0.55\linewidth]{fig1ans.eps}
	\end{center}
	\caption{問題1の解答に用いた3つの系.}
	\label{fig:fig1}
\end{figure}
%--------------------
\begin{figure}[h]
	\begin{center}
	\includegraphics[width=0.7\linewidth]{fig2ans.eps} 
	\end{center}
	\caption{支点部Bにおける曲げモーメント$M_B$の影響線}
	\label{fig:fig2}
\end{figure}
%%%%%%%%%%%%%%%%%%%%%%%%%%%%%%%%%%%%%%%%%%%%%%%%%%%%%%%%%%%%%%%%
%%
%
%
\subsubsection*{問題2.}
\begin{enumerate}
\item
支点A,C,D,Fにおける鉛直反力と,ヒンジ部BとEにおいて伝達される鉛直力の
正方向を各々図\ref{fig:fig3}-(a)のように定める.
これらの鉛直力は,区間AB,BE,EFそれぞれにおいて力とモーメントの釣り合いを考える
ことで,以下のように決定される.
\begin{equation}
	R_A=R_B=\frac{q_0l}{4}, \ \ 
	R_C=q_0l, \ \ R_D=-\frac{q_0l}{4}, \ \ R_E=R_F=0
\end{equation}
\item
区間AB,BE,EFの単位で構造を見ると,すべて静定構造となっているため,区間毎に釣り合い条件を適用して
曲げモーメントを求めることができる.その結果を曲げモーメント図として示せば,図\ref{fig:fig4}のように
なる.この図には,曲げモーメント分布は連続で,等分布荷重が加わる区間は2次関数的に,
荷重が作用しない区間では直線的に曲げモーメント値が変化することが示されていることに注意する.
\item
図\ref{fig:fig4}より,曲げモーメントは点Cにおいて最大値をとる.
\begin{equation}
	M_{max}=\left|-\frac{q_0l^2}{4}\right| = \frac{q_0l^2}{4}
\end{equation}
\item
曲げ応力$\sigma=\frac{M}{I}y$は,曲げモーメントが最大となる断面の上縁$y=-a$あるいは下縁$y=a$において最大となる.
また,半径$a$の円断面の中立軸に関する断面2次モーメント$I$は
\begin{equation}
	I=\frac{\pi a^4}{4}	
\end{equation}
だから,
\begin{equation}
	\sigma_{max}=\frac{M_{max}}{I}a=\frac{q_0l^2}{\pi a^3}
\end{equation}
となる.
\item
上で求めた$\sigma_{max}$を不等式$\sigma_{max}<\sigma_a$に代入し,$a$に関して
整理すれば,
\begin{equation}
	a \geq \sqrt[3]{\frac{q_0l^2}{\pi \sigma_a}} 
\end{equation}
となるので,許容される$a$の最小値は$\sqrt[3]{\frac{q_0l^2}{\pi \sigma_a}}$である.
\end{enumerate}
%--------------------
\begin{figure}[h]
	\begin{center}
	\includegraphics[width=0.7\linewidth]{fig3ans.eps} 
	\end{center}
	\caption{
		(a)支点反力の正方向と切断面二つの位置a-a', b-b'. 
		(b) 曲げモーメント図.} 
	\label{fig:fig3}
\end{figure}
\begin{figure}[h]
	\begin{center}
	\includegraphics[width=0.7\linewidth]{fig4ans.eps} 
	\end{center}
	\caption{
		(a)支点反力の正方向と切断面二つの位置a-a', b-b'. 
		(b) 曲げモーメント図.} 
	\label{fig:fig4}
\end{figure}
\end{document}
