\documentclass[10pt,a4j]{jarticle}
\usepackage{graphicx,wrapfig}
\setlength{\topmargin}{-1.5cm}
\setlength{\textwidth}{15.5cm}
\setlength{\textheight}{25.2cm}
\newlength{\minitwocolumn}
\setlength{\minitwocolumn}{0.5\textwidth}
\addtolength{\minitwocolumn}{-\columnsep}
%\addtolength{\baselineskip}{-0.1\baselineskip}
%
\def\Mmaru#1{{\ooalign{\hfil#1\/\hfil\crcr
\raise.167ex\hbox{\mathhexbox 20D}}}}
%
\begin{document}
\newcommand{\fat}[1]{\mbox{\boldmath $#1$}}
\newcommand{\D}{\partial}
\newcommand{\w}{\omega}
\newcommand{\ga}{\alpha}
\newcommand{\gb}{\beta}
\newcommand{\gx}{\xi}
\newcommand{\gz}{\zeta}
\newcommand{\vhat}[1]{\hat{\fat{#1}}}
\newcommand{\spc}{\vspace{0.7\baselineskip}}
\newcommand{\halfspc}{\vspace{0.3\baselineskip}}
\bibliographystyle{unsrt}
\pagestyle{empty}
\newcommand{\twofig}[2]
 {
   \begin{figure}[h]
     \begin{minipage}[t]{\minitwocolumn}
         \begin{center}   #1
         \end{center}
     \end{minipage}
         \hspace{\columnsep}
     \begin{minipage}[t]{\minitwocolumn}
         \begin{center} #2
         \end{center}
     \end{minipage}
   \end{figure}
 }
%%%%%%%%%%%%%%%%%%%%%%%%%%%%%%%%%
%\vspace*{\baselineskip}
\begin{center}
{\Large \bf 2019年度 構造力学I及び演習B 演習問題6 解答} \\
\end{center}
%%%%%%%%%%%%%%%%%%%%%%%%%%%%%%%%%%%%%%%%%%%%%%%%%%%%%%%%%%%%%%%%
\subsubsection*{問題1.}
図\ref{fig:fig1}-(a)に示すように$x$座標をとり,$x=a$の位置に単位荷重が
加えられたときに,支点Bに発生する鉛直反力$R_B$を求める.
そのために,(a)に示した構造を,図\ref{fig:fig1}の(b)と(c)のような2つの静定構造の重ね合わせで表現する.
はじめに,系(c)のたわみ$v_c$を,たわみの微分方程式を単純支持条件の元で
解くことによって求めると,その結果は$\alpha=a/l,\, \beta=b/l, \, \xi=x/l$として,
\begin{equation}
 v_c=\frac{\tilde Pl^3}{6EI}\left\{
 \left< \xi-\alpha \right>^3 -\xi^3 +3\alpha \xi 
 \right\}
 \label{eqn:vc}
\end{equation}
と表される.一方,系(b)のたわみ$v_b$は,式(\ref{eqn:vc})において$\tilde P=-R_B, a=b=\frac{l}{2}$と
置くことで,
\begin{equation}
 v_b=\frac{(-R_B)l^3}{6EI}\left\{
		\left< \xi-\frac{1}{2} \right>^3 -\xi^3 +\frac{3}{2}\xi 
	\right\}
	\label{eqn:vb}
\end{equation}
と得られる.以上の結果を,系(a)の支点Bにおける拘束条件:
\begin{equation}
	v_b\left(\frac{l}{2}\right) 
	+
	v_c\left(\frac{l}{2}\right) =0
\end{equation}
に代入すれば,未知の支点反力$R_B$が
\begin{equation}
	R_B=
	\tilde P
	\left\{
		\left< \frac{1}{2}-\alpha \right>^3-\frac{1}{2}+3\alpha
	\right\}
	=\left\{
	\begin{array}{cc}
		2\alpha^2(3-2\alpha)\tilde P & \left(\alpha < \frac{1}{2}\right) \\
		\left(3\alpha-\frac{1}{2}\right)\tilde P & \left(\alpha > \frac{1}{2}\right)
	\end{array}
	\right.
	\label{eqn:RB}
\end{equation}
と求めれられる.
この結果を$a$の関数として図示すれば,$R_B$の影響線が図\ref{fig:fig2}のようになることが分かる.
%%--------------------
\begin{figure}[h]
	\begin{center}
	\includegraphics[width=0.55\linewidth]{fig1ans.eps}
	\end{center}
	\caption{問題1の解答に用いた3つの系.}
	\label{fig:fig1}
\end{figure}
%--------------------
\begin{figure}[h]
	\begin{center}
	\includegraphics[width=0.7\linewidth]{fig2ans.eps} 
	\end{center}
	\caption{支点部Bにおける鉛直反力$R_B$の影響線.}
	\label{fig:fig2}
\end{figure}
%%%%%%%%%%%%%%%%%%%%%%%%%%%%%%%%%%%%%%%%%%%%%%%%%%%%%%%%%%%%%%%%
%%
%
%
\subsubsection*{問題2.}
\begin{enumerate}
\item
支点AとEにおける鉛直反力と,ヒンジ部BとDにおいて伝達される鉛直力の正方向をそれぞれ図\ref{fig:fig3}-(a)のように定める.
これらの鉛直力は,区間AB,BD,DEのそれぞれの区間における力とモーメントの釣り合いを考えることで決定できる.
はじめに,区間BDにおける釣り合い条件から,
\begin{equation}
	R_B=\frac{3}{8}q_0l, \ \ R_D=\frac{1}{8}q_0l
\end{equation}
が言える.
\item
問1の結果を踏まえて区間ABとDEの釣り合い条件を用いれば,両固定端における反力が以下のように求められる.
\begin{equation}
	R_A=\frac{7}{8}q_0l, \ \ M_A=-\frac{5}{16}q_0l^2  
\end{equation}
\begin{equation}
	R_E=\frac{1}{8}q_0l, \ \ M_E=-\frac{1}{16}q_0l^2
\end{equation}
\item
区間AB,BD,DEの単位で構造を見ると,すべて静定構造となっているため,区間毎に釣り合い条件を適用して
曲げモーメントを求めることができる.その結果を曲げモーメント図として示せば,図\ref{fig:fig4}のように
なる.この図には,曲げモーメント分布は連続で,等分布荷重が加わる区間は2次関数的に,
荷重が作用しない区間では直線的に曲げモーメント値が変化することが示されている.
\item
図\ref{fig:fig4}より,曲げモーメントは点A($x=0$)において最大値をとる.
\begin{equation}
	M_{max}=\left|-\frac{5}{16}q_0l^2\right| = \frac{5}{16}q_0l^2
\end{equation}
\item
曲げ応力$\sigma=\frac{M}{I}y$は,曲げモーメントが最大となる断面の上縁$y=-a$あるいは下縁$y=a$において最大となる.
また,半径$a$の円断面の中立軸に関する断面2次モーメント$I$は
\begin{equation}
	I=\frac{\pi a^4}{4}	
\end{equation}
だから,
\begin{equation}
	\sigma_{max}=
	\frac{M_{max}}{I}a=
	\frac{5}{4} \frac{q_0l^2}{\pi a^3}
\end{equation}
となる.
\item
上で求めた$\sigma_{max}$を不等式$\sigma_{max}<\sigma_a$に代入し,$a$に関して
整理すれば,
\begin{equation}
	a \geq \sqrt[3]{\frac{5}{4\pi}\frac{q_0l^2}{\sigma_a}} 
\end{equation}
		となるので,許容される$a$の最小値は$\sqrt[3]{\frac{5}{4\pi}\frac{q_0l^2}{\sigma_a}}$である.
\end{enumerate}
%--------------------
\begin{figure}[h]
	\begin{center}
	\includegraphics[width=0.7\linewidth]{fig3ans.eps} 
	\end{center}
	\caption{
		(a)支点反力の正方向と切断面二つの位置a-a', b-b'. 
		(b) 曲げモーメント図.} 
	\label{fig:fig3}
\end{figure}
\begin{figure}[h]
	\begin{center}
	\includegraphics[width=0.7\linewidth]{fig4ans.eps} 
	\end{center}
	\caption{
		(a)支点反力の正方向と切断面二つの位置a-a', b-b'. 
		(b) 曲げモーメント図.} 
	\label{fig:fig4}
\end{figure}
\end{document}
